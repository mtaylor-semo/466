%!TEX TS-program = lualatex
%!TEX encoding = UTF-8 Unicode

\documentclass[12pt, addpoints]{exam}

%\printanswers

\usepackage[left=1in,right=0.75in,top=1in]{geometry}     

\usepackage{fontspec}
\def\mainfont{Linux Libertine O}
\setmainfont[Ligatures={TeX, Common}, BoldFont={* Bold}, ItalicFont={* Italic}, Numbers={Proportional, OldStyle}]{\mainfont}
%\setmonofont[Scale=MatchLowercase]{Inconsolata} 
\setsansfont[Scale=MatchLowercase]{Linux Biolinum O} 
\usepackage{microtype}
\usepackage[parfill]{parskip} 

% This defines \amper for the fancy ampersand
% to be used in the header. See
% https://tex.stackexchange.com/a/58185/39194
\usepackage{xspace}
\newfontfamily\amperfont[Style=Alternate]{Linux Libertine O}
\makeatletter
\DeclareRobustCommand{\amper}{{\amperfont\ifx\f@shape\scname\smaller[1.2]\fi\&}\xspace}
\makeatother

\usepackage{graphicx}
	\graphicspath{%
	{/Users/goby/Pictures/teach/466/exams/}}%

% Used to get the last two digits of the year for the header below.
\def\short#1{\csname @gobbletwo\expandafter\endcsname\number#1}

\pagestyle{headandfoot}
\firstpageheader{Ornithology, Field \& Taxonomy Quiz, \textsc{s}\short{\year}, \numpoints~points.}{}{%
	\ifprintanswers\textbf{KEY}\else Name: \enspace \makebox[2.5in]{\hrulefill}\fi}
\runningheader{}{}{\small{pg. \thepage}}
\runningheadrule


\footer{\iflastpage{\small \rule{0.6in}{0.4pt} / \numpoints\ points}{}}{}{\makebox[1in]{\hrulefill}}%{\makebox[0.75in]{\hrulefill}}
\extrafootheight{-0.5in}


\pointsinmargin
\marginpointname{ pts}

\usepackage{multicol}
\usepackage{booktabs}
\usepackage{array}
\newcolumntype{L}[1]{>{\raggedright\let\newline\\\arraybackslash\hspace{0pt}}p{#1}}
\newcolumntype{C}[1]{>{\centering\let\newline\\\arraybackslash\hspace{0pt}}p{#1}}
\newcolumntype{R}[1]{>{\raggedleft\let\newline\\\arraybackslash\hspace{0pt}}p{#1}}

%% Remove comment %% to print answer key.
\unframedsolutions
\renewcommand{\solutiontitle}{}
\SolutionEmphasis{\bfseries}

%% Create a Matching question format
\newcommand*\Matching[1]{
\ifprintanswers
	\textbf{#1}
\else
	\rule{2in}{0.5pt}
\fi
}
\newlength\matchlena
\newlength\matchlenb
\settowidth\matchlena{\rule{2.1in}{0pt}}
\newcommand\MatchQuestion[2]{%
	\setlength\matchlenb{0.98\linewidth}
	\addtolength\matchlenb{-\matchlena}
	\parbox[t]{\matchlena}{\Matching{#1}}\enspace\parbox[t]{\matchlenb}{#2}}

%\usepackage{enumitem}
%\setlist{leftmargin=*}
%\setlist[1]{labelindent=\parindent}
%\setlist[enumerate]{label=\textsc{\alph*}., ref=\textsc{\alph*}}

\newcommand*\AnswerBox[2]{%
    \parbox[t][#1]{0.92\textwidth}{%
    \begin{solution}#2\end{solution}}
    \vspace{\stretch{1}}
}

\newenvironment{AnswerPage}[1]
    {\begin{minipage}[t][#1]{0.92\textwidth}%
    \begin{solution}}
    {\end{solution}\end{minipage}
    \vspace{\stretch{1}}}

\newlength{\basespace}
\setlength{\basespace}{5\baselineskip}

\newcommand{\bumppoints}[1]{%
	\addtocounter{numpoints}{#1}
}

\newcommand{\qblank[1]}{\ifprintanswers \textbf{#1} \else \rule{2.5in}{0.4pt} \fi}

\newcommand{\bqblank[1]}{\ifprintanswers \textbf{#1~(e.c.)} \else \rule{2.5in}{0.4pt}~(e.c.) \fi}

\begin{document}

Orders and families from taxonomy lectures 5–7. Some field quiz questions will ask you to name a taxonomic level. The answers are included below.


\begin{multicols}{3}
	Cardinalidae \\
	Corvidae \\
	Fringillidae \\
	Hirudinidae \\
	Icteridae \\
	Mimidae \\
	Paridae \\
	Parulidae \\
	Passeridae \\
	Passerellidae \\
	Passeriformes \\
	Picidae \\
	Piciformes \\
	Sittidae \\
	Sturnidae \\
	Thraupidae \\
	Troglodytidae \\
	Turdidae \\
	Tyrannidae \\
	Vireonidae 
\end{multicols}

\begin{questions}

\bumppoints{46}

\vspace{\baselineskip}

\begin{multicols}{2}
\question
\qblank[Wood Thrush]

\bigskip

\question
\qblank[Canada Goose]

\bigskip

\question
\qblank[White-eyed Vireo]

\bigskip

\question
\qblank[Northern Mockingbird]

\bigskip

\question
\qblank[Mimidae]

\bigskip

\question
\qblank[Red-winged Blackbird]

\bigskip

\question
\qblank[Eastern Bluebird]

\bigskip

\question
\qblank[Turdidae]

\bigskip

\question
\qblank[Red-belled Woodpecker]

\bigskip

\question
\qblank[Lark Sparrow]

\bigskip

\question
\qblank[Passerellidae]

%\bigskip

\question
\qblank[Hirudinidae]

\bigskip

\question
\qblank[Carolina Chickaeee]

\bigskip

\question
\qblank[Paridae]

\bigskip

\question
\qblank[Eastern Wood Pewee]

\bigskip

\question
\qblank[Red-winged Blackbird]

\bigskip

\question
\qblank[Icteridae]

\bigskip

\question
\qblank[Piciformes]

\bigskip

%\question
%\qblank[Female]

\question
\qblank[American Robin]

\bigskip

\question
\qblank[American Crow]

\bigskip

\question
\qblank[Corvidae]

\bigskip

\question
\bqblank[Blue-gray gnatcatcher]

\end{multicols}


\vspace{1\baselineskip}

\question\MatchQuestion{Cardinalidae/Thraupidae}{Name the family of tanagers and Northern Cardinal.}

\vspace{0.7\baselineskip}

\question\MatchQuestion{Parulidae}{Name the family of warblers, such as the Common Yellowthroat.}

\end{questions}

\end{document}