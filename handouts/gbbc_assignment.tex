%!TEX TS-program = lualatex
%!TEX encoding = UTF-8 Unicode

\documentclass[11pt]{article}
%\usepackage{graphicx}
%	\graphicspath{{/Users/goby/Pictures/teach/153/lab/}} % set of paths to search for images

\usepackage{geometry}
\geometry{letterpaper}                   
\geometry{bottom=1in}
%\geometry{landscape}                % Activate for for rotated page geometry
\usepackage[parfill]{parskip}    % Activate to begin paragraphs with an empty line rather than an indent
%\usepackage{amssymb}
%\usepackage{mathtools}
%	\everymath{\displaystyle}

\pagenumbering{gobble}

\usepackage{fontspec}
\setmainfont[Ligatures={Common,TeX}, BoldFont={* Bold}, ItalicFont={* Italic}, Numbers={Proportional}]{Linux Libertine O}
\setsansfont[Scale=MatchLowercase,Ligatures=TeX]{Linux Biolinum O}
%\setmonofont[Scale=MatchLowercase]{Inconsolata}
\usepackage{microtype}

\usepackage{unicode-math}
\setmathfont[Scale=MatchLowercase]{Asana-Math.otf}
%\setmathfont{XITS Math}

% To define fonts for particular uses within a document. For example, 
% This sets the Libertine font to use tabular number format for tables.
%\newfontfamily{\tablenumbers}[Numbers={Monospaced}]{Linux Libertine O}
%\newfontfamily{\libertinedisplay}{Linux Libertine Display O}


\usepackage{booktabs}
\usepackage{multicol}
%\usepackage{longtable}
%\usepackage{siunitx}
%\usepackage[justification=raggedright, singlelinecheck=off]{caption}
%\captionsetup{labelsep=period} % Removes colon following figure / table number.
%\captionsetup{tablewithin=none}  % Sequential numbering of tables and figures instead of
%\captionsetup{figurewithin=none} % resetting numbers within each chapter (Intro, M&M, etc.)
%\captionsetup[table]{skip=0pt}

\usepackage{array}
\newcolumntype{L}[1]{>{\raggedright\let\newline\\\arraybackslash\hspace{0pt}}p{#1}}
\newcolumntype{C}[1]{>{\centering\let\newline\\\arraybackslash\hspace{0pt}}p{#1}}
\newcolumntype{R}[1]{>{\raggedleft\let\newline\\\arraybackslash\hspace{0pt}}p{#1}}

\usepackage{enumitem}
\setlist{leftmargin=*}
\setlist[1]{labelindent=\parindent}

\usepackage[sc]{titlesec}

\usepackage[hidelinks]{hyperref}
%\usepackage{placeins} %PRovides \FloatBarrier to flush all floats before a certain point.

%\usepackage{titling}
%\setlength{\droptitle}{-60pt}
%\posttitle{\par\end{center}}
%\predate{}\postdate{}

\usepackage{fancyhdr}
\fancyhf{}
\pagestyle{fancy}
\setlength{\headheight}{13.6pt}
\lhead{}
\chead{}
\rhead{\footnotesize pg.~\thepage }
\renewcommand{\headrulewidth}{0.4pt}

\fancypagestyle{plain}{%
	\fancyhf{}
	\lhead{\textsc{bi}~466/666: Ornithology}
	\rhead{Name: \enspace \makebox[2.5in]{\hrulefill}}
	\renewcommand{\headrulewidth}{0pt}
}

\newcommand{\VSpace}{\vspace{\baselineskip}}
\newcommand{\BigVSpace}{\vspace{2\baselineskip}}

\newcommand{\ProCon}[1]{\textit{\textsc{\small #1}}}

%\title{Great Backyard Bird Count}
%\author{Ornithology Assignment}
%\date{}                                           % Activate to display a given date or no date

\begin{document}
%\maketitle
\thispagestyle{plain}

\subsection*{Great backyard bird count (20 points)}

The Cornell Lab of Ornithology, in cooperation with the National Audubon Society, hosts the Great Backyard Bird Count \textsc{(gbbc)} every February. The \textsc{gbbc} depends on ``citizen scientists'' throughout North America to record the abundance of birds in their areas. The \textsc{gbbc} requirements are simple. They ask that participants count birds for at least 15 minutes from one or more locations on one or more days between 14--17 February. You submit your counts via eBird so they will be available for scientists globally to study bird populations, migrations patterns, and more.  For details, visit their website at \url{http://www.birdcount.org}.\medskip

I require that you participate in the \textsc{gbbc} as an assignment for this course, as outlined below.

\begin{enumerate}
	\item Go the the \textsc{gbbc} website at \url{http://www.birdcount.org}. Click on the “Get Started” tab near the top of the page.

	\item Click on the “Participate” tab at the upper right of the page, then choose “How to participate.” Read and follow the steps listed.
	
	%You will be asked to provide a user name and valid email address. You may use your SEMO email address or a personal email account, such as gmail.  After you have registered, you will receive a confirmation email. Register no later than Thursday, 17 February, 11:59 pm. \emph{Forward your email confirmation immediately} to Mike (\href{mailto:mtaylor@semo.edu}{mtaylor@semo.edu}).
	  
%	\item Scroll farther down the Get Started page and download the “\href{http://gbbc.birdcount.org/wp-content/uploads/2014/02/2014GBBC_DownloadableInstructions_NewHeader_time.pdf}{Downloadable Instructions}" PDF. Follow their instructions carefully to accurately identify the site where you counted birds.
	
%	\item Download the “\href{http://gbbc.birdcount.org/wp-content/uploads/2014/11/2015GBBC_DataForm.pdf}{Optional Data Form}” PDF. Use this form to count your birds.  Use a new form for each unique counting event you perform (see “Why count?” below).
	
%	\item Click on the “\href{http://gbbc.birdcount.org/help-faqs/}{Help \& FAQs}” link. Review the information.
	
	\item Count birds on \emph{at least one day for at least 30 minutes.} Although the \textsc{gbbc} requires you to count birds for at least 15 minutes, I require that you count birds for at least 30 minutes. See details below.

	\item Submit your count results via your eBird Mobile app. Share your submissions with “semo\_ornithology” when you submit your results to eBird. Submit your count results no later than 11:59~\textsc{pm} on the last day of the \textsc{gbbc}.
	
%	\item Submit your results to the GBBC website, following their instructions. When you submit your results, you will receive a confirmation page. Click the “Email Yourself” button at the top right of the confirmation page. \emph{Forward the email immediately} to Mike (\href{mailto:mtaylor@semo.edu}{mtaylor@semo.edu}). Copies of all submissions must be forwarded to Mike no later than 11:59 pm, 17 February or they will not be counted.
	
	\item You may and are encouraged to submit more than one count. Only one needs to be 30 minutes long.
    
\end{enumerate}

\subsection*{Do not make up data.}

eBird checklists submitted during the \textsc{gbbc} are used by scientists. These are real data and cannot be made up.  Do not lie about your time, effort, or species seen just to get the points for this assignment. I depend on your honor for this exercise so be honorable. You and I will be very displeased if I catch you cheating.

\subsection*{Why count?}
In addition to the scientific contribution, you earn points for your grade. You earn 20 points for your required counting event. I have defined a counting event as one location on one day for a minimum of 30 minutes. I encourage you to spend more time counting but you must spend at least 30 minutes counting.  You can earn an additional 12 points of extra credit by performing additional unique counting events on each day of the \textsc{gbbc}. For example, if you complete your first counting event on the first day, then you can perform another count on the next day, a count on the third day, and a count on the final day.  For the extra credit counts, points are awarded in blocks of 15 minutes. If you can count for only 15 minutes on one day (after you have completed your primary count), then you will earn 1 extra credit point. If you count 15 minutes on the second day and 30 minutes on the third day, you earn 3 extra credit points.  You can count for more than one hour during the additional counting events but you can earn no more than 4 extra credit points per extra day of counting.  Remember to ensure any one checklist is no longer than 60 minutes to follow eBird best practices. If, for example, you want to count for 90 minutes, submit two checklists, one for 30 minutes and one for 60 minutes (or two 45 minute checklists).

For each extra counting event you report, be sure you share the checklist with semo\_ornithology. Failure to do so means that you will not receive the extra credit.

\subsection*{What to count.}

Birds, of course!  However, be positive of your identification before you count it. Do not count a bird unless you are certain you have identified it correctly.  Use your field guide to help you identify the birds you see. This will be a great learning experience for you. Plus, you can earn bragging rights if you count more than your classmates.  If you can confidently narrow down the identification of an bird to a specific group such as blackbirds, \textsc{gbbc} will accept those data.

\subsection*{When to count.}

Counting begins at midnight (12:00 am) local time on the first day of the event and extends until 11:59 pm local time on the final day. Each day lasts from midnight until 11:59 pm. The best times to count is the early morning (around sunrise) followed by the evening (around sundown) because that is when bird activity is the greatest. If you start about 30--60 minutes before sunrise, you'll have a good chance of hearing and perhaps seeing owls. You can count during the midday but you might not see as many birds.

\subsection*{Where to count.}

You may count anywhere, including your backyard, around campus, or around town. I encourage you to visit a conservation area, state park, or wildlife refuge. You can find nearby eBird hotspots in eBird online. You'll see a greater variety of birds and you'll get to enjoy being outdoors. Most conservation areas and parks have one or more hiking trails. Hiking a trail is a good way to cover lots of habitat.  Nearby places include Kelso Wildlife Refuge, Judan Creek Conservation Area, the Missouri Department of Conservation Nature Center, Trail of Tears State Park, Horseshoe Lake Refuge (where we went for our second trip, 30 minutes away), or Mingo National Wildlife Refuge (about one hour away).  For a list of MDC conservation areas in the southeast region, click on the link near the top of the course website.

\subsection*{How to count.}

Count every bird you see. If you see four Northern Cardinals when you first start counting, then later see two more Northern Cardinals, you will report six Northern Cardinal. If you see large flocks, do your best to estimate the number of birds. You can round to the nearest 10 or 100. For large flocks, count a “block” of 10 birds, then estimate how many blocks are in the entire flock. That's an acceptable and reasonably accurate way to estimate the number of birds present in the flock. If you hike a trail, do not count birds if you walk back the same way you came unless you know you are seeing or hearing something you did not count earlier. This avoids artificially inflating the number by counting the same birds more than once.  If you hike a loop trail, count the entire loop.  

\subsection*{Who can count?}

You may count with a classmate but you can only submit one counting event per student. So, if you count with a friend, you will need to participate in more than one counting event. For example, if you and a classmate hike the trails by the MDC Nature Center for one hour,  you can submit that event for the \textsc{gbbc.} Then, you and your classmate can go to another place to count (one the same day or a different day) or you can go back to the Nature Center and count again later in the day or the next day (I recommend a different place for variety). Your classmate can report the second counting event.  \emph{Each student must submit a unique counting event.}  You can also bring along friends and family but remember that the focus is on accurately identifying and counting birds.

\end{document}  