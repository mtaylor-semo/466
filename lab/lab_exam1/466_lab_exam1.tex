%!TEX TS-program = lualatex
%!TEX encoding = UTF-8 Unicode

\documentclass[12pt, addpoints]{exam}

%\printanswers


\usepackage[left=1in,right=0.75in,top=1in]{geometry}     
\usepackage[parfill]{parskip}

\usepackage{fontspec}
\def\mainfont{Linux Libertine O}
\setmainfont[Ligatures={TeX, Common}, BoldFont={* Bold}, ItalicFont={* Italic}, Numbers={Proportional, OldStyle}]{\mainfont}
%\setmonofont[Scale=MatchLowercase]{Inconsolata} 
\setsansfont[Scale=MatchLowercase]{Linux Biolinum O} 
\usepackage{microtype}

% This defines \amper for the fancy ampersand
% to be used in the header. See
% https://tex.stackexchange.com/a/58185/39194
%\usepackage{xspace}
%\newfontfamily\amperfont[Style=Alternate]{Linux Libertine O}
%\makeatletter
%\DeclareRobustCommand{\amper}{{\amperfont\ifx\f@shape\scname\smaller[1.2]\fi\&}\xspace}
%\makeatother

\usepackage{graphicx}
	\graphicspath{%
	{/Users/goby/Pictures/teach/466/exams/}}%


% Used to get the last two digits of the year for the header below.
\def\short#1{\csname @gobbletwo\expandafter\endcsname\number#1}

\usepackage[sc]{titlesec}

\pagestyle{headandfoot}

\firstpageheader{Ornithology, Lab Exam 1, \textsc{s}\short{\year}, \numpoints~points.}{}{%
	\ifprintanswers\textbf{KEY}\else Name: \enspace \makebox[2.5in]{\hrulefill}\fi}
\runningheader{}{}{\small{pg. \thepage}}
\runningheadrule

\footer{\iflastpage{\small \rule{0.6in}{0.4pt} / \numpoints\ points}{}}{}{\rule{1in}{0.4pt}}%{\makebox[0.75in]{\hrulefill}}
\extrafootheight{-0.5in}


%\fancypagestyle{labexam}{
%\fancyfoot[L]{\small \rule{0.6in}{0.4pt} / \numpoints\ points}
%}
%\pagestyle{foot}
%\footer{\small \rule{0.6in}{0.4pt} / 25~points}{}{}


%\fancypagestyle{alim}{\fancyhf{}\renewcommand{\headrulewidth}{0pt}\fancyfoot[R]{Alim, from Unknown City}}


\pointsinmargin
\marginpointname{ pts}
\marginbonuspointname{\textsc{e.c.}}

\usepackage{multicol}

%% Remove comment %% to print answer key.
\unframedsolutions
\renewcommand{\solutiontitle}{}
\SolutionEmphasis{\bfseries}

%% Create a Matching question format
\newcommand*\Matching[1]{
\ifprintanswers
	\textbf{#1}
\else
	\rule{2in}{0.4pt}
\fi
}
\newlength\matchlena
\newlength\matchlenb
\settowidth\matchlena{\rule{2.1in}{0pt}}
\newcommand\MatchQuestion[2]{%
	\setlength\matchlenb{0.98\linewidth}
	\addtolength\matchlenb{-\matchlena}
	\parbox[t]{\matchlena}{\Matching{#1}}\enspace\parbox[t]{\matchlenb}{#2}}


%% Create a Matching question format
\newcommand*\LabMatching[1]{
\ifprintanswers
	\textbf{#1}
\else
	\rule{2.0in}{0.4pt}
\fi
}
\newlength\matchlaba
\newlength\matchlabb
\settowidth\matchlaba{\rule{2in}{0pt}}
\newcommand\LabMatch[2]{%
	\setlength\matchlabb{0.98\linewidth}
	\addtolength\matchlabb{-\matchlaba}
	\parbox[t]{\matchlaba}{\LabMatching{#1}} \parbox[t]{\matchlabb}{#2}}

% change indentation for multiple choice
\renewcommand{\choiceshook}{%
	\setlength{\leftmargin}{0.25in}%
	\setlength{\parsep}{3pt}%
}


\newcommand*\AnswerBox[2]{%
    \parbox[t][#1]{0.92\textwidth}{%
    \begin{solution}#2\end{solution}}
    \vspace{\stretch{1}}
}

\newenvironment{AnswerPage}[1]
    {\begin{minipage}[t][#1]{0.92\textwidth}%
    \begin{solution}}
    {\end{solution}\end{minipage}
    \vspace{\stretch{1}}}

\newlength{\basespace}
\setlength{\basespace}{5\baselineskip}

\newcommand{\bumppoints}[1]{%
	\addtocounter{numpoints}{#1}
}


\begin{document}


\fullwidth{%
This lab exam covers external and skeletal features and external topology. Each question is worth 2 points.

Fill in the blanks with the region of the body or the feather group that correspond to the numbered parts in the figure at right. Ignore unnumbered labels.
}

\begin{questions}

%\vspace{2\baselineskip}
\bigskip

%\setcounter{question}{0}
\bumppoints{50}
\begin{multicols}{2}
	\question 
	\LabMatch{Mantle}{body region.}
	%\rule{2in}{0.4pt} body region.
	\vspace{0.4\baselineskip}
	
	\question
	\LabMatch{Nape}{body region.}
	%\rule{2in}{0.4pt} body region.
	\vspace{0.4\baselineskip}
	
	\question 
	\LabMatch{\small Greater secondary coverts}{feather group.}
	\vspace{0.4\baselineskip}
	
	\question 
	\LabMatch{Primaries}{feather group.}
	\vspace{0.6\baselineskip}
	
	\question 
	\LabMatch{Rectices}{feather group.} 
	\vspace{0.6\baselineskip}
	
	\question 
	\LabMatch{Upper tail coverts}{feather group.}
	\vspace{0.6\baselineskip}
	
	\question 
	\LabMatch{Rump}{body region.} 
	
	\columnbreak
	
	\includegraphics[width=\linewidth]{bird_upper_numbered}
	
\end{multicols}

\fullwidth{%
Choose the answer that matches the number of the numbered head region. Ignore the unnumbered labels.}
%\vspace{\baselineskip}

\begin{multicols}{2}
	\question
	Region that covers the top of the head.
	
	\begin{choices}
		\choice Auricular.
		\correctchoice Crown.
		\choice Supercilium.
		\choice Nape.
		\choice Lores.
	\end{choices}
	
	
	\question
	Stripe above the eye.
	
	\begin{choices}
		\choice Malar.
		\choice Lores.
		\choice Median stripe.
		\choice Auricular.
		\correctchoice Supercilium
	\end{choices}
	
	\columnbreak
	
	\includegraphics[width=\linewidth]{bird_head_numbered}
	
	\question
	Region of the face between the eye and bill.
	
	\begin{choices}
		\choice Malar.
		\choice Nasal.
		\choice Supercilium.
		\correctchoice Lores.
		\choice Crown.
	\end{choices}
	
\end{multicols}

\newpage

\begin{multicols}{2}
	
	\vspace*{2\baselineskip}
	
	\question\LabMatch{Furcula}{Proper name for the fused collar bone.}
	\vspace{0.6\baselineskip}
	
	\question\LabMatch{Carina}{Proper name for the “keel” of the sternum.}
	\vspace{0.6\baselineskip}
	
%	\question\LabMatch{Tarsus}{Overall name for the leg.}
	\question\LabMatch{Tarsometatarsus}{Proper name for this bone.}
	\vspace{0.6\baselineskip}
	
		\question\LabMatch{Allular digit}{Proper name for this small bone.}
	\vspace{0.6\baselineskip}
	
		\question\LabMatch{Supracoracoid}{Proper name for this bone.} \vspace{0.6\baselineskip}
	\columnbreak
	
	\vspace{-\baselineskip}
	
	\includegraphics[width=\linewidth]{skeleton_pigeon_numbered}
	
\end{multicols}

\bigskip

\begin{multicols}{2} \raggedcolumns
\question
The bill of this hummingbird is not straight.

%\vspace{0.5\baselineskip}

\bigskip

It is \ifprintanswers \textbf{decurved}. \else \rule{2.5in}{0.4pt}.\fi

\bigskip

\question
The bill of this hummingbird is round or circular in cross section, which is called

\bigskip

\ifprintanswers \textbf{terete}. \else \rule{2.5in}{0.4pt}.\fi


\columnbreak
\includegraphics[width=2in]{decurved_lucifer_hummingbird}
\end{multicols}

\vspace{\baselineskip}

\begin{multicols}{2} \raggedcolumns
	\question
	The proper name for the pouch on the lower bill is
	
	%\vspace{0.5\baselineskip}
	
	\bigskip
	
	\ifprintanswers \textbf{gular sac}. \else \rule{2.5in}{0.4pt}. \fi
	
	\columnbreak
	\includegraphics[width=2in]{gular_sac_pelican2}
\end{multicols}

\newpage

\begin{multicols}{2} \raggedcolumns
	\question
	The proper name for the shape of this
	deep, narrow bill (side to side) is 
	
	\bigskip
	
	\ifprintanswers \textbf{compressed}. \else \rule{2.5in}{0.4pt}. \fi
	
	\columnbreak
	\includegraphics[width=2in]{compressed_razorbill}
\end{multicols}

\begin{multicols}{2} \raggedcolumns
	\question
	 Roman numerals are toe numbers. The proper name for this foot shape is
	
	\bigskip
	
	\ifprintanswers \textbf{zygodactyl}. \else \rule{2.5in}{0.4pt}. \fi
	
	\smallskip
	
	

	\bigskip \bigskip
	
	\question
	The proper name for the toe numbered I is
	
	\bigskip
	
	 \ifprintanswers \textbf{hallux}. \else \rule{2.5in}{0.4pt}. \fi
	
	\columnbreak
	\includegraphics[width=2in]{foot_zygodactyl}
\end{multicols}


\begin{multicols}{2} \raggedcolumns
	\question
	The proper name for the toothed edges of \newline 	
	this bill is
	
	\bigskip
	
	\ifprintanswers \textbf{serrated} \else \rule{2.5in}{0.4pt}. \fi
		
	\columnbreak
	\includegraphics[width=3in]{serrated_merganser}
\end{multicols}


\begin{multicols}{2} \raggedcolumns
	\question
	The type of webbing shown in the left and right images is 
	
	\bigskip
	
	left:~\ifprintanswers \textbf{semipalmate} \else \rule{2.5in}{0.4pt}. \fi
	
	\bigskip
	
	right:~\ifprintanswers \textbf{lobate} \else \rule{2.5in}{0.4pt}. \fi
	
	\columnbreak
	\includegraphics[width=1.5in]{foot_semipalmate} \includegraphics[width=1.5in]{foot_lobate_grebe}\newline
	{\footnotesize Webbing is shaded gray. }
\end{multicols}


\newpage

\begin{multicols}{2} \raggedcolumns
	\question
	The proper name for this bill shape is
	
	\bigskip \bigskip
	
	\ifprintanswers \textbf{bent} \else \rule{2.5in}{0.4pt}. \fi
	
	\columnbreak
	\includegraphics[width=2in]{bent_flamingo}
\end{multicols}

\vspace{2\baselineskip}

\question
The proper name for \textit{all} of the main flight feathers of the wing is ~\ifprintanswers \textbf{remiges}. \else \rule{2.25in}{0.4pt}. \fi

\vspace{2\baselineskip}

\bonusquestion[]
The name for the most common foot type for perching birds is

\bigskip

\ifprintanswers \textbf{anisodactyl}. \else \rule{2.5in}{0.4pt}. \fi


\vspace{2\baselineskip}

\bonusquestion[]
Name your favorite bird. The only restriction is that you must use the full and proper common name.

\bigskip

\ifprintanswers \textbf{whatever}. \else \rule{2.5in}{0.4pt}. \fi



\end{questions}





\end{document}