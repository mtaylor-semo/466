%!TEX TS-program = lualatex
%!TEX encoding = UTF-8 Unicode

\documentclass[12pt, addpoints]{exam}

%\printanswers


\usepackage[left=1in,right=0.75in,top=1in]{geometry}     
\usepackage[parfill]{parskip}

\usepackage{fontspec}
\def\mainfont{Linux Libertine O}
\setmainfont[Ligatures={TeX, Common}, BoldFont={* Bold}, ItalicFont={* Italic}, Numbers={Proportional, OldStyle}]{\mainfont}
%\setmonofont[Scale=MatchLowercase]{Inconsolata} 
\setsansfont[Scale=MatchLowercase]{Linux Biolinum O} 
\usepackage{microtype}

% This defines \amper for the fancy ampersand
% to be used in the header. See
% https://tex.stackexchange.com/a/58185/39194
%\usepackage{xspace}
%\newfontfamily\amperfont[Style=Alternate]{Linux Libertine O}
%\makeatletter
%\DeclareRobustCommand{\amper}{{\amperfont\ifx\f@shape\scname\smaller[1.2]\fi\&}\xspace}
%\makeatother

\usepackage{graphicx}
	\graphicspath{%
	{/Users/goby/Pictures/teach/466/exams/}}%


% Used to get the last two digits of the year for the header below.
\def\short#1{\csname @gobbletwo\expandafter\endcsname\number#1}

\usepackage[sc]{titlesec}

\pagestyle{headandfoot}

\firstpageheader{Ornithology, Lab Exam 2, \textsc{s}\short{\year}, \numpoints~points.}{}{%
	\ifprintanswers\textbf{KEY}\else Name: \enspace \makebox[2.5in]{\hrulefill}\fi}
\runningheader{}{}{\small{pg. \thepage}}
\runningheadrule

\footer{\iflastpage{\small Lecture: \rule{0.6in}{0.4pt} / \numpoints\ points}{}}{}{\rule{1in}{0.4pt}}%{\makebox[0.75in]{\hrulefill}}
\extrafootheight{-0.5in}


%\fancypagestyle{labexam}{
%\fancyfoot[L]{\small \rule{0.6in}{0.4pt} / \numpoints\ points}
%}
%\pagestyle{foot}
%\footer{\small \rule{0.6in}{0.4pt} / 25~points}{}{}


%\fancypagestyle{alim}{\fancyhf{}\renewcommand{\headrulewidth}{0pt}\fancyfoot[R]{Alim, from Unknown City}}


\pointsinmargin
\marginpointname{ pts}

\usepackage{multicol}

%% Remove comment %% to print answer key.
\unframedsolutions
\renewcommand{\solutiontitle}{}
\SolutionEmphasis{\bfseries}

%% Create a Matching question format
\newcommand*\Matching[1]{
\ifprintanswers
	\textbf{#1}
\else
	\rule{2in}{0.4pt}
\fi
}
\newlength\matchlena
\newlength\matchlenb
\settowidth\matchlena{\rule{2.1in}{0pt}}
\newcommand\MatchQuestion[2]{%
	\setlength\matchlenb{0.98\linewidth}
	\addtolength\matchlenb{-\matchlena}
	\parbox[t]{\matchlena}{\Matching{#1}}\enspace\parbox[t]{\matchlenb}{#2}}


%% Create a Matching question format
\newcommand*\LabMatching[1]{
\ifprintanswers
	\textbf{#1}
\else
	\rule{2.0in}{0.4pt}
\fi
}
\newlength\matchlaba
\newlength\matchlabb
\settowidth\matchlaba{\rule{2in}{0pt}}
\newcommand\LabMatch[2]{%
	\setlength\matchlabb{0.98\linewidth}
	\addtolength\matchlabb{-\matchlaba}
	\parbox[t]{\matchlaba}{\LabMatching{#1}} \parbox[t]{\matchlabb}{#2}}

% change indentation for multiple choice
\renewcommand{\choiceshook}{%
	\setlength{\leftmargin}{0.25in}%
	\setlength{\parsep}{3pt}%
}


\newcommand*\AnswerBox[2]{%
    \parbox[t][#1]{0.92\textwidth}{%
    \begin{solution}#2\end{solution}}
    \vspace{\stretch{1}}
}

\newenvironment{AnswerPage}[1]
    {\begin{minipage}[t][#1]{0.92\textwidth}%
    \begin{solution}}
    {\end{solution}\end{minipage}
    \vspace{\stretch{1}}}

\newlength{\basespace}
\setlength{\basespace}{5\baselineskip}

\newcommand{\bumppoints}[1]{%
	\addtocounter{numpoints}{#1}
}


\begin{document}


\fullwidth{%
This lab exam covers external and skeletal features and taxonomy (order and family). Each question is worth 2 points.

Fill in the blanks with the region of the body or the feather group that correspond to the numbered parts in the figure at right. Ignore unnumbered labels.
}

\begin{questions}

%\vspace{2\baselineskip}
\bigskip

%\setcounter{question}{0}
\bumppoints{14}
\begin{multicols}{2}
	\question 
	\LabMatch{Mantle}{body region.}
	%\rule{2in}{0.4pt} body region.
	\vspace{0.4\baselineskip}
	
	\question
	\LabMatch{Nape}{body region.}
	%\rule{2in}{0.4pt} body region.
	\vspace{0.4\baselineskip}
	
	\question 
	\LabMatch{\small Greater secondary coverts}{feather group.}
	\vspace{0.4\baselineskip}
	
	\question 
	\LabMatch{Primaries}{feather group.}
	\vspace{0.6\baselineskip}
	
	\question 
	\LabMatch{Rectices}{feather group.} 
	\vspace{0.6\baselineskip}
	
	\question 
	\LabMatch{Upper tail coverts}{feather group.}
	\vspace{0.6\baselineskip}
	
	\question 
	\LabMatch{Rump}{body region.} 
	
	\columnbreak
	
	\includegraphics[width=\linewidth]{bird_upper_numbered}
	
\end{multicols}

\fullwidth{%
Choose the answer that matches the number of the numbered head region. Ignore the unnumbered labels.}
%\vspace{\baselineskip}

\bumppoints{6}
\begin{multicols}{2}
	\question
	Region that covers the top of the head.
	
	\begin{choices}
		\choice Auricular.
		\correctchoice Crown.
		\choice Supercilium.
		\choice Nape.
		\choice Lores.
	\end{choices}
	
	
	\question
	Stripe above the eye.
	
	\begin{choices}
		\choice Malar.
		\choice Lores.
		\choice Median stripe.
		\choice Auricular.
		\correctchoice Supercilium
	\end{choices}
	
	\columnbreak
	
	\includegraphics[width=\linewidth]{bird_head_numbered}
	
	\question
	Region of the face between the eye and bill.
	
	\begin{choices}
		\choice Malar
		\choice Nasal
		\choice Supercilium
		\correctchoice Lores.
		\choice Crown
	\end{choices}
	
\end{multicols}

\newpage

\bumppoints{6}
\begin{multicols}{2}
	
	\vspace*{2\baselineskip}
	
	\question\LabMatch{Furcula}{Proper name for the fused collar bone.}
	\vspace{0.6\baselineskip}
	
	\question\LabMatch{Carina}{Proper name for the “keel” of the sternum.}
	\vspace{0.6\baselineskip}
	
%	\question\LabMatch{Tarsus}{Overall name for the leg.}
	\question\LabMatch{Tarsometatarsus}{Proper name for this bone of the lower leg.}
	\vspace{0.6\baselineskip}
	
	\columnbreak
	
	\vspace{-\baselineskip}
	
	\includegraphics[width=\linewidth]{skeleton_pigeon_numbered}
	
\end{multicols}


\question
A character that separates all diving ducks from all dabbling ducks is

	\begin{choices}
		\correctchoice The hallux of diving ducks is lobed. The hallux of dabbling ducks is not lobed.
		\choice The bill of diving ducks is terrete. The bill of dabbling ducks is spatulate.
		\choice The front toes of diving ducks are fully webbed. The front toes of dabbling ducks are partially webbed.
		\choice The tomium of diving ducks is serrate. The tomium of dabbling ducks is lamellate.
	\end{choices}


\bumppoints{6}

\question
The only grebe in Missouri is the Pied-billed Grebe. It belongs to the family

	\begin{choices}
		\choice Anatidae
		\choice Anseriformes
		\correctchoice Podicipedidae
		\choice Phasianidae
		\choice Galloanseriformes
	\choice Columbiformes
	\end{choices}

\question
This family of weak-footed flyers has pamprodactyl feet.

	\begin{choices}
		\choice Caprimulgidae
		\correctchoice Apodidae
		\choice Trochilidae
		\choice Podicipedidae
		\choice Passeridae
	\end{choices}

\question
This family of birds have a pectinate middle toe as a key characteristic.

\begin{choices}
	\correctchoice Ardeidae
	\choice Anatidae
	\choice Picidae
	\choice Laridae
	\choice Scolopacidae
\end{choices}

\bumppoints{16}
\fullwidth{Match the proper taxonomic level to each description. Not all names are used. No name is used twice.}
\vspace{-1\baselineskip}
\fullwidth{%
	% Edit the list to be sure it has all the terms used in the matching. 
% Use no more than 30 terms but be sure word list is multiple of 3.

\begin{multicols}{3}
\noindent Accipitridae \\
Accipitriformes \\
Buteonidae \\
Caprimulgiformes \\
Cathartidae \\
Charadriidae \\
Charadriformes \\
Columbidae \\
Falconidae \\
Falconiformes \\
Galliformes \\
Laridae \\
Odontophoridae \\
Phasianidae \\
Scolopacidae \\
Scolopaciformes \\
Strigidae \\ 
Strigiformes
\end{multicols}

}
\vspace{0.6\baselineskip}

\question\MatchQuestion{Galliformes}{Order with the families Odontophoridae and Phasianidae.}
\vspace{0.6\baselineskip}

\question\MatchQuestion{Falconiformes}{Order with the American Kestrel and Perigrine Falcon.}
\vspace{0.6\baselineskip}

\question\MatchQuestion{Columbidae}{Family of doves and pigeons.}
\vspace{0.6\baselineskip}

\question\MatchQuestion{Phasianidae}{Family with Wild Turkey and Greater Prairie Chicken.}
\vspace{0.6\baselineskip}

\question\MatchQuestion{Caprimulgiforrmes}{Order with nightjars, swifts, and hummingbirds.}
\vspace{0.6\baselineskip}

\question\MatchQuestion{Charadriformes}{Order with all shorebirds, gulls, and terns.}
\vspace{0.6\baselineskip}

\question\MatchQuestion{Cathartidae}{Family with Turkey and Black Vultures.}
\vspace{0.6\baselineskip}

\question\MatchQuestion{Strigidae / (Tytonidae)}{Family of owls. (Bonus +2 if you name the family you were not required to learn.)}
\vspace{0.6\baselineskip}

%\bonusquestion (extra credit)
%Name the type of bill found on the Atlantic Puffin, pictured at right.
\bumppoints{2}
\question
Name the type of bill found on most waterfowl \textit{or} the type of bill found on mergansers. Two answers are possible. Bonus +2 if you name both.

\ifprintanswers
\vspace{\baselineskip}

\textbf{Waterfowl: Spatulate. Mergansers: terrete}
\fi
%\columnbreak

%\includegraphics[width=0.75\linewidth]{puffin_head}

%\newpage
%
%

%

%	\vfill


\end{questions}





\end{document}