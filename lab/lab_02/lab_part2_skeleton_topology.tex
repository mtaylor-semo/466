%!TEX TS-program = lualatex
%!TEX encoding = UTF-8 Unicode

\documentclass[10pt]{article}  
\usepackage[left=0.75in,right=0.75in,top=1in,bottom=0.5in]{geometry} 
\geometry{letterpaper}                   		% 
\usepackage[parfill]{parskip}    		% Activate to begin paragraphs with an empty line rather than an indent
\setlength{\parindent}{0pt}

\usepackage{graphicx}
\graphicspath{%
	{/Users/goby/Pictures/teach/466/lab/}}
	
% FONTS
\usepackage{fontspec}
\def\mainfont{Linux Libertine O}
%\defaultfontfeatures{Mapping=tex} % converts LaTeX specials (``quotes'' --- dashes etc.) to unicode
\setmainfont[Ligatures={Common, TeX}, BoldFont={* Bold}, ItalicFont={* Italic}, Numbers={Proportional, OldStyle}]{\mainfont}
%\setmonofont[Scale=MatchLowercase]{Inconsolata} 
%\setsansfont[Scale=MatchLowercase]{Linux Biolinum O} 
\usepackage{microtype}



\usepackage[singlelinecheck=false]{caption}
\usepackage{array}
\newcolumntype{L}[1]{>{\raggedright\let\newline\\\arraybackslash\hspace{0pt}}m{#1}}
\newcolumntype{C}[1]{>{\centering\let\newline\\\arraybackslash\hspace{0pt}}p{#1}}
\newcolumntype{R}[1]{>{\raggedleft\let\newline\\\arraybackslash\hspace{0pt}}m{#1}}


%\pagenumbering{gobble}
%\usepackage{pdflscape}
\usepackage{longtable}
\usepackage[frenchlinks]{hyperref}
\hypersetup{
	frenchlinks=true,
	pdfborder={0 0 0}}
\usepackage{booktabs}
\usepackage{multicol}
%\usepackage{amssymb}
%\usepackage{enumitem}
%\setlist{noitemsep}
%\setlist[description]{style=multiline, leftmargin=1.25cm, parsep=1ex}

\usepackage[sc]{titlesec}

\usepackage{fancyhdr}
\fancyhf{}
\pagestyle{fancy}
\lhead{}
\chead{}
\rhead{\footnotesize pg.~\thepage }
\renewcommand{\headrulewidth}{0.4pt}

\fancypagestyle{plain}{%
	\fancyhf{}
	\lhead{\textsc{bi}~466/666: Ornithology}
	\rhead{Name: \enspace \makebox[2.5in]{\hrulefill}}
	\renewcommand{\headrulewidth}{0pt}
}

\newcommand{\blankline}{\rule{2in}{0.4pt}}



\begin{document}
\thispagestyle{plain}
%\begin{landscape}

\subsection*{Skeleton and external topology, part ii: apply your knowledge}

\subsubsection*{Skeleton}

	Get with Dr. Taylor and show him the following structures on the large skeleton at the front of the room. This skeleton is (probably) from a Double-crested Cormorant.

\begin{multicols}{3}
	alular digit\\
	carina (extension of the sternum)\\
	carpometacarpus\\
	coracoid\\
	femur\\
	furcula (“wishbone”)\\
	hallux (toe I)\\
	humerous\\
	major and minor digits (wing)\\
	% mandible, lower\\
	% mandible, upper\\
	%pygostyle\\
	radius\\
	sternum\\
	tarsometatarsus\\
	tibiotarsus\\
	%tomium, upper and lower\\
	toes (toes II–IV)\\
	ulna
\end{multicols}

\subsubsection*{External topology}

Visit the various stations around the room. Answer the following questions.


\medskip

\textsc{Green Jay:} What color is the crown? \blankline

\bigskip

\textsc{Green Jay:} What color are the outermost rectrices? \blankline

\bigskip

%\textsc{Green Jay:} What color is the breast? \blankline

%\bigskip
\textsc{Masked Crimson Tanager:} What color is the mantle? \blankline

\bigskip

\textsc{Masked Crimson Tanager:} What color are the lores ? \blankline

\bigskip

 
\textsc{African Pied Wagtail} (tag 876): What color is the supercilium? \blankline

\bigskip

\textsc{African Pied Wagtail} (tag 876): What color is the throat? \blankline

\bigskip

\textsc{Silver-eared Mesia} (tag 546/547): What color is the auricular? \blankline

\bigskip

\textsc{Silver-eared Mesia} (tag 546/547): What color is the breast? \blankline

\bigskip

\textsc{Silver-eared Mesia} (tag 546/547): What color is the rump/upper tail coverts? \blankline

\bigskip


\textsc{Rose-breasted Parakeet} (tag 1015): What color is the malar? \blankline

\bigskip

\textsc{Rose-breasted Parakeet} (tag 1015): What color is the flank? \blankline

\bigskip

\textsc{Superb Starling} (tag 1482): What color is the nape? \blankline

\bigskip

\textsc{Superb Starling} (tag 1482): What color is the belly? \blankline

\bigskip
\textsc{Superb Starling} (tag 1482): What color are the undertail coverts?   \blankline

\subsubsection*{Wings}

Visit the mounted Snow Goose and Wild Turkey wing specimens. Complete this table. The primaries in the Snow Goose are dark. 

Also, identify the greater and median primary and secondary coverts.  Try to identify the tertials. These feather groups are not included in the table.

\begin{tabular}{lcc}
	\toprule
	Number	&	Snow Goose	& Wild Turkey \tabularnewline
	\midrule
	&	& \\[0.3in]
	Primaries	& \rule{1in}{0.4pt} & \rule{1in}{0.4pt} \\[0.5in]
	Secondaries & \rule{1in}{0.4pt} & \rule{1in}{0.4pt} \\[0.5in]
	Allular quills & \rule{1in}{0.4pt} & \rule{1in}{0.4pt} \\
	\bottomrule
\end{tabular}

\vspace{2\baselineskip}

Examine the duck wing next to the above two birds. The iridescent patch at the rear edge of the wing is called the speculum.  Give the number range of primaries or secondaries or both that include the speculum.  Zero is an acceptable answer if the speculum is found only on the primaries or secondaries. For example, if the speculum includes the first three primaries and the first five secondaries, your answer would be P1-3 and S1-5.

\vspace{2\baselineskip}

The speculum covers \rule{3in}{0.4pt}

\vspace{2\baselineskip}

If present, make a sketch of the prepared skin of American Coot. Include the approximate locations of the pterylae. Look at a couple skins to note the variation in the pterylae location.


\end{document}  