%!TEX TS-program = lualatex
%!TEX encoding = UTF-8 Unicode

\documentclass[nofonts, letterpaper]{tufte-handout}

%\geometry{showframe} % display margins for debugging page layout

\usepackage{graphicx} % allow embedded images
  \setkeys{Gin}{width=\linewidth,totalheight=\textheight,keepaspectratio}
  \graphicspath{{img/}} % set of paths to search for images
  
\usepackage{fontspec}
  \setmainfont[Ligatures={Common,TeX},Numbers={Proportional}]{Linux Libertine O}
  \setsansfont{Linux Biolinum O}
\usepackage{microtype}
\usepackage{enumitem}
\usepackage{multicol} % multiple column layout facilities
%\usepackage{hyperref}
%\usepackage{fancyvrb} % extended verbatim environments
%  \fvset{fontsize=\normalsize}% default font size for fancy-verbatim environments

% Change the header to shift the title to the left side of the page. 
% Replaced \quad with \hfill.  See \plaintitle in tufte-common.def
{\fancyhead[RE,RO]{\scshape{\newlinetospace{\plaintitle}}\hfill\thepage}}

\makeatletter
% Paragraph indentation and separation for normal text
\renewcommand{\@tufte@reset@par}{%
  \setlength{\RaggedRightParindent}{1.0pc}%
  \setlength{\JustifyingParindent}{1.0pc}%
  \setlength{\parindent}{1pc}%
  \setlength{\parskip}{0pt}%
}
\@tufte@reset@par

% Paragraph indentation and separation for marginal text
\renewcommand{\@tufte@margin@par}{%
  \setlength{\RaggedRightParindent}{0pt}%
  \setlength{\JustifyingParindent}{0.5pc}%
  \setlength{\parindent}{0.5pc}%
  \setlength{\parskip}{0pt}%
}

\makeatother

\title{Study Guide 11}
\author{Conservation and management}

\date{} % without \date command, current date is supplied

\begin{document}

\maketitle	% this prints the handout title, author, and date

%\printclassoptions

\section{Concepts}
%\marginnote{\textbf{Study:} Chap. 20: 611--614.\\ Chap. 21: 660--678.}

These concept-questions cover most of the lecture material but exam questions are not restricted to these questions. Questions may also come from the related pages from the textbook.\vspace{\baselineskip}

\begin{enumerate}

\item Diagram the model of island biogeography. Include immigration and extinction rates. Include both small and large islands and close and distant islands. Relate the model to species richness of islands of various sizes and distances (e.g., small, close islands and large, distant islands).

\item Explain how the principles of island biogeography can contribute to the understanding of bird conservation and reserve management in a fragmented habitat.

\item What is a metapopulation?  What is a source population?  What is a sink population?  (For sources and sinks, consider birth rates vs. death rates, and immigration rates vs. emigration rates). What habitat conditions are associated with source populations? What conditions are associated with sink populations?  (For habitat conditions, consider available resources and fragment sizes).

\item Why is fledging success greater in large fragments rather than smaller fragments.

\item \label{itm:design} Carefully study Fig. 21--13 of your text. For each example of reserve design, explain why the diagram in the right column is considered better than the diagram in the left column.  Be able to explain these individual examples together in a broader context. That is, if you were able to design a reserve from scratch, how would you design it, using the examples from the figure.

\item What is population viability analysis?  What factors are included in such an analysis? How was it used for the Florida Scrub Jays? What is the number of territories of Florida Scrub Jays needed in an area to have the greatest chance of sustaining the population for more than 400 years?  Be sure to read this section of your text!

\item What is a landscape mosaic?  Explain the concept of patch dynamics across a landscape mosaic. How does this increase bird diversity?  How does this relate back to the principles of reserve design from question \ref{itm:design} above?

\item Why should migratory birds be managed as part of an international network? How does this relate back to the principles of reserve design from question \ref{itm:design} above?

 
\end{enumerate}


\end{document}