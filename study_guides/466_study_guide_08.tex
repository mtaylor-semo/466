%!TEX TS-program = lualatex
%!TEX encoding = UTF-8 Unicode

\documentclass[nofonts, letterpaper]{tufte-handout}

%\geometry{showframe} % display margins for debugging page layout

\usepackage{graphicx} % allow embedded images
  \setkeys{Gin}{width=\linewidth,totalheight=\textheight,keepaspectratio}
  \graphicspath{{img/}} % set of paths to search for images
  
\usepackage{fontspec}
  \setmainfont[Ligatures={Common,TeX},Numbers={Proportional}]{Linux Libertine O}
  \setsansfont{Linux Biolinum O}
\usepackage{microtype}
\usepackage{enumitem}
\usepackage{multicol} % multiple column layout facilities
%\usepackage{hyperref}
%\usepackage{fancyvrb} % extended verbatim environments
%  \fvset{fontsize=\normalsize}% default font size for fancy-verbatim environments

% Change the header to shift the title to the left side of the page. 
% Replaced \quad with \hfill.  See \plaintitle in tufte-common.def
{\fancyhead[RE,RO]{\scshape{\newlinetospace{\plaintitle}}\hfill\thepage}}

\makeatletter
% Paragraph indentation and separation for normal text
\renewcommand{\@tufte@reset@par}{%
  \setlength{\RaggedRightParindent}{1.0pc}%
  \setlength{\JustifyingParindent}{1.0pc}%
  \setlength{\parindent}{1pc}%
  \setlength{\parskip}{0pt}%
}
\@tufte@reset@par

% Paragraph indentation and separation for marginal text
\renewcommand{\@tufte@margin@par}{%
  \setlength{\RaggedRightParindent}{0pt}%
  \setlength{\JustifyingParindent}{0.5pc}%
  \setlength{\parindent}{0.5pc}%
  \setlength{\parskip}{0pt}%
}

\makeatother

% Set up the spacing using fontspec features
\renewcommand\allcapsspacing[1]{{\addfontfeatures{LetterSpace=15}#1}}
\renewcommand\smallcapsspacing[1]{{\addfontfeatures{LetterSpace=10}#1}}

\title{Study Guide 09}
\author{Migration}

\date{} % without \date command, current date is supplied

\begin{document}

\maketitle	% this prints the handout title, author, and date

%\printclassoptions

\section{Vocabulary}
\vspace{-1\baselineskip}
\begin{multicols}{2}
migration \\
primary production \\
eastern flyway \\
central flyway \\
western flyway \\
retinal cryptochromes 
\end{multicols}

\section{Concepts}
%\marginnote{\textbf{Study:} Chap. 18: 533--562.\\ Chap. 21: 635--645. Skim 645--650 for examples of human persecution of birds.}

These concept-questions cover most of the lecture material but exam questions are not restricted to these questions. Questions may also come from the related pages from the textbook.\vspace{\baselineskip}

\begin{enumerate}

\item Why do birds migrate?

\item The migratory routes of many seabirds that migrate between the two hemispheres show a figure-8 pattern. Explain why in terms of high and low pressure systems in each hemisphere.

\item Some individuals of these seabirds don't strictly follow the figure-8 pattern. Some may instead deviate and follow close to a coastline. Explain why.

\item Why do most birds migrate at night?

\item Explain the role of prevailing winds within the northern hemisphere to describe why migratory birds of a particular flyway (e.g., eastern) fly looped routes between spring and fall migrations. 

\item Prevailing winds do not \emph{always} explain looped routes in the North American flyways. Give a specific example of a looped route that might not depend on prevailing winds. What does influence this route?

\item Why do many species that use the western flyway migrate at higher altitudes that birds that use the eastern flyway. Does this have conservation implications?

\item Describe broadly how climate change might affect migration or nesting success after arrival on breeding grounds.

\item How do birds navigate during migration? Describe some of the studies that have shown how birds orient.

\item What is the hypothesized role of retinal cryptochromes for orientation? Name the \emph{specific} structure where retinal crytochromes are found (the overall organ is not specific enough; I'm not being tricky here; I'm being obvious).

\item Soaring birds tend to have the shortest migration routes. Why?

\item Assume a continuous flapping species and a soaring species. Both have similar wing loads. In general, the continuous flapper will arrive sooner. Why?

\end{enumerate}


\end{document}