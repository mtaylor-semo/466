%!TEX TS-program = lualatex
%!TEX encoding = UTF-8 Unicode

\documentclass[nofonts, letterpaper]{tufte-handout}

%\geometry{showframe} % display margins for debugging page layout

\usepackage{graphicx} % allow embedded images
  \setkeys{Gin}{width=\linewidth,totalheight=\textheight,keepaspectratio}
  \graphicspath{{img/}} % set of paths to search for images
  
\usepackage{fontspec}
  \setmainfont[Ligatures={Common,TeX},Numbers={Proportional}]{Linux Libertine O}
  \setsansfont{Linux Biolinum O}
\usepackage{microtype}
\usepackage{enumitem}
\usepackage{multicol} % multiple column layout facilities
%\usepackage{hyperref}
%\usepackage{fancyvrb} % extended verbatim environments
%  \fvset{fontsize=\normalsize}% default font size for fancy-verbatim environments

% Change the header to shift the title to the left side of the page. 
% Replaced \quad with \hfill.  See \plaintitle in tufte-common.def
{\fancyhead[RE,RO]{\scshape{\newlinetospace{\plaintitle}}\hfill\thepage}}

\makeatletter
% Paragraph indentation and separation for normal text
\renewcommand{\@tufte@reset@par}{%
  \setlength{\RaggedRightParindent}{1.0pc}%
  \setlength{\JustifyingParindent}{1.0pc}%
  \setlength{\parindent}{1pc}%
  \setlength{\parskip}{0pt}%
}
\@tufte@reset@par

% Paragraph indentation and separation for marginal text
\renewcommand{\@tufte@margin@par}{%
  \setlength{\RaggedRightParindent}{0pt}%
  \setlength{\JustifyingParindent}{0.5pc}%
  \setlength{\parindent}{0.5pc}%
  \setlength{\parskip}{0pt}%
}

\makeatother

% Set up the spacing using fontspec features
\renewcommand\allcapsspacing[1]{{\addfontfeatures{LetterSpace=15}#1}}
\renewcommand\smallcapsspacing[1]{{\addfontfeatures{LetterSpace=10}#1}}

\title{Study Guide 03}
\author{Anatomy and physiology}

\date{} % without \date command, current date is supplied

\begin{document}

\maketitle	% this prints the handout title, author, and date

%\printclassoptions

\section{Vocabulary}
\vspace{-1\baselineskip}
\begin{multicols}{2}
sternum \\
carina \\
furcula \\
pectoralis \\
supracoracoideus \\
pneumatic bone \\
parabronchus \\
cross-current exchange \\
counter-current exchange  \\
crop \\
gizzard \\
thermal neutral zone 
\end{multicols}

\section{Concepts}
%\marginnote{\textbf{Study:} Chap. 18: 533--562.\\ Chap. 21: 635--645. Skim 645--650 for examples of human persecution of birds.}

These concept-questions cover most of the lecture material but exam questions are not restricted to these questions. Questions may also come from the related pages from the textbook.\vspace{\baselineskip}

\begin{enumerate}

\item What is the function of the carina? The carina is an extension of what bone?

\item Provide some examples of skeletal structures that are fused in birds to increase strength and rigidity for flight.

\item Describe how the pectoralis and supracoracoideus muscles work to power flight. Do this for a full flap cycle (downstroke and upstroke).

\item What is the function of pneumatic bone? How does air exchange occur in pneumatic bone? 

\item Name a group of birds that do not have pneumatic bone? Why does this group lack pneumatic bone?

\item How many air sacs are found in a typical bird? How many are paired? How many are unpaired? (You do not have to name air sacs,)

\item Describe and explain the complete respiratory cycle of a typical bird. Include the roll of the air sacs in the cycle.

\item Compare and contrast cross-current and counter-current exchange.

\item Does cross- and counter-current exchange work by active transport or passive diffusion? Explain.

\item Describe\marginnote{Parabronchus is singular; parabronchi is plural.} how cross-current exchange in the parabronchi maximizes oxygen uptake.

\item What is the thermal neutral zone for birds? Why is this important for metabolism? What might a bird do if body temperature falls outside the thermal neutral zone (too hot or too cold).

\item Describe how a bird standing on snow or ice maintains a warm core body temperature.

\item Describe oxygen demand for birds based on (1) body size and (2) metabolic rate.

\item Why\marginnote{Hint: do birds have big or small eyes?} is the brain restricted to the posterior region of the skull? Explain.

\item Describe\marginnote{Most but not all birds have a crop. All birds have a gizzard but it is greatly reduced in size for some species, such as those that consume nectar primarily.} the roles of the crop and gizzard as part of the overall digestive system,


\item Explain eye position in the head (facing sideways or forward) whether a bird sees primarily in monocular or binocular vision.

\item Why do some birds bob their head back and forth while they walk or swim?

\item What macromolecules are the main energy sources for birds?

\item Why might a migratory bird consume sugary nectar during migration but switch to fats and proteins once it has arrived on its breeding ground.

\end{enumerate}


\end{document}