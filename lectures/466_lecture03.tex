%!TEX TS-program = lualatex
%!TEX encoding = UTF-8 Unicode

\documentclass[t]{beamer}

%%%% HANDOUTS For online Uncomment the following four lines for handout
%\documentclass[t,handout]{beamer}  %Use this for handouts.
%\includeonlylecture{student}
%\usepackage{handoutWithNotes}
%\pgfpagesuselayout{3 on 1 with notes}[letterpaper,border shrink=5mm]
%	\setbeamercolor{background canvas}{bg=black!5}


%%% Including only some slides for students.
%%% Uncomment the following line. For the slides,
%%% use the labels shown below the command.

%% For students, use \lecture{student}{student}
%% For mine, use \lecture{instructor}{instructor}

% FONTS
\usepackage{fontspec}
\def\mainfont{Linux Biolinum O}
\setmainfont[Ligatures=TeX, Contextuals={NoAlternate}, BoldFont={* Bold}, ItalicFont={* Italic}, Numbers={Proportional, OldStyle}]{\mainfont}
\setsansfont[Scale=MatchLowercase]{Linux Biolinum O} 
\usepackage{microtype}

\usepackage{graphicx}
	\graphicspath{%
	{/Users/goby/Pictures/teach/466/lectures/}}%
%	{/Users/goby/Pictures/teach/common/}} % set of paths to search for images

\usepackage{amsmath,amssymb}

%\usepackage{units}

\usepackage{booktabs}
\usepackage{multicol}
%	\setlength{\columnsep=1em}

%\usepackage{textcomp}
%\usepackage{setspace}
\usepackage{tikz}
	\tikzstyle{every picture}+=[remember picture,overlay]

\mode<presentation>
{
  \usetheme{Lecture}
  \setbeamercovered{invisible}
  \setbeamertemplate{items}[square]
}

%\usepackage{calc}
\usepackage{hyperref}

\newcommand\HiddenWord[1]{%
	\alt<handout>{\rule{\widthof{#1}}{\fboxrule}}{#1}%
}

\usepackage{xcolor}
\definecolor{cotr}{HTML}{b0130c}

\begin{document}
%\lecture{instructor}{instructor}

\lecture{student}{student}
{
\usebackgroundtemplate{\includegraphics[width=\paperwidth]{cock_of_the_rock}}
\begin{frame}[b,plain]{\textcolor{white}{Feathers and flight}}
	\tiny \hspace{4em}\textcolor{cotr}{\href{https://www.flickr.com/photos/24201429@N04/13995876801}{Cock-of-the-Rock by felixú, Flickr Creative Commons, \ccbysa{2}}}
\end{frame}
}


% Feather structure
{
\usebackgroundtemplate{\includegraphics[width=\paperwidth]{feather_structure}}

\begin{frame}[t,plain]{Most feathers have a \highlight{vane, rachis,} and \highlight{calamus.}}
	\hangpara Shaft
	
	\quad Rachis: vaned \\
	\quad Calamus: unvaned
	
	\hangpara Vanes are
	
	\quad pennaceous,\\
	\quad plumulaceous, or\\
	\quad combination.
	
	\vfilll
	
	\tiny Lovette and Fitzpatrick, \textit{Handbook of Bird Biology} \textcopyright\,Cornell University
\end{frame}
}

{
\begin{frame}[t,plain]{Barbs and barbules affect structure.}

	\vspace{-\baselineskip}
	
	\centering
	
	\includegraphics[height=0.9\textheight]{feather_vane_structure}
	
	\tiny \hfill Lovette and Fitzpatrick, \textit{Handbook of Bird Biology} \textcopyright\,Cornell University
\end{frame}
}

\begin{frame}[t,plain]{Feathers grow in defined \highlight{pterylae.}}

\vspace{-0.75\baselineskip}

\includegraphics[width=0.48\textwidth]{pterylae_chicken}\hfill \includegraphics[width=0.48\textwidth]{pterylae_diagram}

	\vfilll

	\tiny \hfill Lovette and Fitzpatrick, \textit{Handbook of Bird Biology} \textcopyright\,Cornell University

\end{frame}
%
\begin{frame}[t,plain]{Feather types reflect evolutionary history.}

%\vspace{-0.5\baselineskip}

\includegraphics[width=\textwidth]{feather_types}\\
\begin{tikzpicture}
\draw [<-, very thick] (1,0) -- (11,0) node [midway, below]{Evolutionary trend};
\end{tikzpicture}

	\vfilll


	\tiny \href{https://www.youtube.com/watch?v=MdNyeasi0GI}{Link to video}\hfill Lovette and Fitzpatrick, \textit{Handbook of Bird Biology} \textcopyright\,Cornell University

\end{frame}

{
\usebackgroundtemplate{\includegraphics[width=\paperwidth]{sinosauropteryx_feather_evoloution}}
\begin{frame}[t,plain]

	\vspace{8em}
	
	\hspace{12em}\hangpara Why do you think feathers first evolved?

	
	\vskip0pt plus 1filll
	
	\tiny\hfill Illustration of \textit{Sinosauropteryx} by Jim Robbins.
\end{frame}
}

{
\usebackgroundtemplate{\includegraphics[width=\paperwidth]{theropod_phylogeny}}
\begin{frame}[b,plain]{Feathers show a clear evolutionary trend from simple filaments to asymmetric remiges with hooks {\normalsize \&} barbules.}

\vfilll

\tiny \hfill Zimmer 2011. \textit{The Tangled Bank} \textcopyright\,Roberts and Co.

\end{frame}
}

\begin{frame}[t,plain]{Several hypotheses have been proposed to explain the origins of flight.}

%	\vspace{1em}
	
	\hangpara \highlight{Arboreal} hypothesis
	
	\vspace{1em}
	
	\hangpara \highlight{Cursorial} hypothesis

	\vspace{1em}
	
	\hangpara \highlight{Ontogenetic Transition Wing} hypothesis
\end{frame}

{
\usebackgroundtemplate{\includegraphics[width=\paperwidth]{microraptor_climb}}
\begin{frame}[b,plain]{The \highlight{arboreal hypothesis} suggests that flight evolved in tree-climbing theropods.}
\tiny\hfill\textit{Microraptor gui} drawing \copyright Portia Sloan.
\end{frame}
}

{
\usebackgroundtemplate{\includegraphics[width=\paperwidth]{microraptor_biplane}}
\begin{frame}[b,plain]{\textit{Microraptor} may have glided with a biplane-like wing arrangement.}
    \tiny Chatterjee and Templin 2007. Proc. Natl. Acad. Sci. 104: 1576. \hfill\href{https://www.youtube.com/watch?v=yL0UIzU0EEc}{Link to Video (44 min)}
\end{frame}
}

\begin{frame}[t,plain]{The \highlight{cursorial hypothesis} suggests that flight evolved in running theropods.}
	\begin{center}
		\includegraphics[width=1\textwidth]{cursorial_hypothesis}
	\end{center}

	\vskip0pt plus 1filll
	
	\tiny	Burgers and Chiappe 1999. Nature 399: 60.
\end{frame}

{
\usebackgroundtemplate{\includegraphics[width=\paperwidth]{wair_chukar}}
\begin{frame}[b,plain]{Many birds can climb steep inclines using their wings.}
	\tiny\hspace{1.1in} \href{https://www.youtube.com/watch?v=Owf2iEwV-gk}{Link to Video}
\end{frame}
}
{
\usebackgroundtemplate{\includegraphics[width=\paperwidth]{ontogenetic_transition}}
\begin{frame}[t,plain]{\highlight{O\textsc{tw}} suggests that changes of flight mechanics during developmental mirror evolutionary transitions to flight.}

\vspace{1cm}

\hangpara\hspace{19em}\href{https://www.youtube.com/watch?v=5Rjin-tjOxU}{Wing-assisted climbing}

\hangpara\hspace{19em}\href{https://www.youtube.com/watch?v=FhDQvRBngkQ}{Controlled-descent flapping}

\vskip0pt plus 1filll

\tiny\hfill Dial et al. 2008. Nature 451: 985.
\end{frame}
}


\end{document}

