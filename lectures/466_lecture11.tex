%!TEX TS-program = lualatex
%!TEX encoding = UTF-8 Unicode

\documentclass[t]{beamer}

%%%% HANDOUTS For online Uncomment the following four lines for handout
%\documentclass[t,handout]{beamer}  %Use this for handouts.
%\includeonlylecture{student}
%\usepackage{handoutWithNotes}
%\pgfpagesuselayout{3 on 1 with notes}[letterpaper,border shrink=5mm]
%	\setbeamercolor{background canvas}{bg=black!5}


%%% Including only some slides for students.
%%% Uncomment the following line. For the slides,
%%% use the labels shown below the command.

%% For students, use \lecture{student}{student}
%% For mine, use \lecture{instructor}{instructor}


%\usepackage{pgf,pgfpages}
%\pgfpagesuselayout{4 on 1}[letterpaper,border shrink=5mm]

% FONTS
\usepackage{fontspec}
\def\mainfont{Linux Biolinum O}
\setmainfont[Ligatures=TeX, Contextuals={NoAlternate}, BoldFont={* Bold}, ItalicFont={* Italic}, Numbers={Proportional}]{\mainfont}
\setsansfont[Scale=MatchLowercase]{Linux Biolinum O} 
\usepackage{microtype}

\usepackage{graphicx}
	\graphicspath{%
	{/Users/goby/Pictures/teach/466/lectures/}%
	{img/}%
	{/Users/goby/Pictures/teach/common/}} % set of paths to search for images

\usepackage{amsmath,amssymb}

%\usepackage{units}

\usepackage{booktabs}
\usepackage{multicol}
%	\setlength{\columnsep=1em}

\usepackage{array}
\newcolumntype{L}[1]{>{\raggedright\let\newline\\\arraybackslash\hspace{0pt}}p{#1}}
\newcolumntype{C}[1]{>{\centering\let\newline\\\arraybackslash\hspace{0pt}}p{#1}}
\newcolumntype{R}[1]{>{\raggedleft\let\newline\\\arraybackslash\hspace{0pt}}p{#1}}

%\usepackage{textcomp}
%\usepackage{setspace}
%\usepackage{tikz}
%	\tikzstyle{every picture}+=[remember picture,overlay]

\mode<presentation>
{
  \usetheme{Lecture}
  \setbeamercovered{invisible}
  \setbeamertemplate{items}[square]
}

\usepackage{calc}
\usepackage{hyperref}

\newcommand\HiddenWord[1]{%
	\alt<handout>{\rule{\widthof{#1}}{\fboxrule}}{#1}%
}

\newcommand\imagecredit[1]{%
	\vskip0pt plus 1filll \tiny #1}%
	

\begin{document}


\lecture{instructor}{instructor}

{
\begin{frame}[t,plain]{Grassland bird richness increases with habitat area.}
	\centering
		\includegraphics[width=0.85\textwidth]{grassland_fragment_size}

	\imagecredit{\hfill Herkert 1994. Ecol. Appl. 4: 461.}
\end{frame}
}

\lecture{student}{student}

{
\begin{frame}[t,plain]{Island bird richness increases with island area.}
	\centering
		\includegraphics[height=0.85\textheight]{island_species_area}

	\imagecredit{\hfill Fig. 20–6 of textbook.}
\end{frame}
}

{
\begin{frame}[t,plain]{A model of island biogeography provides a framework for conservation management.}
	\centering
		\includegraphics[height=0.8\textheight]{island_biogeography_model}

	\imagecredit{\hfill Fig. 20–5 of textbook.}
\end{frame}
}


{

\begin{frame}[t,plain]{Global forest cover is decreasing, especially in SE Asia.}
	\begin{center}
		\includegraphics[width=1\textwidth]{forest_global_loss}
	\end{center}	
	\imagecredit{Global Forest Watch}
\end{frame}
}

{
\begin{frame}[t,plain]{The U.S. was heavily deforested in the late 1800s.}
	\begin{center}
		\includegraphics[width=1\textwidth]{forest_land_trends}
	\end{center}	
	\imagecredit{U.S. Forest Service.}
\end{frame}
}

{
\usebackgroundtemplate{\includegraphics[width=\paperwidth]{forest_us} }
\begin{frame}[t,plain]{Forest density in the U.S. remains greatly decreased}
	\imagecredit{\hfill NASA Earth Observatory.}
\end{frame}
}

%% Carolina Parakeet, Ivory-billed Woodpecker by James St. John, Wikimedia Commons
%% Passenger Pigeon, Cornell University Museum, Flickr CC
%% Bachman's Warbler Louis Agassiz Fuertest, public domain

%Kirtland's Warbler, USFWS, FCC
% Distribution map, Thayerbirding.com

{
\usebackgroundtemplate{\includegraphics[width=\paperwidth]{extinct_birds} }
\begin{frame}[t,plain]{}
\imagecredit{\textcolor{white}{Wikimedia Commons, Cornell University.}}
\end{frame}
}

{
\usebackgroundtemplate{\includegraphics[width=\paperwidth]{kirtlands_warbler} }
\begin{frame}[t,plain]{}
\imagecredit{US Fish \& Wildlife Service, Flickr Creative Commons.}
\end{frame}
}

\begin{frame}[t,plain]{Kirtland's Warbler requires specific breeding habitat.}
	\begin{center}
		\includegraphics[height=0.8\textheight]{kirtlands_warbler_distribution}
	\end{center}	

	\imagecredit{\hfill Stolen from \url{Thayersoftware.com}.}
\end{frame}

%% Flying flicker against blue sky is Jamie Chaves, FCC
{
\usebackgroundtemplate{\includegraphics[width=\paperwidth]{bosque_del_apache} }
\begin{frame}[t,plain]{}
	\imagecredit{\hfill\textcolor{white}{Gail Diane Yovanovich, National Wildlife Refuge Association.}}
\end{frame}
}

\lecture{instructor}{instructor}

{
\usebackgroundtemplate{\includegraphics[width=\paperwidth]{reserve_design_1} }
\begin{frame}[t,plain]{}
	\imagecredit{Fig. 21–13 of textbook.}
\end{frame}
}

{
\usebackgroundtemplate{\includegraphics[width=\paperwidth]{reserve_design_2} }
\begin{frame}[t,plain]{}
	\imagecredit{Fig. 21–13 of textbook.}
\end{frame}
}

\lecture{student}{student}

\begin{frame}[t,plain]{Songbird fledging success increases in larger fragments.}
	\begin{center}
		\includegraphics[width=1\textwidth]{forest_edge_fledging_success}
	\end{center}	

	\imagecredit{\hfill Fig. 21–19 of textbook.}
\end{frame}

\begin{frame}[t,plain]{Nest parasitism decreases with increased forest cover.}
	\begin{center}
		\includegraphics[width=1\textwidth]{forest_parasitism}
	\end{center}	

	\imagecredit{\hfill Fig. 21–18 of textbook.}
\end{frame}

{
\usebackgroundtemplate{\includegraphics[width=\paperwidth]{scrub_jay} }
\begin{frame}[t,plain]{}
	\imagecredit{\textcolor{white}{Andrea Westmoreland, Flickr Creative Commons.}}
\end{frame}
}

\begin{frame}[t,plain]{The Florida Scrub-Jay distribution forms a metapopulation.}
	\centering
		\includegraphics[height=0.8\textheight]{scrub_jay_distribution}

	\imagecredit{\hfill Fig. 21–14 of textbook.}
\end{frame}


\begin{frame}[t,plain]{Florida Scrub-Jay habitat requires burn management.}
	\centering
		\includegraphics[height=0.8\textheight]{scrub_jay_life_table}

	\imagecredit{\hfill U.S. Fish \& Wildlife Service.}
\end{frame}

\begin{frame}[t,plain]{PVA shows 40 or more territories are needed to sustain a population.}
	\centering
		\includegraphics[height=0.8\textheight]{scrub_jay_pva}

	\imagecredit{\hfill Fig. 21–15 of textbook.}
\end{frame}

\begin{frame}[t,plain]{Unsuitable habitat forces jays to disperse greater distances.}
	\begin{center}
		\begin{tabular}{@{}L{5cm}C{2.5cm}C{2.5cm}@{}}
		\toprule
		\vfill Habitat Type 	&	Normal Dispersal Distance (km)	&	Maximum Dispersal Distance (km) \\
		\midrule
		Breeding pair dispersal & 0 & \textemdash \\
		Open water	&	2	&	2 \\
		Urban areas	&	2	&	2  \\
		Dense pine forest	&	2	&	3  \\
		Unbroken, open pasture	&	3	&	7  \\
		Cropland	&	3	&	7  \\
		Densely wooded suburbs	&	5	&	8  \\
		Suburbs with few trees	&	5	&	13  \\
		Broken pasture, fence rows	&	8	&	24  \\
		Overgrown scrub 	&	8	&	24  \\
		\bottomrule
		\end{tabular}
	\end{center}
	\imagecredit{\hfill U.S. Fish \& Wildlife Service.}
\end{frame}

\begin{frame}[t,plain]{Patch dynamics across landscape mosaics increases diversity.}
	\centering
		\includegraphics[width=1\textwidth]{forest_mosaic}

	\imagecredit{\hfill Fig. 21–23 of textbook.}
\end{frame}

\begin{frame}[t,plain]{Clear-cutting and regeneration creates a diverse landscape.}
	\centering
		\includegraphics[width=0.9\textwidth]{forest_mosaic_clearcut}

	\imagecredit{\hfill\copyright Nature Education.}
\end{frame}

\begin{frame}[t,plain]{Landscape mosaics contain an array of habitat types.}
	\centering
		\includegraphics[width=1\textwidth]{landscape_mosaic}

	\imagecredit{\hfill Federal Interagency Stream Restoration Working Group. 1998.}
\end{frame}

\begin{frame}[t,plain]{Landscape mosaics increases bird diversity.}
	\centering
		\includegraphics[height=0.85\textheight]{landscape_mosaic_bird_community}

	\imagecredit{\hfill Mabry et al. 2010. Forest Ecol. Manage. 260: 42.}
\end{frame}


\begin{frame}[t,plain]{Migratory birds require global conservation networks.}
	\centering
		\includegraphics[height=0.85\textheight]{important_bird_areas}

	\imagecredit{\hfill Fig. 21–25 of textbook.}
\end{frame}

{
\usebackgroundtemplate{\includegraphics[width=\paperwidth]{western_shorebird_network} }
\begin{frame}[t,plain]{The Western Hemisphere Shorebird Reserve network targets wetland birds.}
\end{frame}
}



\end{document}
