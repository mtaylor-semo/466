%!TEX TS-program = lualatex
%!TEX encoding = UTF-8 Unicode

\documentclass[t]{beamer}

%%%% HANDOUTS For online Uncomment the following four lines for handout
%\documentclass[t,handout]{beamer}  %Use this for handouts.
%\includeonlylecture{student}
%\usepackage{handoutWithNotes}
%\pgfpagesuselayout{3 on 1 with notes}[letterpaper,border shrink=5mm]
%	\setbeamercolor{background canvas}{bg=black!5}


%%% Including only some slides for students.
%%% Uncomment the following line. For the slides,
%%% use the labels shown below the command.

%% For students, use \lecture{student}{student}
%% For mine, use \lecture{instructor}{instructor}

% FONTS
\usepackage{fontspec}
\def\mainfont{Linux Biolinum O}
\setmainfont[Ligatures=TeX, Contextuals={NoAlternate}, BoldFont={* Bold}, ItalicFont={* Italic}, Numbers={Proportional, OldStyle}]{\mainfont}
\setsansfont[Scale=MatchLowercase]{Linux Biolinum O} 
\usepackage{microtype}

\usepackage{graphicx}
	\graphicspath{%
	{/Users/goby/Pictures/teach/466/lectures/}}%
%	{/Users/goby/Pictures/teach/common/}} % set of paths to search for images

\usepackage{amsmath,amssymb}

%\usepackage{units}

\usepackage{booktabs}
\usepackage{multicol}
%	\setlength{\columnsep=1em}

%\usepackage{textcomp}
%\usepackage{setspace}
\usepackage{tikz}
	\tikzstyle{every picture}+=[remember picture,overlay]

\mode<presentation>
{
  \usetheme{Lecture}
  \setbeamercovered{invisible}
  \setbeamertemplate{items}[square]
}

%\usepackage{calc}
\usepackage{hyperref}


\newcommand{\cornell}[1]{Fig.~#1~Lovette and Fitzpatrick, 2016. 3rd ed.}
\begin{document}
%\lecture{instructor}{instructor}

\lecture{student}{student}
{
\usebackgroundtemplate{\includegraphics[width=\paperwidth]{food_bee_eater_intro}}
\begin{frame}[b,plain]{\textcolor{yellow}{Food and foraging}}
	
	\tinyfill{\href{https://commons.wikimedia.org/w/index.php?curid=20802692}{\textcolor{white}{European Bee-eater by Pierre Dalous, Wikimedia Commons, \ccbysa{3}}}} 
\end{frame}
}

%% functional response
\begin{frame}[t,plain]{\highlight{Functional response} describes how feeding rate changes with food availability.}

	\vspace{-0.5\baselineskip}
	\centering 
	\includegraphics[height=0.8\textheight]{food_functional_response}
	
	\vfilll
	
	\tinyfill{\cornell{8.02}}
\end{frame}

%% Foraging process
\begin{frame}[t,plain]{Birds use a variety of ecological and evolutionary foraging strategies.}

	\vspace{-0.25\baselineskip}
	\centering 
	\includegraphics[height=0.78\textheight]{food_foraging_process}
	
	\vfilll
	
	\tinyfill{\cornell{8.08}}
\end{frame}



%% LEvy flight
\begin{frame}[t,plain]{Large areas can be covered with combination of random and directed flight.}

%	\vspace{-0.5\baselineskip}
	\centering 
	\includegraphics[width=\linewidth]{food_levy_flight}
	
	\medskip
	
	\reflectbox{\includegraphics[width=4cm]{food_black-browed_albatross}}
	\vfilll
	
	\tiny Black-browed Albatross by 0ystercatcher, \href{https://www.flickr.com/photos/49828152@N00/9172107327}{Flickr}, \ccbyncsa{2} \tinyfill{\cornell{8.04}}
\end{frame}



%% Stable strategy
\begin{frame}[t,plain]{Alternate foraging methods converge on \highlight{evolutionary stable strategy.}}

%	\vspace{-0.5\baselineskip}
	\centering 
	\includegraphics[width=\linewidth]{food_finder_Scrounger}

	\vfilll
	
	\tiny Scaly-breasted Munias \tinyfill{\cornell{8.07}}
\end{frame}

%% flush pursuit
\begin{frame}[t,plain]{Some birds use \highlight{flush-pursuit foraging.}}

%	\vspace{-0.5\baselineskip}
	\begin{multicols}{2}
	\includegraphics[width=\linewidth]{food_redstart_foraging_success}

	\columnbreak

	\includegraphics[width=\linewidth]{food_slate_throated_redstart}
	\end{multicols}

	\hangpara Our Northern Mockingbird uses this strategy (\href{https://www.youtube.com/watch?v=GF5xBvmeOjo}{video}).

	\vfilll
	
	\tiny Slate-throated Redstart by Alan Schmierer \href{https://www.flickr.com/photos/8101022@N05/14107576517}{Flickr public domain.} \tinyfill{\cornell{8.10}}
\end{frame}

%% sit and wait
\begin{frame}[t,plain]{Some birds \highlight{sit-and-wait.}}

%	\vspace{-0.5\baselineskip}
	\begin{multicols}{2}
	\reflectbox{\includegraphics[width=\linewidth]{food_great_blue_heron}}

	\columnbreak

	\includegraphics[width=\linewidth]{food_red_shouldered_hawk}
	\end{multicols}


	\vfilll
	
	\tiny Great Blue Heron, \textsc{usfws} \href{https://www.flickr.com/photos/8101022@N05/14107576517}{Flickr public domain.} \hfill Red-shouldered Hawk, Charles Patrick Ewing,  \href{https://www.flickr.com/photos/132033298@N04/31366747854}{Flickr} \ccby{2}
\end{frame}

%% baiting
\begin{frame}[t,plain]{Some birds use bait to attract prey.}

%	\vspace{-0.5\baselineskip}
	\includegraphics[width=\linewidth]{food_bait_tool}

	\vfilll
	
	\tinyfill{\cornell{8.11}}
\end{frame}

%% Foraging risk

\begin{frame}[t]{Foraging increases risk of becoming prey.}

	\includegraphics[width=\linewidth]{food_innate_response}

	\vspace{1.5\baselineskip}
	
	\centering
	
	\includegraphics[width=5cm]{food_black_capped_chickadee}
	
	\vfilll
	
	\tiny Black-capped Chickadee, DaPuglet \href{https://www.flickr.com/photos/43810158@N07/33349783941}{Flickr} \ccbysa{2} \tinyfill{\cornell{8.14}}

\end{frame}

%%
{
\usebackgroundtemplate{\includegraphics[width=\paperwidth]{food_capture}}
\begin{frame}[b,plain]{Once found, food must be captured.}
	
	\tinyfill{\textcolor{white}{\cornell{8.18}}}
\end{frame}
}

%%

\begin{frame}[t]{Some species make and use tools to extract food.}

	\vspace{-0.5\baselineskip}

	\begin{multicols}{2}
	\includegraphics[width=\linewidth]{food_tool_caledonia_crow}
	
	\columnbreak

	\includegraphics[width=\linewidth]{food_tool_woodpecker_finch}
	
	\end{multicols}

	\vfilll
	
	\tiny New Caledonia Crow Fig.~8.B4.01  
	\tinyfill{\cornell{8.14}}

\end{frame}

%%

\begin{frame}[t]{Rhynchokinesis is found in shorebirds and a few other groups.}

\vspace{-0.5\baselineskip}

\centering

\includegraphics[height=0.78\textheight]{food_rhynchokinesis}

\vfilll


\tiny \href{https://www.youtube.com/watch?v=RggFFi2MMt0}{Link to video}\tinyfill{\cornell{8.22}}

\end{frame}

%% Woodpeckers
\begin{frame}[t]{Woodpeckers have \emph{very} long tongues to extract food.}

\begin{multicols}{2}
\includegraphics[width=\linewidth]{food_woodpecker_hyoid}

\columnbreak

\href{https://macaulaylibrary.org/asset/225045361}{\includegraphics[width=\linewidth]{food_downy_sonogram}}\\
\href{https://www.youtube.com/watch?v=bqBxbMWd8O0}{Why don't woodpeckers get headaches? (video)}

\end{multicols}

\vfilll

\tinyfill{\cornell{8.23}}
\end{frame}

%% Tongues

\begin{frame}[t]{Many species have specialized tongues.}

\vspace{-0.5\baselineskip}

\centering

\includegraphics[height=0.85\textheight]{food_tongues}

\vfilll

\tinyfill{\cornell{8.24}}

\end{frame}


%% Bill morphology
\begin{frame}[t]{Bill morphology often reflects specialized diet.}

\vspace{-0.5\baselineskip}

\begin{multicols}{2}
\includegraphics[width=\linewidth]{food_red_crossbill}

\columnbreak

\includegraphics[width=\linewidth]{food_flamingo}

\end{multicols}


\vfilll

\tiny Red Crossbill, Ryan Schain, \href{https://macaulaylibrary.org/asset/42506401}{Macaulay Library ML42506401.} \hfill Flamingo, Robert Claypool, \href{https://www.flickr.com/photos/35106989@N08/8058547935}{Flickr}, \ccby{2}
\end{frame}

{
\usebackgroundtemplate{\includegraphics[width=\paperwidth]{food_kleptoparasitism}}
\begin{frame}[b]
	
	\tiny \href{https://www.youtube.com/watch?v=zKzeMovBwL0}{Link to Video}\tinyfill{Magnificent Frigatebird by Kurt Bauschardt, \href{https://www.flickr.com/photos/48503330@N08/5662065683}{Flickr} \ccbysa{2}}
\end{frame}
}



\begin{frame}[t]{Some birds mimic other birds to scare yet other birds.}

\vspace{-\baselineskip}

\begin{multicols}{2}
\centering
\includegraphics[width=\linewidth]{food_red_shouldered_hawk2}

\columnbreak

\includegraphics[width=\linewidth]{food_blue_jay}
\end{multicols}

\vfilll

\tiny \textsc{rsha} by Charles Patrick Ewing \href{https://www.flickr.com/photos/132033298@N04/39669645164}{Flickr} \ccby{2} \hfill \href{https://www.youtube.com/watch?v=8gmVOMEhMj8}{Link to video} \hfill{Blue Jay by \href{https://www.flickr.com/photos/45964493@N04/5304330230}{Michael Baglole, Flickr} \ccbyncsa{2}}
\end{frame}

\begin{frame}[t]

\centering
\includegraphics[height=0.92\textheight]{food_dominance_guilds}

\vfilll

{\tinyfill{\cornell{8.B7.01}}}

\begin{tikzpicture}
\node at (-5, 9) [text width = 2cm] {\highlight{Dominance structure}};
\node at (5, 9) [text width = 2cm] {\highlight{Foraging guilds}};
\end{tikzpicture}


\end{frame}


\end{document}

