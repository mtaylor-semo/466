%!TEX TS-program = lualatex
%!TEX encoding = UTF-8 Unicode

\documentclass[t]{beamer}

%%%% HANDOUTS For online Uncomment the following four lines for handout
%\documentclass[t,handout]{beamer}  %Use this for handouts.
%\includeonlylecture{student}
%\usepackage{handoutWithNotes}
%\pgfpagesuselayout{3 on 1 with notes}[letterpaper,border shrink=5mm]
%	\setbeamercolor{background canvas}{bg=black!5}


%%% Including only some slides for students.
%%% Uncomment the following line. For the slides,
%%% use the labels shown below the command.

%% For students, use \lecture{student}{student}
%% For mine, use \lecture{instructor}{instructor}

% FONTS
\usepackage{fontspec}
\def\mainfont{Linux Biolinum O}
\setmainfont[Ligatures=TeX, Contextuals={NoAlternate}, BoldFont={* Bold}, ItalicFont={* Italic}, Numbers={Proportional, OldStyle}]{\mainfont}
\setsansfont[Scale=MatchLowercase]{Linux Biolinum O} 
\usepackage{microtype}

\usepackage{graphicx}
	\graphicspath{%
	{/Users/goby/Pictures/teach/466/lectures/}}%
%	{/Users/goby/Pictures/teach/common/}} % set of paths to search for images

\usepackage{amsmath,amssymb}

%\usepackage{units}

\usepackage{booktabs}
\usepackage{multicol}
%	\setlength{\columnsep=1em}

%\usepackage{textcomp}
%\usepackage{setspace}
\usepackage{tikz}
	\tikzstyle{every picture}+=[remember picture,overlay]
\usetikzlibrary{calc}

\mode<presentation>
{
  \usetheme{Lecture}
  \setbeamercovered{invisible}
  \setbeamertemplate{items}[square]
}

%\usepackage{calc}
\usepackage{hyperref}


\newcommand{\cornell}[1]{Fig.~#1~Lovette and Fitzpatrick, 2016. 3rd ed.}

\newcommand{\backskip}{\vspace{-0.5\baselineskip}}

\begin{document}
%\lecture{instructor}{instructor}

\lecture{student}{student}
{
\usebackgroundtemplate{\includegraphics[width=\paperwidth]{vocals_singing_intro}}
\begin{frame}[b,plain]
	
	\tinyfill \textcolor{white}{Marsh Wren by Still Vision, \href{https://www.flickr.com/photos/27286306@N07/7616501040}{Flickr}, \ccby{2}} 
\end{frame}
}

%%

{
\usebackgroundtemplate{\includegraphics[width=\paperwidth]{vocals_signals}}
\begin{frame}[b,plain]{\highlight{Vocal signals} influence the behavior of others.}
	
	\tinyfill \textcolor{white}{House Sparrows, Mitch Surprenant, \href{https://www.flickr.com/photos/14236207@N04/2053598066}{Flickr}, \ccbyncsa{2}} 
\end{frame}
}

%%

\begin{frame}[t]{Sound is pressure vibrations that travel through a medium.}

\includegraphics[width=\linewidth]{vocals_sound_wave}


\end{frame}

%%

\begin{frame}[c]{\highlight{Frequency (Hz)} is the number of waves produced in a fixed time.}


\begin{tikzpicture}
\draw[x=1cm,y=1cm, ultra thick, blue, yshift=-0.25cm] 
        (0,2) sin (1,3) cos (2,2) sin (3,1) cos (4,2)
              sin (5,3) cos (6,2) sin (7,1) cos (8,2)
              sin (9,3) cos (10,2) sin (11,1) cos (12,2);
              
\draw[x=0.5cm,y=1cm, ultra thick, blue] 
        (1,-1) sin (2,0) cos (3,-1) sin (4,-2) cos (5,-1)
               sin (6,0) cos (7,-1) sin (8,-2) cos (9,-1)
               sin (10,0) cos (11,-1) sin (12,-2) cos (13,-1)
               sin (14,0) cos (15,-1) sin (16,-2) cos (17,-1)
               sin (18,0) cos (19,-1) sin (20,-2) cos (21,-1)
               sin (22,0) cos (23,-1);
               
\draw[x=1cm, y=1cm, thick]
		(1,-2.5) node [below] {\Large 0s} -- (1,3.25) ;
\draw[x=1cm, y=1cm, thick]
		(9,-2.5) node [below] {\Large 1s}-- (9,3.25);
\end{tikzpicture}

\end{frame}

%%

\begin{frame}{Birds hear a slightly narrower frequency range with \emph{much} better resolution.}

%\backskip

\includegraphics[width=\linewidth]{vocals_audiogram}

%\begin{tikzpicture}[yshift=0.5cm]
%
%\draw (6,0) -- (6,6.67) node [above, yshift=6pt] {Frequency (kHz)};
%\foreach \y in {0,1,...,20}
%	\draw (5.9, \y/3) -- (6.1,\y/3);
%\foreach \y in {0,2,...,21}
%	\draw (6,\y/3) -- (6,\y/3) node [right] {\y};
%
%\draw (7.1,5) -- (7.5,5) node [right] {15 kHz};
%\draw (7.1,0.02) -- (7.5,0.02) node [right] {20 Hz};
%\draw [<->] (7.3,0.09) -- (7.3,4.93) node [midway, right] {Most humans};
%
%\draw (4.5,1.67) node [left] {5 khz} -- (4.9,1.67);
%\draw (4.5,0.33) node [left] {1 khz} -- (4.9,0.33);
%\draw [<->] (4.7,0.40) -- (4.7,1.6) node [midway, left] {Most birds};
%
%\end{tikzpicture}

\end{frame}

%%

\begin{frame}{\highlight{Spectrograms} are visual representations of sound.}
%\backskip

\includegraphics[width=0.95\linewidth]{vocals_spectrogram_chickadee}

\vfilll

\tiny \href{https://youtu.be/baCamXF-610}{Audio} \hfill \cornell{10.02}
\end{frame}

%%

\begin{frame}{Single-frequency notes sound whistle-like. Added harmonics add harsh buzz.}

%\backskip


\includegraphics[width=\linewidth]{vocals_clear_buzzy}

\vfilll

\tiny \href{https://youtu.be/m5hHwTl8-Zo}{White-throated Sparrow} \quad \href{https://youtu.be/oNKb8rIYCoQ}{Black-capped Chickadee} \hfill \cornell{10.03, 10.07}
\end{frame}

%%


\begin{frame}{Tics, taps, and other “sharp” sounds appear as vertical lines.}
%\backskip

\centering
\noindent \includegraphics[width=0.88\linewidth]{vocals_woodpecker_taps}

\vfilll

\tiny \href{https://youtu.be/ygT-Nk_eyJQ}{Downy Woodpecker (top)} \quad \href{https://youtu.be/GHwtWug4998}{Hairy Woodpecker (bottom)}\hfill \cornell{10.04}
\end{frame}

%%

{
\usebackgroundtemplate{\includegraphics[width=\paperwidth]{vocals_notes_phrases}}
\begin{frame}[b,plain]


	\tiny \href{https://youtu.be/alpYJ6M7XYk}{Common Yellowthroat} \quad \href{https://youtu.be/6dj6rBncpkY}{Half speed} \textcolor{white}{\cornell{10.05}} 
\end{frame}
}

%%

\begin{frame}{A song can consist of many distinct phrases.}
\centering

\includegraphics[height=0.83\textheight]{vocals_pacific_wren_song}

\vfilll

\tiny \href{https://youtu.be/xHymewtefgk}{Pacific Wren} \quad \href{https://youtu.be/lUpax9rI3-s}{Half speed}
\hfill \cornell{10.06}
\end{frame}

%%

{\setbeamercolor{background canvas}{bg=black}
\begin{frame}{Superb Lyrebird}
\href{https://www.youtube.com/watch?v=VjE0Kdfos4Y}{%
\includegraphics[width=\linewidth]{vocals_superb_lyrebird}}

\vfilll

\tinyfill \textcolor{white}{Superb Lyrebird by Alex Maisey}
\end{frame}
}

\begin{frame}{The \highlight{syrinx} enables birds to make complex vocalizations.}

\backskip

\begin{multicols}{2}

\centering
\noindent \includegraphics[width=0.9\linewidth]{vocals_syrinx_structure}

\columnbreak

\includegraphics[width=0.95\linewidth]{vocals_cardinal_throat}

\end{multicols}

\vfilll

\tiny \cornell{10.25} \hfill \href{https://academy.allaboutbirds.org/wood-thrush-singing/}{Wood Thrush} \hfill \href{https://www.youtube.com/watch?v=o0mATRdzZSc}{Hermit Thrush}\hfill Riede et al.~2006. P\textsc{nas} 103:5543.
\end{frame}

%%

\begin{frame}{\highlight{Oscine} songbirds have additional musculature on the syrinx compared to “non-songbirds.”}

\backskip
\centering
\includegraphics[height=0.78\textheight]{vocals_syrinx_oscine_suboscine}

\vfilll

\tiny \href{https://youtu.be/cOi35DdUNS4}{Eastern Wood-Pewee}
\quad
\href{https://macaulaylibrary.org/asset/112641}{Scott's Oriole} 
\hfill 
\href{https://meetings.ami.org/2018/project/syrinx-musculature-compared-in-oscine-vs-suboscine-passerines/}{Illustration \textcopyright\,Sarah Mientka, Assoc.~Medical Illustrators, 2018.}
\end{frame}

%%

\begin{frame}{Nearly all birds vocalize.}

\begin{multicols}{2}

\hangpara \highlight{Calls} are relatively short and simple.

\hangpara \highlight{Songs} are often loud and sometimes complex.

\hangpara \highlight{Non-vocal} sounds just that.

\columnbreak


\noindent \includegraphics[width=\linewidth]{vocals_bluejay_calling}
\end{multicols}

\vfilll

\tinyfill \href{https://www.flickr.com/photos/98774255@N00/5925119364}{Panting Blue Jay, Indiana Ivy Nature Photography, Flickr, \ccby{2}}

\end{frame}


\begin{frame}{Songs or calls?}

\begin{multicols}{3}

\noindent\reflectbox{\includegraphics[width=\linewidth]{vocals_icterid_cogr}}

\tiny\href{https://www.flickr.com/photos/43581314@N08/33973682813}{Tim Sackton, Flickr, \ccbysa{2}}

\columnbreak

\noindent\includegraphics[width=\linewidth]{vocals_icterid_rwbl}

\centering

\href{https://www.flickr.com/photos/122259497@N03/14022247867}{nature80020, Flickr, \ccby{2}}
\columnbreak


\noindent\includegraphics[width=\linewidth]{vocals_icterid_oror}

\hfill \href{https://www.flickr.com/photos/10017367@N03/5611912358}{Dan Pancamo, Flickr, \ccbysa{2}}


\end{multicols}


\vfilll

\tiny \href{https://macaulaylibrary.org/asset/133330}{Common Grackle}
\hfill
\href{https://macaulaylibrary.org/asset/314523871}{Red-winged Blackbird}
\hfill
\href{https://macaulaylibrary.org/asset/94445}{Orchard Oriole}
\end{frame}

%%

\begin{frame}{Calls serve a variety of social functions.}

\hangpara \href{https://youtu.be/oNKb8rIYCoQ}{\highlight{Contact calls}} calls keep the group in contact.

\hangpara \href{https://youtu.be/j-2vQ5mg2vo}{\highlight{Agression calls} occur between individuals of same or different sex.}

\hangpara \href{https://youtu.be/l_w_3AbkuV0?t=46}{\highlight{Alarm calls}} warn others of potential danger.

\hangpara \highlight{Begging calls} of nestlings and fledglings.

\end{frame}

%%

\begin{frame}{Most birds have a few to many \highlight{song types.}}

\backskip

\centering

\includegraphics[height=0.85\textheight]{vocals_song_types}

\vfilll

\tiny \href{https://youtu.be/pdMrd5-vbf4}{Common Nightingale} 
\quad 
\href{https://youtu.be/erdU5AIYbNk}{Brown Thrasher}
\hfill
\cornell{10.11}

\end{frame}


%%

\begin{frame}{Singing behaviors worth remark.}

\begin{multicols}{2}
\hangpara \href{https://www.youtube.com/watch?v=r6_LYIdYxz4}{\highlight{Dawn chorus}}\\[1ex]
\hspace{1em}\href{https://academy.allaboutbirds.org/peterson-field-guide-to-bird-sounds/?speciesCode=sumtan\&species=Summer\%20Tanager\%20-\%20Piranga\%20rubra}{Dawn vs day songs}

\hangpara \highlight{Duets}\\[1ex]
\hspace{1em}\href{https://academy.allaboutbirds.org/peterson-field-guide-to-bird-sounds/?speciesCode=placha\&species=Plain\%20Chachalaca\%20-\%20Ortalis\%20vetula}{Plain Chachalaca}\\
\hspace{1em}\href{https://macaulaylibrary.org/asset/28907}{Buff-breasted Wren (3:36)}

\hangpara \href{}{\highlight{Flight songs}}\\[1ex]
\hspace{1em}\href{https://youtu.be/9pcvhDKDKv4}{Eurasian Sky Lark}\\
\hspace{1em}\href{https://macaulaylibrary.org/asset/239370861}{Bobolink}\\

\hangpara \highlight{Mimicry}\\[1ex]
\hspace{1em} \href{https://macaulaylibrary.org/asset/243816281}{Northern Mockingbird}

\columnbreak

\centering
\noindent \includegraphics[width=0.85\linewidth]{vocals_bobolink}

\end{multicols}

\vfilll

\tinyfill \href{https://commons.wikimedia.org/w/index.php?curid=26015716}{Bobolink, Andrea Westmoreland, Wikimedia, \ccbysa{2}}


\end{frame}
%%
\begin{frame}{Some birds produce \highlight{non-vocal} sounds.}

\begin{multicols}{2}

Most are used for courtship or territory.

\hangpara Woodpecker drumming

\hangpara \href{https://www.youtube.com/watch?v=2MIoXh_ORMw}{Palm Cockatoo drumming}

\hangpara \href{https://www.youtube.com/watch?v=9BrJghW-WEk}{Wing beats}

\hangpara \href{https://youtu.be/-_1pyFEYptI}{Wing and tail feathers}

\columnbreak

\centering

\noindent \reflectbox{\includegraphics[width=0.9\linewidth]{vocals_nonvocal_rugr}}

\smallskip

\noindent\includegraphics[width=0.9\linewidth]{vocals_nonvocal_wisn}

\end{multicols}

\vfilll

\tiny \href{https://www.youtube.com/watch?v=tSHjhCN6NC0}{Club-tailed Manakin} 
\hfill
\href{https://www.flickr.com/photos/80223459@N05/15116108746}{Ruffed Grouse, YellowstoneNPS, Flickr, \textsc{cc-pdm 1.0}}
\hfill
\href{https://www.flickr.com/photos/22170893@N06/11635736575}{Wilson's Snipe, Gregory Smith, Flickr, \ccbysa{2}}

\end{frame}

%%

\begin{frame}{Songs are innate in a few species of birds.}

\includegraphics[width=\linewidth]{vocals_song_innate}

\vfilll

\tinyfill \cornell{10.18}
\end{frame}

%%

\begin{frame}{Songs are learned by most species of birds.}
\includegraphics[width=\linewidth]{vocals_song_learned}

\hangpara Most birds learn during a \highlight{sensitive period.}\\ Some species are \highlight{open-ended learners.}

\vfilll

\tinyfill \cornell{10.16}
\end{frame}

%%

\begin{frame}{Birds learn through exposure and practice.}

\includegraphics[width=\linewidth]{vocals_practice}

\vfilll

\tiny \href{https://youtu.be/gVYpcFAcpLo}{Bewick's Wren practice} 
\quad
\href{https://youtu.be/bCQVNalg93Y}{Adult} \hfill  \cornell{10.17}
\end{frame}

%%

\begin{frame}{Adult birds switch to local dialects.}
\backskip

\centering

\includegraphics[height=0.85\textheight]{vocals_dialect_switching}

\vfilll

\tiny \href{https://youtu.be/q7P3LX0V1DQ}{Adult}\quad\href{https://youtu.be/lP7sfF9mzCU}{Neighbor} \hfill \cornell{0.19}
\end{frame}

%%

\begin{frame}{Some birds share song types among neighbors.}
\centering
\includegraphics[height=0.8\textheight]{vocals_shared_songtypes}

\vfilll

\tinyfill \cornell{10.20}
\end{frame}

%%

\begin{frame}{Territorial males are more aggressive toward strangers.}

\backskip
\centering 
\includegraphics[height=0.85\textheight]{vocals_stranger_respond}

\vfilll

\tinyfill \cornell{10.28}
\end{frame}

%%
\tikzset{
    hyperlink node/.style={
        alias=sourcenode,
        append after command={
            let     \p1 = (sourcenode.north west),
                \p2=(sourcenode.south east),
                \n1={\x2-\x1},
                \n2={\y1-\y2} in
            node [inner sep=0pt, outer sep=0pt,anchor=north west,at=(\p1)] {\href{#1}{\XeTeXLinkBox{\phantom{\rule{\n1}{\n2}}}}}
                    %xelatex needs \XeTeXLinkBox, won't create a link unless it
                    %finds text --- rules don't work without \XeTeXLinkBox.
                    %Still builds correctly with pdflatex and lualatex
        }
    }
}


\begin{frame}{Territorial males might use different song types for different functions.}

\includegraphics[width=\linewidth]{vocals_song_type_function}

%\begin{tikzpicture}
%\draw [ultra thick] (0,0) rectangle (4,4) node [draw, inner sep=2ex,hyperlink node=https://youtu.be/A8LLDqycato] {};

\tikz \node (rect) at (4.9,1.9) [draw, ultra thin, black, minimum width=4.2cm, minimum height=1.8cm, inner sep=2ex,hyperlink node=https://youtu.be/A8LLDqycato] {};

\tikz \node (rect) at (10,2.4) [draw, ultra thin, minimum width=3.8cm, minimum height=1.8cm, inner sep=2ex,hyperlink node=https://youtu.be/acxCsZxX0vo] {};
%\end{tikzpicture}



\vfilll

\tiny \href{https://youtu.be/A8LLDqycato}{Chestnut-sided Warbler attraction} \quad \href{https://youtu.be/acxCsZxX0vo}{aggression} \hfill \cornell{10.34}

\end{frame}

%%

\begin{frame}{Local ecological processes can cause local dialects to form.}

\backskip

\includegraphics[width=\linewidth]{vocals_local_dialects}

\vfilll

\tinyfill \cornell{10.31}
\end{frame}

%%

\begin{frame}{Regional dialects may suggest distinct species.}

%\backskip

\includegraphics[width=\linewidth]{vocals_regional_dialect}

\vfilll

\tiny \href{https://youtu.be/DHckNi-A8w8}{Marsh Wren western} 
\quad \href{https://youtu.be/cBEQStYB1xI}{eastern}
\hfill
\cornell{10.32}

\end{frame}
\end{document}


