%!TEX TS-program = lualatex
%!TEX encoding = UTF-8 Unicode

\documentclass[t]{beamer}

%%%% HANDOUTS For online Uncomment the following four lines for handout
%\documentclass[t,handout]{beamer}  %Use this for handouts.
%\includeonlylecture{student}
%\usepackage{handoutWithNotes}
%\pgfpagesuselayout{3 on 1 with notes}[letterpaper,border shrink=5mm]


%%% Including only some slides for students.
%%% Uncomment the following line. For the slides,
%%% use the labels shown below the command.

%% For students, use \lecture{student}{student}
%% For mine, use \lecture{instructor}{instructor}


%\usepackage{pgf,pgfpages}
%\pgfpagesuselayout{4 on 1}[letterpaper,border shrink=5mm]

% FONTS
\usepackage{fontspec}
\def\mainfont{Linux Biolinum O}
\setmainfont[Ligatures={Common,TeX}, Contextuals={NoAlternate}, Numbers={Proportional, OldStyle}]{\mainfont}
\setsansfont[Ligatures={Common,TeX}, Scale=MatchLowercase, Numbers={Proportional,OldStyle}, BoldFont={* Bold}, ItalicFont={* Italic},]\mainfont

\newfontface\lining[Numbers={Lining}]\mainfont
\usepackage{microtype}

\usepackage{graphicx}
	\graphicspath{%
	{/Users/goby/Pictures/teach/466/lectures/}%
	{/Users/goby/Pictures/teach/466/handouts/}%
	{/Users/goby/Pictures/teach/163/common/}} % set of paths to search for images

\usepackage{amsmath,amssymb}

%\usepackage{units}

\usepackage{booktabs}
\usepackage{multicol}
%	\setlength{\columnsep=1em}

\usepackage{textcomp}
\usepackage{setspace}
\usepackage{tikz}
	\tikzstyle{every picture}+=[remember picture,overlay]

\mode<presentation>
{
  \usetheme{Lecture}
  \setbeamercovered{invisible}
  \setbeamertemplate{items}[square]
}

\usepackage{calc}
\usepackage{hyperref}

\newcommand\HiddenWord[1]{%
	\alt<handout>{\rule{\widthof{#1}}{\fboxrule}}{#1}%
}



\begin{document}
%\lecture{instructor}{instructor}
%\lecture{student}{student}


{
\usebackgroundtemplate{\includegraphics[width=\paperwidth]{ornithology_intro}}
\begin{frame}[b,plain]
	\tiny\textcolor{gray}{Rose Breasted Grosbeak by Paul VanDerWerf, Flickr Creative Commons.}
\end{frame}
}

%% Contact Info
{
\usebackgroundtemplate{\includegraphics[width=\paperwidth]{mike_macaws} }
\begin{frame}[t]
	\large
	\vspace{2ex}
	\hangpara\hspace{15em} Dr.~Mike Taylor

	\hangpara\hspace{15em} \textsc{rh} 217

	\hangpara\hspace{15em} mtaylor@semo.edu
	
\end{frame}
}


\lecture{student}{student}
\begin{frame}[t,plain]{This textbook is \highlight{recommended} for this course.}
	\begin{center}
		\vspace{-\baselineskip}
		\includegraphics[height=0.8\textheight]{text_cover_lovette}
	\end{center}
\end{frame}

\begin{frame}[t,plain]{This field guide is \highlight{required} for this course.}
	\begin{center}
		\vspace{-\baselineskip}
		\includegraphics[height=0.8\textheight]{sibley_cover}
	\end{center}
\end{frame}


%% Grades
\begin{frame}[t]{You \highlight{earn} your grade with}
	\begin{center}\large\begin{tabular}{@{}lll@{}}
	UG		&	Three 75-point exams 				& 	50\% \\
			&	Lab exams and quizzes	  &   25\% \\
			&	Other assignments							& 	25\% \\
			&										&	\\
	G / HC	&	Three 75-point exams 			& 	50\% \\
			&	Lab exams and quizzes		&   25\% \\
			&	Other assignments					 		&		15\% \\
			&	Hotspot survey 								&		10\% \\
		\end{tabular}
	\end{center}

%\hangpara Two (?) lab exams, field quizzes, other assignments.

\end{frame}



%% Field Trips
\begin{frame}[t,plain]{We will take \highlight{several field trips} during the semester.}

	\vspace{-0.5\baselineskip}
	
	\hangpara \highlight{10 points per trip (one free miss).}
	
	\hangpara We will depart early. 
	
	\hangpara Bring your field guide.

	\hangpara Smart phones apps like Merlin may aid identification. You must otherwise keep your phones off and away.

	\hangpara I supply binoculars but you may use your own. 

	\hangpara Dress appropriately. We will go out in most weather conditions.
		
	\hangpara Some field trips may be on Monday if Wednesday forecast is inclement weather (rain, ice, or snow).
	
\end{frame}

\begin{frame}[t,plain]{Follow basic etiquette on field trips.}

	\vspace{-0.5\baselineskip}
	
	\hangpara Bring snacks. We stop only for birds and gas. Haul out \textit{all} trash.
	
	\hangpara No smoking.
	
	\hangpara Cells phones (smart or dumb) must be off. Use only for the birds.

	\hangpara Respect private property.
	
	\hangpara Speak quietly. Avoid sudden movements. Wear clothes with subdued colors.
	
	\hangpara Be on time. Do not dawdle.
	
	\hangpara Keep your focus on the birds.
	
\end{frame}

% Basic structure
\begin{frame}[t]{Planned structure of the course.}

\vspace{-\baselineskip}

\hangpara Monday usually lecture, Wednesday usually lab or field trip.

\hangpara Most lectures will last about an hour or so. Second hour may consist of a systematics lecture or field quiz.

\hangpara Field quizzes will be given in class, after every 1–2 field trips. They will assess your ability to identify \textit{known} birds by sight and sound. 

\hangpara Field guide quizzes will assess your ability to use your field guide for identification of \textit{unknown} birds.

\hangpara Lab exams will cover external anatomy, taxonomy and systematics, as appropriate.

\hangpara eBird and other assignments$\dots$

\end{frame}


{
\begin{frame}{You must open a free eBird account at \href{https://ebird.org}{ebird.org.}}
	\vspace{-\baselineskip}
	\begin{center}
		\includegraphics[width=\textwidth]{ebird_taylor}
	\end{center}
	
	Download and use eBird Mobile (iOS, Android). May also use Merlin ID.
	
	\highlight{Watch the tutorial videos on our Canvas page.}
%	
\end{frame}
}

{
\begin{frame}{All students must do these tasks.}

\vspace{-\baselineskip}

\hangpara Email me with your eBird user name. I will add you as a friend to the semo\_ornithology group account.

\hangpara Participate in the campus eBird project. See the separate handout for details.

\hangpara Bird at least 30 minutes during the \href{https://www.birdcount.org/}{Great Backyard Bird Count} (Feb~14-17). This is a separate requirement from above.

\hangpara Share every checklist you do for this class with semo\_ornithology.

\hangpara Upload one photo and one audio file to eBird.

\hangpara \href{https://support.ebird.org/en/support/solutions/articles/48000795623-ebird-rules-and-best-practices}{\highlight{Follow eBird best practices. (link)}}

\end{frame}
}

{
\begin{frame}{All \highlight{grad and honors} students must do these tasks.}

\vspace{-\baselineskip}

\hangpara Select an eBird hotspot, such as a conservation area, state park, or wildlife refuge. You may choose one in Cape county or close to where you live. Tell me your eBird chosen hotspot by the end of January. First come, first serve.

\hangpara Visit your chosen hotspot at least two hours each in February, March, April, and one hour during first week of May. You may bird longer.

\hangpara \href{https://support.ebird.org/en/support/solutions/articles/48000795623-ebird-rules-and-best-practices}{\highlight{Follow eBird best practices. (link)}} 

% \hangpara Your hotspot visits count towards your total checklists and time.

\hangpara Upload at least four media files (photo or audio) from the hotspot. At least one must be audio.

\hangpara Submit a final report that details the bird species observed at the hotspot during the visits and how they changed during the semester. 

\end{frame}
}

\begin{frame}{Particpate in the Great Backyard Bird Count for 30 minutes.}
	\includegraphics[width=\linewidth]{gbbc_banner}
	
	\highlight{14–17 February 2025.} \hfill Campus eBird stations do \emph{not} count as \textsc{gbbc.}
	
\end{frame}

\begin{frame}
	\centering
	\includegraphics[height=0.85\textheight]{campus_ebird_map1}
\end{frame}

\begin{frame}
	\centering
	\rotatebox{-90}{\includegraphics[width=0.9\textheight]{campus_ebird_map2}}
\end{frame}


{
\usebackgroundtemplate{\includegraphics[width=\paperwidth]{birding_basics}}
\begin{frame}[b,plain]
	\tiny\textcolor{yellow}{Common Yellowthroat by Don McCullough, Flickr Creative Commons.}
\end{frame}
}

\begin{frame}[b,plain]{Compare size and shape.}
	\begin{center}
		\includegraphics[height=0.8\textheight]{size_shape1}
	\end{center}
	\tiny Bird silhouettes from \href{http://allaboutbirds.com}{All About Birds, Cornell Lab of Ornithology}.
\end{frame}


{
\usebackgroundtemplate{\includegraphics[width=\paperwidth]{silhouettes_perched}}
\begin{frame}[t,plain]{Many birds can be \textsc{id}'d by size, shape, and behavior.}

	\vfilll
	
	\tiny\hfill\rotatebox{90}{Peterson \textit{A Field Guide to the Birds East of the Rockies.}}
\end{frame}
}

{
\usebackgroundtemplate{\includegraphics[width=\paperwidth]{silhouettes_flight}}
\begin{frame}[t,plain]{Many birds can be \textsc{id}'d by size, shape, and behavior.}

	\vfilll
	
	\tiny\hfill{Peterson \textit{A Field Guide to the Birds East of the Rockies.}}
\end{frame}
}

\begin{frame}[b,plain]{Compare proportions.}
	\begin{center}
		\includegraphics[height=0.8\textheight]{size_shape2}
	\end{center}
	\tiny Downy and Hairy Woodpecker heads from \href{http://allaboutbirds.com}{All About Birds, Cornell Lab of Ornithology}.
\end{frame}


{
\usebackgroundtemplate{\includegraphics[width=\paperwidth]{color_pattern3}}
\begin{frame}[b,plain]{\hfill\textcolor{white}{Learn color and pattern.}}
	\tiny\textcolor{white}{Prairie Warbler by Kenneth Cole Schneider, Flickr Creative Commons.}
\end{frame}
}

{
\usebackgroundtemplate{\includegraphics[width=\paperwidth]{color_pattern4}}
\begin{frame}[b,plain]{\hfill\textcolor{white}{Compare color and pattern.}}
	\tiny\textcolor{white}{Magnolia Warbler by Bill Majoros, Flickr Creative Commons.}
\end{frame}
}

{
\usebackgroundtemplate{\includegraphics[width=\paperwidth]{behavior1}}
\begin{frame}[b,plain]{Learn behavior.}
	\tiny\hfill\textcolor{white}{Brown Creeper by David Mitchell, Flickr Creative Commons.}
\end{frame}
}

{
\usebackgroundtemplate{\includegraphics[width=\paperwidth]{behavior2}}
\begin{frame}[b,plain]{\hfill\textcolor{white}{Compare behavior.}}
	\tiny\textcolor{white}{White-Breasted Nuthatch by Matt MacGillivray, Flickr Creative Commons.}
\end{frame}
}

{
%\usebackgroundtemplate{\includegraphics[width=\paperwidth]{flight_pattern}}
\begin{frame}[c,plain]{Learn flight behaviors.}
	\includegraphics[width=\textwidth]{flight_pattern}
\vskip0pt plus 1filll	
	\hfill\tiny Bird flight patterns from \href{http://allaboutbirds.com}{All About Birds, Cornell Lab of Ornithology}.
\end{frame}
}

{
\usebackgroundtemplate{\includegraphics[width=\paperwidth]{range1_junco}}
\begin{frame}[b,plain]{Learn habitat, range, and seasons.}
	\tiny\hfill Dark-Eyed Junco by NatureServe, Flickr Creative Commons.
\end{frame}
}

\begin{frame}[b,plain]
	\begin{center}
		\includegraphics[height=0.9\textheight]{range2_junco_map}
	\end{center}
\end{frame}

%{
%\usebackgroundtemplate{\includegraphics[width=\paperwidth]{know_your_penguins}}
%\begin{frame}[b,plain]
%\end{frame}
%}
%{
%\usebackgroundtemplate{\includegraphics[width=\paperwidth]{binoc_basics}}
%\begin{frame}[b,plain]
%	\tiny Birdwatcher by Daniele Zanni, Flickr Creative Commons.
%\end{frame}
%}
%
%% Binocs provided
%{
%\begin{frame}[t,plain]{Binoculars are provided for you.}
%
%
%\begin{multicols}{2}
%Nikon Action Extreme ATP 8$\times$40.
%
%\smallskip
%
%“8” is the magnification and the “40” is the diameter (in mm) of the objective.
%
%\smallskip
%
%Cost is about \$150.
%
%\smallskip
%
%This is about as cheap as you should go for first set of binoculars. Anything cheaper will be frustrating.
%
%\smallskip
%
%More info on course website.
%
%\columnbreak
%
%\includegraphics[width=0.45\textwidth]{nikon_binocs}
%
%\end{multicols}
%%\begin{columns}
%%\begin{column}[t]{0.4\textwidth}
%%
%%\end{column}
%%
%%\begin{column}{0.55\textwidth}
%%\includegraphics[width=\columnwidth]{nikon_binocs}
%%\end{column}
%%\end{columns}
%
%\end{frame}
%}
%
%%binoc parts
%{
%\begin{frame}[t,plain]{Know your binoculars.}
%
%\includegraphics[width=\textwidth]{binoc_parts}
%
%\end{frame}
%}
%
%
%{
%\usebackgroundtemplate{\includegraphics[width=\paperwidth]{bird_for_binocs}}
%\begin{frame}
%
%\vfilll
%
%\tinyfill \href{https://www.flickr.com/photos/36436392@N05/3373117530}{Photo by Phonton, \ccbync{2.0}}
%\end{frame}
%}

\end{document}
