%!TEX TS-program = lualatex
%!TEX encoding = UTF-8 Unicode

%\documentclass[t]{beamer}

%%%% HANDOUTS For online Uncomment the following four lines for handout
%\documentclass[t,handout]{beamer}  %Use this for handouts.
%\includeonlylecture{student}
%\usepackage{handoutWithNotes}
%\pgfpagesuselayout{3 on 1 with notes}[letterpaper,border shrink=5mm]
%	\setbeamercolor{background canvas}{bg=black!5}


%%% Including only some slides for students.
%%% Uncomment the following line. For the slides,
%%% use the labels shown below the command.

%% For students, use \lecture{student}{student}
%% For mine, use \lecture{instructor}{instructor}


%\usepackage{pgf,pgfpages}
%\pgfpagesuselayout{4 on 1}[letterpaper,border shrink=5mm]

% FONTS
\usepackage{fontspec}
\def\mainfont{Linux Biolinum O}
\setmainfont[Ligatures=TeX, Contextuals={NoAlternate}, BoldFont={* Bold}, ItalicFont={* Italic}, Numbers={Proportional}]{\mainfont}
%\setmonofont[Scale=MatchLowercase]{Inconsolata} 
\setsansfont[Scale=MatchLowercase]{Linux Biolinum O} 
\usepackage{microtype}

\usepackage{graphicx}
	\graphicspath{%
	{/Users/goby/Pictures/teach/466/lectures/}}%
%	{/Users/goby/Pictures/teach/common/}} % set of paths to search for images

\usepackage{amsmath,amssymb}

%\usepackage{units}

\usepackage{booktabs}
\usepackage{multicol}
%	\setlength{\columnsep=1em}

\usepackage{textcomp}
\usepackage{setspace}
\usepackage{tikz}
	\tikzstyle{every picture}+=[remember picture,overlay]

\mode<presentation>
{
  \usetheme{Lecture}
  \setbeamercovered{invisible}
  \setbeamertemplate{items}[square]
}

\usepackage{calc}
\usepackage{hyperref}

\newcommand\HiddenWord[1]{%
	\alt<handout>{\rule{\widthof{#1}}{\fboxrule}}{#1}%
}



\begin{document}
%\lecture{instructor}{instructor}

\lecture{student}{student}
{
\usebackgroundtemplate{\includegraphics[width=\paperwidth]{what_is_a_bird_carolina_wren}}
\begin{frame}[b,plain]
	\tiny\hspace{4em}\textcolor{white}{Carolina Wren by Ken Slade, Flickr Creative Commons.}
\end{frame}
}

%{
%\usebackgroundtemplate{\includegraphics[width=\paperwidth]{intro_how_many_species}}
%\begin{frame}[b,plain]{How many species of birds\dots}
%	\hfill\tiny\textcolor{white}{Limpkin and young by Mark Vance, Flickr Creative Commons.}
%\end{frame}
%}

{
\usebackgroundtemplate{\includegraphics[width=\paperwidth]{intro_how_many_species}}
\begin{frame}[t,plain]
	\begin{tikzpicture}
		\node at (12.4,-9.3) [left] {\tiny\textcolor{white}{Brown Thrasher \copyright Dustin Siegel, All Rights Reserved. Used with permission.}};
	\end{tikzpicture}
	\vspace{7em}
	
	\pause
	\hangpara\Large\hspace{14em}\textcolor{white}{23 orders}

	\hangpara\Large\hspace{14em}\textcolor{white}{142 families}
	
	\hangpara\Large\hspace{14em}\textcolor{white}{2,057 genera}
	
	\hangpara\Large\hspace{14em}\textcolor{white}{\textasciitilde10,000 species}

%\vskip0pt plus 1filll
	
%	\hfill\tiny\textcolor{white}{Brown Thrasher \copyright Dustin Siegel, All Rights Reserved. Used with permission.}

\end{frame}
}


{
\usebackgroundtemplate{\includegraphics[width=\paperwidth]{intro_form_function}}
\begin{frame}[b,plain]{The \highlight{form} of the bird is tied to it’s ecological \highlight{function.}}
	\hfill\tiny Scissor-tailed Flycatcher by Ken Slade, Flickr Creative Commons.
\end{frame}
}



%Wings
{
\begin{frame}[c,plain]{High-speed soaring birds have long, narrow wings.}
	\centering
	\includegraphics[width=\textwidth]{wings_albatross}\par
\end{frame}
}

{
\usebackgroundtemplate{\includegraphics[width=\paperwidth]{wings_albatross_flight}}
\begin{frame}[b,plain]
	\tiny Laysan Albatross by Dick Daniels, Wikimedia Commons.
\end{frame}
}

{
\begin{frame}[c,plain]{Low-speed soaring birds have broad, slotted wings.}
	\centering
	\includegraphics[width=0.9\textwidth]{wings_buteo}\par
\end{frame}
}

{
\usebackgroundtemplate{\includegraphics[width=\paperwidth]{wings_vulture_flight}}
\begin{frame}[b,plain]
	\tiny\textcolor{white}{Turkey Vulture by Tatiana Gettleman, Flickr Creative Commons.}
\end{frame}
}


{
\begin{frame}[c,plain]{Fast birds in open areas have shorter, pointed wings.}
	\centering
	\includegraphics[width=0.9\textwidth]{wings_falcon}\par
\end{frame}
}

{
\usebackgroundtemplate{\includegraphics[width=\paperwidth]{wings_peregrine_flight}}
\begin{frame}[b,plain]
	\tiny\hspace{10em}\textcolor{white}{Juvenile Peregrine Falcon (?) by Magnus Manske, Wikimedia Commons.}
\end{frame}
}

{
\usebackgroundtemplate{\includegraphics[width=\paperwidth]{wings_mourning_dove}}
\begin{frame}[b,plain]
	\tiny\textcolor{white}{Mourning Dove \copyright Stephen Ramirez, All Rights Reserved. Used with permission.}
	\vspace{6ex}
\end{frame}
}


{
\begin{frame}[c,plain]{Many birds have short, rounded wings.}
	\centering
	\includegraphics[width=0.5\textwidth]{wings_songbird}\par
\end{frame}
}

{
\usebackgroundtemplate{\includegraphics[width=\paperwidth]{wings_warbler_flight}}
\begin{frame}[b,plain]
	\tiny\hspace{8em}\textcolor{white}{Yellow-rumped Warbler by Jack Sutton, Flickr Creative Commons.}
\end{frame}
}

%% Feet
%\lecture{student}{student}
{
\begin{frame}[c,plain]{What is this type of foot used for?}
	\centering
	\includegraphics[height=0.7\textheight]{foot_perching}\par
\end{frame}
}

\lecture{instructor}{instructor}
{
\usebackgroundtemplate{\includegraphics[width=\paperwidth]{foot_perching_bananaquits}}
\begin{frame}[b,plain]
	\hfill\tiny\textcolor{white}{Bananaquits by Leon-bojarzcuk, Wikimedia Commons.}
\end{frame}
}

\lecture{student}{student}
{
\begin{frame}[c,plain]{What type of habitat does this bird use?}
	\centering
	\includegraphics[height=0.7\textheight]{foot_webbed}\par
\end{frame}
}

\lecture{instructor}{instructor}
{
\usebackgroundtemplate{\includegraphics[width=\paperwidth]{foot_ringnecked_duck}}
\begin{frame}[b,plain]
	\tiny\textcolor{white}{Ring-necked Duck by Rick Leche, Flickr Creative Commons.}
\end{frame}
}

\lecture{student}{student}
{
\begin{frame}[c,plain]{What type of habitat does this bird use?}
	\centering
	\includegraphics[height=0.7\textheight]{foot_heron}\par
\end{frame}
}

\lecture{instructor}{instructor}
{
\usebackgroundtemplate{\includegraphics[width=\paperwidth]{foot_wading_birds}}
\begin{frame}[b,plain]
	\tiny\textcolor{black}{Wading birds by Anita Gould, Flickr Creative Commons.}
\end{frame}
}

\lecture{instructor}{instructor}
{
\usebackgroundtemplate{\includegraphics[width=\paperwidth]{foot_african_jacana}}
\begin{frame}[b,plain]
	\tiny\textcolor{white}{African Jacana by Arno Meintjes, Flickr Creative Commons.\hfill\href{http://www.youtube.com/watch?v=yVMOZhpVK2g}{Video}}
\end{frame}
}

\lecture{student}{student}
{
\begin{frame}[c,plain]{What type of habitat does this bird use?}
	\centering
	\includegraphics[height=0.7\textheight]{foot_ptarmigan}\par
\end{frame}
}

{
\usebackgroundtemplate{\includegraphics[width=\paperwidth]{foot_ptarmigan_photo}}
\begin{frame}[b,plain]
	\hfill\tiny\textcolor{white}{Arnstein Rønning, Wikimedia Commons.}
\end{frame}
}

\lecture{instructor}{instructor}
{
\usebackgroundtemplate{\includegraphics[width=\paperwidth]{foot_whitetail_ptarmigan}}
\begin{frame}[b,plain]
	\tiny\textcolor{black}{White-tailed Ptarmigan by USFWS, Flickr Creative Commons.}
\end{frame}
}

\lecture{student}{student}
{
\begin{frame}[c,plain]{What type of bill does this bird have?}
	\centering
	\includegraphics[height=0.7\textheight]{foot_talons}\par
\end{frame}
}

\lecture{instructor}{instructor}
{
\usebackgroundtemplate{\includegraphics[width=\paperwidth]{foot_osprey_with_fish}}
\begin{frame}[b,plain]
	\tiny\hfill\textcolor{white}{Osprey by Matt Shiffler Photography, Flickr Creative Commons.}
\end{frame}
}

\lecture{student}{student}
{
\begin{frame}[t,plain]{What differences do you see among these shorebirds?}
	\centering
	\includegraphics[width=\textwidth]{feeding_shorebird_silhouettes}\par
\end{frame}
}

{
\begin{frame}[t,plain]{Shorebirds feed on different types of prey.}
	\begin{center}
	\includegraphics[width=\textwidth]{feeding_shorebird_feeding}\end{center}
\vskip0pt plus 1filll

\hfill\tiny\href{https://www.youtube.com/watch?v=56eU3KLIKZo}{Video}
\end{frame}
}


%%% Take a break
%\lecture{instructor}{instructor}
%{
%\usebackgroundtemplate{\includegraphics[width=\paperwidth]{break_parrot}}
%\begin{frame}[b,plain]
%	\hfill\tiny\textcolor{white}{Rosy-faced Lovebird by Cassidy McDonald, Flickr Creative Commons.}
%\end{frame}
%}

%% Begin Evolutionary History
\lecture{student}{student}
{
\usebackgroundtemplate{\includegraphics[width=\paperwidth]{anchiornis_dinosaur}}
\begin{frame}[b,plain]
	\tiny\textit{Anchiornis huxleyi}
\end{frame}
}

\lecture{instructor}{instructor}
{
\usebackgroundtemplate{\includegraphics[width=\paperwidth]{test1}}
\begin{frame}[b,plain]
\end{frame}
}

{
\usebackgroundtemplate{\includegraphics[width=\paperwidth]{test2}}
\begin{frame}[b,plain]
\end{frame}
}

\lecture{student}{student}
{
\usebackgroundtemplate{\includegraphics[width=\paperwidth]{thomas_huxley}}
\begin{frame}[b,plain]
	\hfill\tiny\textcolor{white}{Wikimedia Commons (public domain)}
\end{frame}
}

{
\usebackgroundtemplate{\includegraphics[width=\paperwidth]{compare_skulls}}
\begin{frame}[t,plain]{Birds and reptiles have a \highlight{single occipital condyle.}}
\end{frame}
}

{
\usebackgroundtemplate{\includegraphics[width=\paperwidth]{compare_ears}}
\begin{frame}[t,plain]{Birds and reptiles have a \highlight{single middle ear bone.}}
\end{frame}
}
{
\usebackgroundtemplate{\includegraphics[width=\paperwidth]{compare_jaws}}
\begin{frame}[t,plain]{Birds and reptiles have \highlight{similar jaw structures.}}
\end{frame}
}

%\lecture{instructor}{instructor}
%\begin{frame}[t,plain]{\highlight{Study} pages 26–27 for other shared characters.}
%	\begin{center}
%		\includegraphics[height=0.7\textheight]{text_cover}
%	\end{center}
%\end{frame}

\lecture{student}{student}
{
\usebackgroundtemplate{\includegraphics[width=\paperwidth]{archaeopteryx_lithographica}}
\begin{frame}[t,plain]
\end{frame}
}

{
\usebackgroundtemplate{\includegraphics[width=\paperwidth]{compare_archaeopteryx}}
\begin{frame}[t,plain]
\end{frame}
}

{
\usebackgroundtemplate{\includegraphics[width=\paperwidth]{sinosauropteryx_fossil}}
\begin{frame}[t,plain]{\textcolor{white}{\textit{Sinosauropteryx}, a theropod dinosaur, had filamentous feathers.}}
\end{frame}
}

{
\usebackgroundtemplate{\includegraphics[width=\paperwidth]{dilong}}
\begin{frame}[t,plain]
\end{frame}
}

{
\usebackgroundtemplate{\includegraphics[width=\paperwidth]{yutyrannus_feathers}}
\begin{frame}[b,plain]
\hfill\tiny\textcolor{white}{Xu et al. 2012. Nature 484: 92.}
\end{frame}
}
{
\usebackgroundtemplate{\includegraphics[width=\paperwidth]{yutyrannus_drawing}}
\begin{frame}[t,plain]{\textcolor{white}{A group of \textit{Yutyrannus}, with other theropods and pterosaurs.} }
\end{frame}
}

\lecture{instructor}{instructor}
{
\usebackgroundtemplate{\includegraphics[width=\paperwidth]{yutyrannus_phylogeny}}
\begin{frame}[b,plain]
\hfill\tiny Xu et al. 2012. Nature 484: 92.
\end{frame}
}

\lecture{student}{student}
{
\usebackgroundtemplate{\includegraphics[width=\paperwidth]{microraptor_fossil}}
\begin{frame}[b,plain]{\textcolor{white}{\textit{Microraptor gui} had flight-like feathers on wings \emph{and} legs but still had long, bony tail.}}
\hfill\tiny\textcolor{white}{\textit{Microraptor gui}, Wikimedia Commons}
\end{frame}
}

{
\usebackgroundtemplate{\includegraphics[width=\paperwidth]{microraptor_flight}}
\begin{frame}[b,plain]
\tiny\hfill\textit{Microraptor gui} drawing \copyright Portia Sloan.
\end{frame}
}

{
\usebackgroundtemplate{\includegraphics[width=\paperwidth]{confuciusornis_fossil}}
\begin{frame}[b,plain]{\textcolor{white}{\textit{Confuciusornis} had short bird-like tail.}}
\hfill\tiny\textcolor{white}{\textit{Confuciusornis}, Wikimedia Commons}
\end{frame}
}

{
\usebackgroundtemplate{\includegraphics[width=\paperwidth]{confuciusornis_drawing}}
\begin{frame}[b,plain]
%\tiny\hfill\textit{Microraptor gui} drawing \copyright Portia Sloan.
\end{frame}
}

{
\begin{frame}[b,plain]
\begin{center}
	\includegraphics[width=\textwidth]{dinosaur_soft_tissue}
\end{center}
\vfill
\tiny\hfill Schweitzer et al. 2007. Science 316: 277.
\end{frame}
}

{
\begin{frame}[b,plain]
	\hspace{-1em}\includegraphics[height=0.9\textheight]{dinosaur_genetics}
\vskip0pt plus 1fill

\tiny\hfill Schweitzer et al. 2009. Science 324: 626.
\end{frame}
}

{
\usebackgroundtemplate{\includegraphics[width=\paperwidth]{dinosaur_phylogeny}}
\begin{frame}[b,plain]
\tiny Futuyma 2005. \textit{Evolution}, Sinauer.
\end{frame}
}

{
\usebackgroundtemplate{\includegraphics[width=\paperwidth]{theropod_phylogeny}}
\begin{frame}[b,plain]{\highlight{Plumulaceous} feathers evolved in basal theropods. \highlight{Pennaceous} feathers evolved later in maniraptorans.}

\begin{tikzpicture}
	\node at (12.4,0.1) [left] {\tiny Zimmer 2011. \textit{The Tangled Bank}, Roberts and Co.};
%	\pause\node at (-0.1,0.17) [right] {\small See also Fig. 2-10 of your text.};
	\node at (3.5,1.2)[left] {\normalsize Avialae};
	\node at (2.55,4.25)[left]{\normalsize Maniraptora};
%	\node at (12.4,7.6)[gray,left]{See also Fig. 2–10 of your text.};
\end{tikzpicture}
\end{frame}
}

{
\usebackgroundtemplate{\includegraphics[width=\paperwidth]{theropod_phylogeny2}}
\begin{frame}[b,plain]
%\tiny Futuyma 2005. \textit{Evolution}, Sinauer Assoc.
\end{frame}
}



{
\usebackgroundtemplate{\includegraphics[width=\paperwidth]{body_size_theropod}}
\begin{frame}[b,plain]{Avialae were much smaller than other Saurischians.}
	\tiny Benson et al. 2014. PLoS Biology 12(5): e1001853
\end{frame}
}

{
\usebackgroundtemplate{\includegraphics[width=\paperwidth]{body_size_mesozoic}}
\begin{frame}[b,plain]{Body size decrease occurred in the early Cretaceous}
	\tiny Benson et al. 2014. PLoS Biology 12(5): e1001853
\end{frame}
}

{
\usebackgroundtemplate{\includegraphics[width=\paperwidth]{phylo_size_reduction}}
\begin{frame}[b,plain]
\begin{tikzpicture}
	\node at (12.4,0.1) [left] {\tiny Lee et al. 2014. Science 345: 562.};
	\node at (6.9,2.5)[right]{\small Maniraptora};
	\draw [very thick,->] (6.9,2.5) -- (4.75,3.5);
	\node at (11.1,8.75){\scriptsize Avialae};
\end{tikzpicture}
\end{frame}
}

{
\usebackgroundtemplate{\includegraphics[width=\paperwidth]{phylo_skeletal_rate}}
\begin{frame}[b,plain]
\begin{tikzpicture}
	\node at (12.4,0.1) [left] {\tiny Lee et al. 2014. Science 345: 562.};
	\node at (5.9,3.25)[right]{\small Maniraptora};
	\draw [very thick,->] (5.9,3.25) -- (4.6,3.9);
\end{tikzpicture}
\end{frame}
}

{
\usebackgroundtemplate{\includegraphics[width=\paperwidth]{digit_phylogeny_blank}}
\begin{frame}[t,plain]

	\vspace{6em}
	
	\hangpara Now it's your turn to identify \\ some evolutionary trends.

	\vspace{3em}
	
	\pause\hangpara What trends did you identify?
	
	\vskip0pt plus 1filll
	
	\tiny\hfill palaeos.com
\end{frame}
}

\lecture{instructor}{instructor}
{
\begin{frame}[t,plain]
	\begin{center}
		\includegraphics[height = 0.9\textheight]{digit_phylogeny}
	\end{center}
	
	\vskip0pt plus 1filll
	
	\tiny\hfill palaeos.com
\end{frame}
}

%\lecture{instructor}{instructor}
%\begin{frame}[t,plain]{Take a break.}
%	\begin{center}
%		\includegraphics[height=0.85\textheight]{xkcd}
%	\end{center}
%\end{frame}
%

%\lecture{student}{student}
%{
%\usebackgroundtemplate{\includegraphics[width=\paperwidth]{phylo_modern_birds}}
%\begin{frame}[b,plain]
%\begin{tikzpicture}
%	\node at (12.4,0.1) [left] {\tiny Jarvis et al. 2014. Science 346: 1320.};
%	\pause\draw [very thick,color=blue6] (-0.2,0.7) rectangle (7,6.5);
%%	\draw [very thick,color=green7] (-0.2,0.7) rectangle (1.75,3.5);
%\end{tikzpicture}
%
%\end{frame}
%}

%\usebackgroundtemplate{\includegraphics[width=\paperwidth]{phylogeny_radiation}}
\begin{frame}[t,plain]{Birds underwent an \highlight{adaptive radiation} about 20 Ma.}

\vspace{-\baselineskip}

\centering
\includegraphics[width=0.9\linewidth]{phylogeny_radiation}\par


\vfilll

\tiny \hfill Oliveros~et~al.~2019. PNAS~116:7916.
\end{frame}
%}

%\usebackgroundtemplate{\includegraphics[width=\paperwidth]{phylogeny_radiation}}
\begin{frame}[t,plain]{Biogeographic realms are based on distribution of modern birds.}

\vspace{-\baselineskip}

%\centering
\includegraphics[width=\linewidth]{biogeographic_realms}\par


\vfilll

\tiny \hfill Fig.~2.16, \emph{Handbook of Bird Biology.}
\end{frame}
%}

%{
%\usebackgroundtemplate{\includegraphics[width=\paperwidth]{phylo_modern_birds_major_groups}}
%\begin{frame}[t,plain]{You must learn these major groups of Neornithes.}
%\end{frame}
%}

\end{document}

