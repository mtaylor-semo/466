%!TEX TS-program = lualatex
%!TEX encoding = UTF-8 Unicode

\documentclass[t]{beamer}

%%%% HANDOUTS For online Uncomment the following four lines for handout
%\documentclass[t,handout]{beamer}  %Use this for handouts.
%\usepackage{handoutWithNotes}
%\pgfpagesuselayout{3 on 1 with notes}[letterpaper,border shrink=5mm]
%	\setbeamercolor{background canvas}{bg=black!5}


%%% Including only some slides for students.
%%% Uncomment the following line. For the slides,
%%% use the labels shown below the command.
%\includeonlylecture{student}

%% For students, use \lecture{student}{student}
%% For mine, use \lecture{instructor}{instructor}


%\usepackage{pgf,pgfpages}
%\pgfpagesuselayout{4 on 1}[letterpaper,border shrink=5mm]

% FONTS
\usepackage{fontspec}
\def\mainfont{Linux Biolinum O}
\setmainfont[Ligatures=TeX, Contextuals={NoAlternate}, BoldFont={* Bold}, ItalicFont={* Italic}, Numbers={Proportional}]{\mainfont}
%\setmonofont[Scale=MatchLowercase]{Inconsolata} 
\setsansfont[Scale=MatchLowercase]{Linux Biolinum O} 
\usepackage{microtype}

\usepackage{graphicx}
	\graphicspath{%
	{/Users/goby/Pictures/teach/466/lectures/}%
	{img/}%
	{/Users/goby/Pictures/teach/common/}} % set of paths to search for images

\usepackage{amsmath,amssymb}

%\usepackage{units}

\usepackage{booktabs}
\usepackage{multicol}
%	\setlength{\columnsep=1em}

\usepackage{textcomp}
\usepackage{setspace}
\usepackage{tikz}
	\tikzstyle{every picture}+=[remember picture,overlay]

\mode<presentation>
{
  \usetheme{Lecture}
  \setbeamercovered{invisible}
  \setbeamertemplate{items}[square]
}

\usepackage{calc}
\usepackage{hyperref}

\newcommand\HiddenWord[1]{%
	\alt<handout>{\rule{\widthof{#1}}{\fboxrule}}{#1}%
}

\newcommand\imagecredit[1]{%
	\vskip0pt plus 1filll \tiny #1}%
	

\begin{document}


\lecture{student}{student}

{\setbeamercolor{background canvas}{bg=black}
\begin{frame}[t,plain]{\textcolor{white}{Bird migration is a complex interaction of many processes.}}
	\centering
		\includegraphics[width=1.0\textwidth]{migrating_barnacle_geese}

	\imagecredit{\hfill\textcolor{white}{Migrating Barnacle Geese by Thermos, Wikimedia Commons.}}
\end{frame}
}


{
\begin{frame}[t,plain]{Migration maximizes association with primary productivity.}
	\centering
		\includegraphics[width=1.0\textwidth]{net_primary_productivity}

	\imagecredit{\href{https://www.youtube.com/watch?v=SulCktsf_PU}{Link to Video}\hfill Foley et al. 1996. Global Biogeochem. Cycles 10: 603;  Kucharik et al. 2000. Global Biogeochem.Cycles 14: 795.}
\end{frame}
}

{
\usebackgroundtemplate{\includegraphics[width=\paperwidth]{sooty_shearwater}}
\begin{frame}[t,plain]{\textcolor{black}{The Sooty Shearwater migrates round-trip more than 64,000 km.}}
	\imagecredit{\hfill\textcolor{white}{J.J. Harrison, Wikimedia Commons}}
\end{frame}
}

{
\usebackgroundtemplate{\includegraphics[width=\paperwidth]{migration_navigation}}
\begin{frame}[t,plain]{Migratory paths of 19 shearwaters.}
	\imagecredit{\hfill\textcolor{white}{Shaffer et al. 2006. Proc. Natl. Acad. Sci. USA 103: 12779.}}
\end{frame}
}

{
\usebackgroundtemplate{\includegraphics[width=\paperwidth]{migration_shearwater_npp}}
\begin{frame}[t,plain]{\textcolor{black}{Shearwater migration corresponds to high productivity.}}
	\imagecredit{Shaffer et al. 2006. Proc. Natl. Acad. Sci. USA 103: 12779.}
\end{frame}
}

%{
%\usebackgroundtemplate{\includegraphics[width=\paperwidth]{arctic_tern_nesting}}
%\begin{frame}[t,plain]{\textcolor{white}{The Arctic Tern also migrates long distances.}}
%	\imagecredit{\hfill\textcolor{white}{ Carsten Egevang, \url{arctictern.info}}}
%\end{frame}
%}

{
\usebackgroundtemplate{\includegraphics[width=\paperwidth]{arctic_tern_transmitter}}
\begin{frame}[t,plain]{\textcolor{white}{The Arctic Tern also migrates long distances.}}
	\imagecredit{\hfill\textcolor{white}{ Carsten Egevang, \url{arctictern.info}}}
\end{frame}
}

{
\usebackgroundtemplate{\includegraphics[width=\paperwidth]{arctic_tern_two_routes}}
\begin{frame}[t,plain]{\textcolor{black}{Terns used one spring route and two fall routes.}}
	\imagecredit{\hfill\textcolor{black}{ Map courtesy \url{arctictern.info}}}
\end{frame}
}

{
\usebackgroundtemplate{\includegraphics[width=\paperwidth]{arctic_tern_consensus_routes}}
\begin{frame}[t,plain]{\textcolor{white}{Can you explain why Arctic Terns follow these routes?}}
	\imagecredit{\hfill\textcolor{white}{ Map courtesy \url{arctictern.info}}}
\end{frame}
}

{
\usebackgroundtemplate{\includegraphics[width=\paperwidth]{migration_high_low_circulation}}
\begin{frame}[t,plain]{Migrants take advantage of wind rotation around high and low pressure systems.}
	\imagecredit{\hfill Vic DiVenere, \url{myweb.cwpost.liu.edu/vdivener/ers_1/chap_6.htm}}
\end{frame}
}


{
\begin{frame}[t,plain]{Terns follow wind rotation around high pressure centers.}
	\centering
		\includegraphics[width=\textwidth]{high_pressure_centers}
\end{frame}
}


{
\begin{frame}[t,plain]{North American birds use three flyways.}
	\centering
		\includegraphics[height=0.85\textheight]{flyways_north_america}
		
	\imagecredit{\hfill \href{http://borealbirds.org/boreal-bird-migrations}{BorealBirds.org}}
\end{frame}
}


{
\begin{frame}[t,plain]{Many eastern flyway birds follow a loop route.}
	\begin{center}
		\includegraphics[width=0.8\textwidth]{flyway_east}
	\end{center}
	\hangpara{Red: autumn. Blue: spring.}
		
	\imagecredit{\hfill La Sorte et al. 2014. J. Biogeogr. 41: 1685.}
\end{frame}
}

{
\begin{frame}[t,plain]{The central flyway is the reverse of the eastern route.}
	\begin{center}
		\includegraphics[width=0.8\textwidth]{flyway_central}
	\end{center}
	\hangpara{Red: autumn. Blue: spring.}
		
	\imagecredit{\hfill La Sorte et al. 2014. J. Biogeogr. 41: 1685.}
\end{frame}
}


{
\begin{frame}[t,plain]{Prevailing wind direction may explain loop routes.}
	\centering
		\includegraphics[height=0.75\textheight]{prevailing_winds2}
\end{frame}
}

{
\begin{frame}[t,plain]{Eastern and central loop routes may be tied to prevailing winds.}
	\vspace{-\baselineskip}
	\centering
		\includegraphics[height=0.8\textheight]{looped_migration_fall_route}

	\imagecredit{\hfill Figure 10–6 of text.}
\end{frame}
}

{
\begin{frame}[t,plain]{Birds migrate at higher altitudes in the west.}

\centering
\includegraphics[width=\textwidth]{migration_altitude}
		
	\imagecredit{\hfill Living Bird August 2021}
\end{frame}
}

{
\usebackgroundtemplate{\includegraphics[width=\paperwidth]{arctic_tern_consensus_routes}}
\begin{frame}[t,plain]{\textcolor{white}{Why do some Arctic Terns follow the African coast in the fall?}}
	\imagecredit{\hfill\textcolor{white}{ Map courtesy \url{arctictern.info}}}
\end{frame}
}


{
\begin{frame}[t,plain]{Shearwaters only feed where productivity is high.}
	\vspace{0.5cm}
	\centering
		\includegraphics[width=\textwidth]{migration_shearwater2}

	\imagecredit{Shaffer et al. 2006. Proc. Natl. Acad. Sci. USA 103: 12779.\hfill}
\end{frame}
}


{
\begin{frame}[t,plain]{Land migrations may also follow primary productivity.}
	\begin{center}
		\includegraphics[width=0.8\textwidth]{flyway_west}
	\end{center}

	\hangpara{Red: autumn. Blue: spring.}

	\imagecredit{\hfill La Sorte et al. 2014. J. Biogeogr. 41: 1685.}
\end{frame}
}

{
\usebackgroundtemplate{\includegraphics[width=\paperwidth]{flyway_npp}}
\begin{frame}[t,plain]{West and east loops are associated with NPP.}

	\imagecredit{\hfill La Sorte et al. 2014. Proc. Royal Soc. B. 281: 1793.}
\end{frame}
}

{
\usebackgroundtemplate{\includegraphics[width=\paperwidth]{migration_nocturnal}}
\begin{frame}[t,plain]{\textcolor{white}{Most migratory birds fly at night.}}

	\imagecredit{\textcolor{white}{www.audobon.org}}
\end{frame}
}

{
\usebackgroundtemplate{\includegraphics[width=\paperwidth]{migration_radar}}
\begin{frame}[t,plain]

	\imagecredit{\textcolor{white}{NEXRAD radar mosaic, 3 October 2010. \url{www.birdcast.info}\hfill\href{https://www.youtube.com/watch?v=uPff1t4pXiI}{Link to Video}}}
\end{frame}
}


{
\usebackgroundtemplate{\includegraphics[width=\paperwidth]{migration_weather_map}}
\begin{frame}[t,plain]{Fall migration follows cold fronts with north winds.}
	\imagecredit{\hfill\url{ebird.org}}
\end{frame}
}


{
\usebackgroundtemplate{\includegraphics[width=\paperwidth]{migration_april_front}}
\begin{frame}[t,plain]{\textcolor{white}{Spring migration follows southern winds from Gulf of Mexico.}}
	\imagecredit{\hfill\textcolor{white}{\href{https://people.mbi.ohio-state.edu/hurtado.10/US_Composite_Radar/}{Link to Birding by Radar}}}
\end{frame}
}

{
\usebackgroundtemplate{\includegraphics[width=\paperwidth]{migration_wind_direction}}
\begin{frame}[t,plain]{}
	\imagecredit{\href{http://earth.nullschool.net}{Link to Animated Global Wind}\hfill\href{http://hint.fm/wind/}{Link to Animated U.S. Wind}}
\end{frame}
}

{
\usebackgroundtemplate{\includegraphics[width=\paperwidth]{migration_april_front_overlay}}
\begin{frame}[t,plain]{\textcolor{white}{10 April 2015, 3:11 am.}}
\end{frame}
}

{
\usebackgroundtemplate{\includegraphics[width=\paperwidth]{migration_april11}}
\begin{frame}[t,plain]{\textcolor{white}{11 April 2015, 10:21 pm}}

\end{frame}
}

%Triggers of migration

{
\usebackgroundtemplate{\includegraphics[width=\paperwidth]{blackcap}}
\begin{frame}[t,plain]{Some birds adjust timing to natural climate variation.}
	\imagecredit{Eurasian Blackcap\hfill \href{https://www.flickr.com/photos/pc_plod/7145668109/}{Tony Smith, Flickr Creative Commons}}
\end{frame}
}


{
\usebackgroundtemplate{\includegraphics[width=\paperwidth]{north_atlantic_oscillation}}
\begin{frame}[t,plain]{Positive NAO correlates with earlier arrival times.}

	\imagecredit{\hfill Rachel Gregory}
\end{frame}
}



{
\usebackgroundtemplate{\includegraphics[width=\paperwidth]{pied_flycatcher}}
\begin{frame}[t,plain]{Pied flycatchers lay sooner after arrival at breeding sites.}
	\imagecredit{\textcolor{black}{Lars Falkdalen Lindahl, Flickr Creative Commons}}
\end{frame}
}

{
\begin{frame}[t,plain]{Earlier laying correlates with warmer spring temperatures.}
	\includegraphics[width=\textwidth]{pied_flycatcher_laying}

	\imagecredit{\hfill Both and Visser 2001. Nature 411: 296.}
\end{frame}
}

{
\begin{frame}[t,plain]{Earlier laying corresponds to stronger population declines.}
	\includegraphics[width=\textwidth]{pied_flycatcher_food}

	\imagecredit{\hfill Both et al. 2006. Nature 441: 81.}
\end{frame}
}

{
\usebackgroundtemplate{\includegraphics[width=\paperwidth]{migration_navigation}}
\begin{frame}[t,plain]{\textcolor{black}{How do birds navigate during migration?}}
	\imagecredit{\hfill\textcolor{white}{Shaffer et al. 2006. Proc. Natl. Acad. Sci. USA 103: 12779}}
\end{frame}
}


{
\usebackgroundtemplate{\includegraphics[width=\paperwidth]{navigation_emlen_cones}}
\begin{frame}[t,plain]{\textcolor{white}{A series of clever experiments revealed navigation cues.}}
	\imagecredit{\hfill\textcolor{white}{www.audubon.org}}
\end{frame}
}

{
\begin{frame}[t,plain]{The North Star is an important part of orientation.}
	\includegraphics[width=\textwidth]{navigation_northstar_orientation}

	\imagecredit{\hfill Emlen 1975. Scientific American.}
\end{frame}
}

{
\begin{frame}[t,plain]{Rotation around North Star determines orientation.}
	\centering
		\includegraphics[height=0.85\textheight]{navigation_northstar_rotation}

	\imagecredit{\hfill Emlen 1975. Scientific American.}
\end{frame}
}

{
\begin{frame}[t,plain]{Birds also use geomagnetism to orient.}
	\centering
		\includegraphics[width=\textwidth]{navigation_magnetism}

	\imagecredit{\hfill Wiltschko and Wiltschko 2005. J. Comp. Physiol A 191: 675.}
\end{frame}
}

{
\begin{frame}[t,plain]{Wavelength is an important part of detecting magnetism.}
	\centering
		\includegraphics[width=0.95\textwidth]{navigation_color}

	\imagecredit{\hfill Wiltschko and Wiltschko 2005. J. Comp. Physiol A 191: 675.}
\end{frame}
}

{
\begin{frame}[t,plain]{Radical electrons in retinal cryptochromes detect magnetism.}
	\begin{center}
		\includegraphics[width=1\textwidth]{navigation_cryptochrome}
	\end{center}
	
	\imagecredit{\hfill Maeda et al. 2008. Nature 453: 387}
\end{frame}
}

{
\begin{frame}[t,plain]{Retinal cryptochromes enable birds to see magnetism.}
	\begin{center}
		\includegraphics[width=0.9\textwidth]{navigation_magnetism_view}
	\end{center}
	
	\imagecredit{\hfill Solov’yov et al. 2010. Biophysical J. 99: 40.}
\end{frame}
}



\end{document}
