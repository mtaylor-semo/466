%!TEX TS-program = lualatex
%!TEX encoding = UTF-8 Unicode

\documentclass[t]{beamer}

%%%% HANDOUTS For online Uncomment the following four lines for handout
%\documentclass[t,handout]{beamer}  %Use this for handouts.
%\includeonlylecture{student}
%\usepackage{handoutWithNotes}
%\pgfpagesuselayout{3 on 1 with notes}[letterpaper,border shrink=5mm]


%%% Including only some slides for students.
%%% Uncomment the following line. For the slides,
%%% use the labels shown below the command.
%\includeonlylecture{student}

%% For students, use \lecture{student}{student}
%% For mine, use \lecture{instructor}{instructor}


%\usepackage{pgf,pgfpages}
%\pgfpagesuselayout{4 on 1}[letterpaper,border shrink=5mm]

% FONTS
\usepackage{fontspec}
\def\mainfont{Linux Biolinum O}
\setmainfont[Ligatures=TeX, Contextuals={NoAlternate}, BoldFont={* Bold}, ItalicFont={* Italic}, Numbers={Proportional}]{\mainfont}
\setmonofont[Scale=MatchLowercase]{Linux Libertine Mono O} 
\setsansfont[Scale=MatchLowercase]{Linux Biolinum O} 
\usepackage{microtype}

\usepackage{graphicx}
	\graphicspath{%
	{/Users/goby/Pictures/teach/466/lectures/}%
	{img/}}%
%	{/Users/goby/Pictures/teach/common/}} % set of paths to search for images

%\usepackage{amsmath,amssymb}

%\usepackage{units}

%\usepackage{booktabs}
\usepackage{multicol}
%	\setlength{\columnsep=1em}

\usepackage{array}
\newcolumntype{L}[1]{>{\raggedright\let\newline\\\arraybackslash\hspace{0pt}}p{#1}}
\newcolumntype{C}[1]{>{\centering\let\newline\\\arraybackslash\hspace{0pt}}p{#1}}
\newcolumntype{R}[1]{>{\raggedleft\let\newline\\\arraybackslash\hspace{0pt}}p{#1}}

\usepackage{tikz}
	\tikzstyle{every picture}+=[remember picture,overlay]


\mode<presentation>
{
  \usetheme{Lecture}
  \setbeamercovered{invisible}
  \setbeamertemplate{items}[square]
}

%\usepackage{calc}
\usepackage{hyperref}
\usepackage{color}

% shortstack needed to highlight across \\ line break.
\newcommand\sshighlight[1]{%
	\highlight{\shortstack[l]{#1}}%
}

\newcommand{\backoneline}{\vspace{-\baselineskip}}

\begin{document}



{
	\usebackgroundtemplate{\includegraphics[width=\paperwidth]{taxonomy_passerea}}
	{	\tikzstyle{every picture}+=[remember picture,overlay]
		\definecolor{orange5}{HTML}{F16913}
		\begin{frame}[b, plain]
			
			\begin{tikzpicture}
				
				
				%\draw[ultra thick, orange5] (0.9,7.3) rectangle (4.8,7.7);
				
				%\draw [<-, orange5, ultra thick] (4.9,7.5) -- (6.5, 7.5) node[minimum width=2cm, align=left, right] {Piciformes};
				
				\draw[ultra thick, orange5] (0.9,8.7) rectangle (4.8,9.3);
				
				\draw [<-, orange5, ultra thick] (4.9,9.0) -- (6.5, 9.0) node[right] {Passeriformes};
				
			\end{tikzpicture}
			\tiny\hfill Jarvis et al. 2014. Science 346: 1320.
		\end{frame}
}}



\begin{frame}{Passeriformes: \highlight{Paridae} — chickadees and titmice.}

	\includegraphics[width=0.49\linewidth]{taxonomy_cach}\hfill
	\includegraphics[width=0.49\linewidth]{taxonomy_tuti}

	\backoneline
	
	\begin{multicols}{2}
		Carolina Chickadee
		
		\medskip
		
		Replaced by Black-capped Chickadee to north and west of our region.
		
		\columnbreak
		
		Tufted Titmouse
	\end{multicols}	
	
	\vfilll
	
	\tiny
	
	\href{https://commons.wikimedia.org/w/index.php?curid=31464294}{Carolina Chickadee, Dick Daniels, \ccbysa{3.0}} \hfill 
	\href{https://www.flickr.com/photos/43322816@N08/6588366549}{Tufted Titmouse, USFWS, Public Domain}

\end{frame}

%

{
	\usebackgroundtemplate{\includegraphics[width=\paperwidth]{taxonomy_swallows}}
	
	\begin{frame}{Passeriformes: \highlight{Hirudinidae} — swallows.}
		
		\vfilll
		
		\tiny \href{https://chicagobirdalliance.org/blog/2023/6/22/identifying-adult-swallows-in-flight}{\textcopyright\,Woody Goss, Chicago Bird Alliance}
		
		
	\end{frame}
}


%

\begin{frame}{Barn and Cliff swallows have dark blue backs\dots}
	\includegraphics[width=0.49\linewidth]{taxonomy_bars_above}\hfill
	\includegraphics[width=0.49\linewidth]{taxonomy_clsw_above}
	
	\backoneline
	
	\begin{multicols}{2}
		Barn Swallow
		
		\medskip
		
		Watch for deeply forked tail.
		
		\medskip
		
		\columnbreak
		
		Cliff Swallow
		
		\medskip
		
		Watch for light rump and forehead.
	\end{multicols}	
	
	\vfilll
	
	\tiny
	
	\href{https://macaulaylibrary.org/asset/240409051}{Barn Swallow, Ryan Sanderson, ML240409051} \hfill 
	\href{https://macaulaylibrary.org/asset/184586801}{Cliff Swallow, Bryan Calk, ML184586801}
	
	
\end{frame}


%

\begin{frame}{\dots with contrasting light breast and belly.}
	\includegraphics[width=0.49\linewidth]{taxonomy_bars_below}\hfill
	\includegraphics[width=0.49\linewidth]{taxonomy_clsw_below}

	\backoneline
	
	\begin{multicols}{2}
		Barn Swallow
		
		\medskip
		
		Watch for deeply forked tail and orangish throat and rusty underparts.
		
		\columnbreak
		
		Cliff Swallow
		
		\medskip
		
		Watch for square tail, orangish throat, light underparts.
		
		\vfill\
		
	\end{multicols}	
	
	\vfilll
	
	\tiny
	
	\href{https://macaulaylibrary.org/asset/257081181}{Barn Swallow, Ad Konings, ML257081181} \hfill 
	\href{https://macaulaylibrary.org/asset/225211571}{Cliff Swallow, Matt Davis, ML225211571}


\end{frame}

%

\begin{frame}{Purple Martin is our largest swallow.}
	\includegraphics[width=0.49\linewidth]{taxonomy_puma_female}\hfill
	\includegraphics[width=0.49\linewidth]{taxonomy_puma_male}
	
	\backoneline
	
	\begin{multicols}{2}
		Female
		
		\medskip
		
		Light underparts but larger than other swallows with light underparts.
		
		\columnbreak
		
		Male
		
		\medskip
		
		Only all dark swallow. Watch for large size and slightly forked tail.
	\end{multicols}	
	
	Purple Martin tend to glide much more often than other swallows.
	
	\vfilll
	
	\tiny
	
	\href{https://macaulaylibrary.org/asset/145026141}{Purple Martin, female, Tom Johnson, ML145026141} \hfill 
	\href{https://macaulaylibrary.org/asset/223134471}{Purple Martin, male, Ryan Sanderson, ML223134471}
		
\end{frame}

%

\begin{frame}{Northern Rough-winged and Tree tend to be near water.}
	\includegraphics[width=0.49\linewidth]{taxonomy_nrws}\hfill
	\includegraphics[width=0.49\linewidth]{taxonomy_tres}
	
	\backoneline
	
	\begin{multicols}{2}
		Northern Rough-winged Swallow
		
		\medskip
		
		Watch for dingy breast and throat, brown back.
		
		\columnbreak
		
		Tree Swallow
		
		\medskip
		
		Watch for high contrast between under and upper parts. White throat.
	\end{multicols}	
	
	
	\vfilll
	
	\tiny
	
	\href{https://macaulaylibrary.org/asset/223134471}{Northern Rough-winged Swallow, ML223134471} \hfill 
	\href{https://macaulaylibrary.org/asset/219176661}{Tree Swallow, Brad Imhof, ML219176661}
	
	
\end{frame}

%

\begin{frame}{Passeriformes: \highlight{Sittidae} – nuthatches}
	\includegraphics[width=\linewidth]{taxonomy_wbnu}
	
	White-breasted Nuthatch often forage with head pointed downward. Watch for rufous lower belly.
	
	\tinyfill \href{https://www.flickr.com/photos/43322816@N08/6869438631}{White-breasted Nuthatch: USFWS, Public Domain}
\end{frame}

%

\begin{frame}{Passeriformes: \highlight{Troglodytidae} — wrens.}
	\includegraphics[width=0.49\linewidth]{taxonomy_howr}\hfill
	\includegraphics[width=0.49\linewidth]{taxonomy_carw}
	
	\backoneline
	
	\begin{multicols}{2}
		House Wren
		
		\medskip
		
		Watch for drab brown, slight eye ring, no eyebrow.
		
		\smallskip
		
		More common in town than Carolina Wren.
		
		\columnbreak
		
		Carolina Wren
		
		\medskip
		
		Watch for rufous back, buffy belly, bold eyebrow.
		
		\smallskip
		
		More common in woods than House Wren.
	\end{multicols}	
	
	
	\vfilll
	
	\tiny
	
	\href{https://flickr.com/photos/dfaulder/14333805721}{House Wren, dfaulder, \ccby{2.0}} \hfill 
	\href{https://flickr.com/photos/snpphotos/45449957822}{Carolina Wren, Shenandoah National Park, Public Domain}
	
	
\end{frame}

%

\begin{frame}{Passeriformes: \highlight{Sturnidae} — starlings.}
	\includegraphics[width=0.49\linewidth]{taxonomy_eust_flight}\hfill
	\includegraphics[width=0.49\linewidth]{taxonomy_eust}
	
	\backoneline
	
	\begin{multicols}{2}
		European Starling
		
		\medskip
		
		Watch for pointed wings and short tail. Sometimes mixed with blackbirds.
		
		
		\columnbreak
		
		European Starling
		
		\medskip
		
		Watch for glossy green-black with white spots, bright yellow bill.
		
	\end{multicols}	
	
	
	\vfilll
	
	\tiny
	
	\href{https://www.flickr.com/photos/mr_t_in_dc/6304096569}{European Starling, Dr.TinMD, \ccbync{2.0}} \hfill 
	\href{https://flickr.com/photos/treegrow/33244397614}{European Starling, Katja Schulz, \ccby{2.0}}
	
	
\end{frame}

%

\begin{frame}{Passeriformes: \highlight{Mimidae} — mockingbird and thrashers.}
	\includegraphics[width=0.49\linewidth]{taxonomy_nomo}\hfill
	\includegraphics[width=0.49\linewidth]{taxonomy_brth}
	
	\backoneline
	
	\begin{multicols}{2}
		Northern Mockingbird
		
		\medskip
		
		Robin-sized light gray bird. Watch for long tail with white outer tail feathers, white wing patches.
		
		
		\columnbreak
		
		Brown Thrasher
		
		\medskip
		
		Robin-sized rufous-brown bird. Watch for long tail, streaked breast and flanks, slightly decurved bill.
		
	\end{multicols}	
	
	
	\vfilll
	
	\tiny
	
	\href{https://www.flickr.com/photos/130819719@N05/39505827385}{Northern Mockingbird, Becky Matsubara, \ccbync{2.0}} \hfill 
	\href{https://flickr.com/photos/rick_al/41869016892}{Brown Thrasher, Rick from Georgia, \ccby{2.0}}
	
	
\end{frame}

%

\begin{frame}{Passeriformes: \highlight{Turdidae} — thrushes.}
	\includegraphics[width=0.49\linewidth]{taxonomy_amro}\hfill
	\includegraphics[width=0.49\linewidth]{taxonomy_woth}
	
	\backoneline
	
	\begin{multicols}{2}
		American Robin
		
		\medskip
		
		Slate-gray back, orange underparts, broken white eye ring. Watch for white corners of tail in flight.
		
		\smallskip
		
		Open habitat, often park-like settings.
		\columnbreak
		
		Wood Thrush
		
		\medskip
		
		Robin-sized brown bird with white underparts. Breast, flanks, and belly heavily spotted. White eye ring.
		
		\smallskip
		
		Understory of deciduous woods.
		
	\end{multicols}	
	
	\vfilll
	
	\tiny
	
	\href{https://flickr.com/photos/52450054@N04/49910016428}{American Robin, Judy Gallagher, \ccby{2.0}} \hfill 
	\href{https://flickr.com/photos/pazzani/40741868431}{Wood Thrush, Mike's Birds, \ccbysa{2.0}}
	
	
\end{frame}

\begin{frame}{Eastern Bluebird is the Missouri state bird.}
	\includegraphics[width=\linewidth]{taxonomy_eabl}
	
	\tinyfill \href{https://flickr.com/photos/bhnuthatch611/49658803853}{Eastern Bluebird, 611catbirds too, \ccby{2.0}}
	
\end{frame}



%

\begin{frame}{Passeriformes: \highlight{Passeridae} — Old World sparrows.}
	\includegraphics[width=0.49\linewidth]{taxonomy_hosp_male}\hfill
	\includegraphics[width=0.49\linewidth]{taxonomy_hosp_female}
	
	\backoneline
	
	\begin{multicols}{2}
		male
		
		\medskip
		
		Watch for black chin that extends to breast during breeding. Gray crown, rufous nape.
		\columnbreak
		
		female
				
		\medskip
		
		Watch for drab head with light tan eyebrow. Stout yellowish bill. \phantom{Hidden words to fill space.}
		
		
	\end{multicols}	
	
	House Sparrows are exotic species found urban and rural areas with buildings.
	
	
	\vfilll
	
	\tiny
	
	\href{https://flickr.com/photos/hedera_baltica/32691733707}{House Sparrow, hedera.baltica, \ccbysa{2.0}} \hfill 
	\href{https://flickr.com/photos/hedera_baltica/52118904483}{House Sparrow, hedera.baltica, \ccbysa{2.0}}
	
	
\end{frame}

\lecture{instructor}{instructor}

\begin{frame}{Pull out your index card and field guide.}
	
	\hangpara You may use your notes and field guide to answer the identification question on each slide.
	
	\hangpara Write all information for one slide on \textit{one} line of your card.
	
	\hangpara One slide, one line.
	
	\hangpara You may not work together or share information.
	
	
\end{frame}

{
	\usebackgroundtemplate{\includegraphics[width=\paperwidth]{taxonomy6_bcti}}
	\begin{frame}[t]{\textcolor{white}{Pg.\,293: Name the order, family, and identify the species.}}
		
		\tinyfill  \textcolor{white}{\href{https://macaulaylibrary.org/asset/289341271}{Bryan Calk, ML289341271}}
	\end{frame}
}

{
	\usebackgroundtemplate{\includegraphics[width=\paperwidth]{taxonomy6_rbnu}}
	\begin{frame}[t]{\textcolor{white}{Pg.\,293: Name the order, family, and identify the species.}}
		
		\tinyfill  \textcolor{white}{\href{https://macaulaylibrary.org/asset/219648251}{Simon Boivin, ML219648251}}
	\end{frame}
}

{
	\usebackgroundtemplate{\includegraphics[width=\paperwidth]{taxonomy6_grca}}
	\begin{frame}[t]{\textcolor{white}{Pg.\,322: Name the order, family, and identify the species.}}
		
		\tinyfill  \textcolor{white}{\href{https://macaulaylibrary.org/asset/51392481}{Evan Lipton, ML51392481}}
	\end{frame}
}

{
	\usebackgroundtemplate{\includegraphics[width=\paperwidth]{taxonomy6_vath}}
	\begin{frame}[t]{\textcolor{white}{Pg.\,322: Name the order, family, and identify the species.}}
		
		\tinyfill  \textcolor{white}{\href{https://macaulaylibrary.org/asset/28703721}{Graham Gerdeman, ML28703721}}
	\end{frame}
}


\end{document}
