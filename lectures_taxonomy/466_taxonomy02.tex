%!TEX TS-program = lualatex
%!TEX encoding = UTF-8 Unicode

\documentclass[t]{beamer}

%%%% HANDOUTS For online Uncomment the following four lines for handout
%\documentclass[t,handout]{beamer}  %Use this for handouts.
%\includeonlylecture{student}
%\usepackage{handoutWithNotes}
%\pgfpagesuselayout{3 on 1 with notes}[letterpaper,border shrink=5mm]


%%% Including only some slides for students.
%%% Uncomment the following line. For the slides,
%%% use the labels shown below the command.
%\includeonlylecture{student}

%% For students, use \lecture{student}{student}
%% For mine, use \lecture{instructor}{instructor}


%\usepackage{pgf,pgfpages}
%\pgfpagesuselayout{4 on 1}[letterpaper,border shrink=5mm]

% FONTS
\usepackage{fontspec}
\def\mainfont{Linux Biolinum O}
\setmainfont[Ligatures=TeX, Contextuals={NoAlternate}, BoldFont={* Bold}, ItalicFont={* Italic}, Numbers={Proportional}]{\mainfont}
\setmonofont[Scale=MatchLowercase]{Linux Libertine Mono O} 
\setsansfont[Scale=MatchLowercase]{Linux Biolinum O} 
\usepackage{microtype}

\usepackage{graphicx}
	\graphicspath{%
	{/Users/goby/Pictures/teach/466/lectures/}%
	{img/}}%
%	{/Users/goby/Pictures/teach/common/}} % set of paths to search for images

%\usepackage{amsmath,amssymb}

%\usepackage{units}

%\usepackage{booktabs}
\usepackage{multicol}
%	\setlength{\columnsep=1em}

%\usepackage{textcomp}
%\usepackage{setspace}
\usepackage{tikz}
%	\tikzstyle{every picture}+=[remember picture,overlay]

\usepackage{forest}
\usetikzlibrary{trees}
\tikzstyle{block} = [rectangle, draw, fill=white, rounded corners, minimum size=2em]
\tikzstyle{branch} = [thick, draw]

%\usetikzlibrary{positioning, backgrounds}


\forestset{
	every leaf node/.style={
		if n children=0{#1}{}
	},
	every tree node/.style={
		if n children=0{}{#1}
	},
	mytree/.style={
		for tree={
			font=\footnotesize\selectfont,
			edge path={
				\noexpand\path [draw, thick, \forestoption{edge}] (!u.parent anchor) |- (.child anchor)\forestoption{edge label};
			},
			every tree node={draw=none,inner sep=0, outer sep=0, minimum size=0},
			%every leaf node/.style={align=left},
			grow'=0,
			parent anchor=east, 
			child anchor=west,
			anchor=west,
			l = 2cm,
			%l sep=1.5cm,
			s sep=0mm,
			draw=none,
			if n children=0{tier=word}{}
		}
	}
}

\mode<presentation>
{
  \usetheme{Lecture}
  \setbeamercovered{invisible}
  \setbeamertemplate{items}[square]
}

%\usepackage{calc}
\usepackage{hyperref}

% shortstack needed to highlight across \\ line break.
\newcommand\sshighlight[1]{%
	\highlight{\shortstack[l]{#1}}%
}

% shortstack needed to hold across \\ line break.
\newcommand\ssbf[1]{%
	\textbf{\shortstack[l]{#1}}%
}

\newcommand{\backoneline}{\vspace{-\baselineskip}}

\begin{document}


\lecture{student}{student}

\begin{frame}{Mirandornithes and Columbimorphae each have one family in Missouri.}

\backoneline

\begin{forest} mytree
[[, l sep=+1.7cm, edge label = {node [xshift=-0.8cm, text width = 1.5cm] {\footnotesize Subclass Neornithes}}
	[,name=neognathae, edge label = {node [text width=2cm, midway, xshift=1.1cm] {\footnotesize Infraclass Neognathae}}
		[, name=neoaves, edge label = {node [text width=2cm, midway, xshift=1.1cm] {\footnotesize Superorder Neoaves}}
			[,l-=1cm
				[Clade\\ Passerea, align=left]
				[\ssbf{Clade\\ Columbimorphae}, align=left]
			]
			[\ssbf{Clade\\ Mirandornithes}, align=left]
		]
		[, name=galloanseres, edge label = {node [text width=2cm, midway, xshift=1.1cm] {\footnotesize Superorder Galloanseres}}
			[Order\\ Galliformes, align=left]
			[Order\\ Anseriformes, align=left]
		]
	]
	[,name=paleognathae, edge label = {node [text width=1.75cm, midway, xshift=1cm] {\footnotesize Infraclass Paleognathae}}
		[Tinamous\\ and “Ratites”, align=left]
		%[Grade\\ “Ratites”, align=left]
	]
]]
\end{forest}

\end{frame}

\begin{frame}[t]{Mirandornithes has two orders, Phoenicopteriformes (flamingos) and \highlight{Podicipediformes} (grebes).}
\centering
\includegraphics[height=0.78\textheight]{taxonomy_mirandornithes}

\vfilll

\tinyfill \textcopyright\,\href{https://www.deviantart.com/gredinia/art/Bird-cladistic-Mirandornithes-diversity-698780401}{Gredinia, Deviant Art}
\end{frame}


{
\usebackgroundtemplate{\includegraphics[width=\paperwidth]{pied_billed_grebe}}
\begin{frame}[b,plain]{\textcolor{white}{All Podicipediformes are in the family Podicipedidae (grebes).}}
	\tiny \textcolor{white}{Pied-Billed Grebe by Tom Talbott, Flickr Creative Commons.}
\end{frame}
}

{
\usebackgroundtemplate{\includegraphics[width=\paperwidth]{taxonomy_podicipedidae}}
\begin{frame}[b]


	\tinyfill \textcolor{white}{Pied-Billed Grebe by  \href{https://macaulaylibrary.org/asset/218667151}{Brad Imhoff, Macaulay Library \textsc{ml218667151}}}
\end{frame}
}


\begin{frame}[t]{Columbimorphae contains the doves and pigeons.}
\centering
\includegraphics[height=0.82\textheight]{taxonomy_columbimorphae}

\vfilll

\tinyfill \textcopyright\,\href{https://www.deviantart.com/gredinia/art/Bird-cladistic-Columbimorphae-diversity-705097984}{Gredinia, Deviant Art}
\end{frame}


%{


{
\usebackgroundtemplate{\includegraphics[width=\paperwidth]{mourning_dove}}
\begin{frame}[b,plain]{\textcolor{white}{Columbiformes has one family, \textcolor{orange4}{Columbidae.}}}
	\tiny\hfill Mourning Dove by Jeff Bryant, Flickr Creative Commons.
\end{frame}
}

\begin{frame}[t]{Four species of Columbidae are found in Missouri.}
\vspace{-0.5\baselineskip}
\centering
\includegraphics[width=0.9\textwidth]{taxonomy_columbidae}

\vfilll

\tiny \href{https://macaulaylibrary.org/asset/45325741}{Eurasian Collared-Dove by Jesse Amesbury}\hfill \href{https://macaulaylibrary.org/asset/50824121}{Mourning Dove by Nancy Christensen}. \hfill \href{https://macaulaylibrary.org/asset/61674401}{Rock Pigeon by Luke Seitz.}  All from Macaulay Library.
\end{frame}

{
\usebackgroundtemplate{\includegraphics[width=\paperwidth]{coot_chicks}}
\begin{frame}[b,plain]{\hfill\textcolor{white}{Everything else belongs to Passerea.}}
	\tiny\textcolor{white}{American Coot by Mike Baird, Flickr Creative Commons.}
\end{frame}
}



\begin{frame}{Most birds you see belong to Clade Passerea.}


\begin{forest} mytree
[[, l sep=+1.7cm, edge label = {node [xshift=-0.8cm, text width = 1.5cm] {\footnotesize Subclass Neornithes}}
	[,name=neognathae, edge label = {node [text width=2cm, midway, xshift=1.1cm] {\footnotesize Infraclass Neognathae}}
		[, name=neoaves, edge label = {node [text width=2cm, midway, xshift=1.1cm] {\footnotesize Superorder Neoaves}}
			[,l-=1cm
				[\ssbf{Clade\\ Passerea}, align=left]
				[Clade\\ Columbimorphae, align=left]
			]
			[Clade\\ Mirandornithes, align=left]
		]
		[, name=galloanseres, edge label = {node [text width=2cm, midway, xshift=1.1cm] {\footnotesize Superorder Galloanseres}}
			[Order\\ Galliformes, align=left]
			[Order\\ Anseriformes, align=left]
		]
	]
	[,name=paleognathae, edge label = {node [text width=1.75cm, midway, xshift=1cm] {\footnotesize Infraclass Paleognathae}}
		[Tinamous\\ and “Ratites”, align=left]
		%[Grade\\ “Ratites”, align=left]
	]
]]
\end{forest}

\end{frame}


{
\usebackgroundtemplate{\includegraphics[width=\paperwidth]{taxonomy_passerea}}
{	\tikzstyle{every picture}+=[remember picture,overlay]
\definecolor{orange5}{HTML}{F16913}
\begin{frame}[b, plain]

\begin{tikzpicture}

\draw [white, fill=white] (5,0.5) rectangle (12,1.1);

\onslide*<2>{\draw[ultra thick, orange5] (0.8,2.1) rectangle (5.0,2.9) node (placeholder){};
\draw [->, orange5, ultra thick] (7,2.5) -- (5.1, 2.5);

\node[minimum width=5.5cm, align=left, color=orange5] at (8.5,2.5) {Caprimulgiformes};

\draw[ultra thick, orange5] (0.8,1.4) rectangle (5.0,1.7) node (placeholder){};
\draw [->, orange5, ultra thick] (7,1.55) -- (5.1, 1.55);

\node[minimum width=5.5cm, align=left, color=orange5] at (8.13,1.55) {Cuculiformes};

}


\end{tikzpicture}
	\tiny\hfill Jarvis et al. 2014. Science 346: 1320.
\end{frame}
}}


\begin{frame}{Cuculiformes: \highlight{Cuculidae} — Cuckoos and Roadrunners}

\backoneline

\begin{multicols}{2}

\includegraphics[width=\linewidth]{taxonomy_ybcu}

\columnbreak

\highlight{Zygodactly feet} (first and fourth toe reversed). 

\vspace{\baselineskip}

Bill slightly decurved.

\vspace{\baselineskip}

Yellow-billed Cuckoo is secretive, more often heard than seen. Distinctive \emph{ku-ku-ku-kddowl-kddowl\dots} heard in the woods. Also dove-like coos.

\vspace{\baselineskip}

Greater Roadrunner sometimes seen in southwestern Missouri.


\end{multicols}

\vfilll


\tiny \href{https://macaulaylibrary.org/asset/94446}{Link to audio} \hfill Yellow-billed Cuckoo: \href{https://macaulaylibrary.org/asset/102608081}{Sue Barth, ML102608081}


\end{frame}




\begin{frame}[t]{Caprimulgiformes: \highlight{Caprimulgidae} — nightjars.}
\includegraphics[width=\textwidth]{taxonomy_caprimulgidae}

\hangpara Very short, depressed, wide bills with long rictal bristles. Long, slender wings. Small, weak feet.

\hangpara Common Nighthawk active during day, Eastern Whip-poor-will and Chuck-will's-widow during night so more often heard than seen.

\vfilll

\tiny \href{https://macaulaylibrary.org/asset/64516111}{Common Nighthawk, Ronnie d'Entremont, ML64516111} \hfill \href{https://www.allaboutbirds.org/guide/Common_Nighthawk/photo-gallery/466538}{Video} \hfill \href{https://macaulaylibrary.org/asset/120753651}{Eastern Whip-poor-will, Tony Dvorak, ML120753651}
\end{frame}


\begin{frame}[t]{Caprimulgiformes: \highlight{Apodidae} — Chimney Swift}

\vspace{-\baselineskip}

\begin{multicols}{2}
\includegraphics[width=\linewidth]{taxonomy_apodidae}

\columnbreak

\highlight{Pamprodactyl feet} (four toes pointed forward), lacks rictal bristles. Otherwise, similar to Caprimulgidae except dark gray and smaller. 

\vspace{\baselineskip}

Most often seen in flight. Stiff wing beats give the \textit{illusion} of alternating flaps. Listen for high-pitched twitter calls on campus, starting in early April.
\end{multicols}

\vfilll

\tiny \href{https://macaulaylibrary.org/asset/232676411}{Chimney Swift, Caleb Putnam, ML232676411} \qquad  \href{https://www.allaboutbirds.org/guide/Chimney_Swift/photo-gallery/440546}{Video}

\end{frame}

\begin{frame}[t]{Caprimulgiformes: \highlight{Trochilidae} — Hummingbirds}
\vspace{-0.5\baselineskip}

%\centering

\includegraphics[width=\textwidth]{taxonomy_trochilidae}

\hangpara Smallest of all birds. Long, \highlight{terete bills,} usually straight or decurved. Weak feet. Most have iridescent feathers, Males usually with brilliant gorget (throat patch). Only birds capable of hovering and flying backwards.

\vfilll

\tiny \href{https://macaulaylibrary.org/asset/97947051}{Ruby-throated Hummingbird male, Brian Kulvete, ML97947051} \hfill  \href{https://macaulaylibrary.org/asset/268051161}{Ruby-throated Hummingbird female, German Garcia, ML268051161}

\end{frame}

\end{document}
