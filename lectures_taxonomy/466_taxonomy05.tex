%!TEX TS-program = lualatex
%!TEX encoding = UTF-8 Unicode

\documentclass[t]{beamer}

%%%% HANDOUTS For online Uncomment the following four lines for handout
%\documentclass[t,handout]{beamer}  %Use this for handouts.
%\includeonlylecture{student}
%\usepackage{handoutWithNotes}
%\pgfpagesuselayout{3 on 1 with notes}[letterpaper,border shrink=5mm]


%%% Including only some slides for students.
%%% Uncomment the following line. For the slides,
%%% use the labels shown below the command.
%\includeonlylecture{student}

%% For students, use \lecture{student}{student}
%% For mine, use \lecture{instructor}{instructor}


%\usepackage{pgf,pgfpages}
%\pgfpagesuselayout{4 on 1}[letterpaper,border shrink=5mm]

% FONTS
\usepackage{fontspec}
\def\mainfont{Linux Biolinum O}
\setmainfont[Ligatures=TeX, Contextuals={NoAlternate}, BoldFont={* Bold}, ItalicFont={* Italic}, Numbers={Proportional}]{\mainfont}
\setmonofont[Scale=MatchLowercase]{Linux Libertine Mono O} 
\setsansfont[Scale=MatchLowercase]{Linux Biolinum O} 
\usepackage{microtype}

\usepackage{graphicx}
	\graphicspath{%
	{/Users/goby/Pictures/teach/466/lectures/}%
	{img/}}%
%	{/Users/goby/Pictures/teach/common/}} % set of paths to search for images

%\usepackage{amsmath,amssymb}

%\usepackage{units}

%\usepackage{booktabs}
\usepackage{multicol}
%	\setlength{\columnsep=1em}

\usepackage{array}
\newcolumntype{L}[1]{>{\raggedright\let\newline\\\arraybackslash\hspace{0pt}}p{#1}}
\newcolumntype{C}[1]{>{\centering\let\newline\\\arraybackslash\hspace{0pt}}p{#1}}
\newcolumntype{R}[1]{>{\raggedleft\let\newline\\\arraybackslash\hspace{0pt}}p{#1}}

%usepackage{tikz}
%	\tikzstyle{every picture}+=[remember picture,overlay]

\usepackage{forest}
\usetikzlibrary{trees}
\tikzstyle{block} = [rectangle, draw, fill=white, rounded corners, minimum size=2em]
\tikzstyle{branch} = [thick, draw]

%\usetikzlibrary{positioning, backgrounds}


\forestset{
	every leaf node/.style={
		if n children=0{#1}{}
	},
	every tree node/.style={
		if n children=0{}{#1}
	},
	mytree/.style={
		for tree={
			font=\footnotesize\selectfont,
			edge path={
				\noexpand\path [draw, thick, \forestoption{edge}] (!u.parent anchor) |- (.child anchor)\forestoption{edge label};
			},
			every tree node={draw=none,inner sep=0, outer sep=0, minimum size=0},
			%every leaf node/.style={align=left},
			grow'=0,
			parent anchor=east, 
			child anchor=west,
			anchor=west,
			l = 1cm,
			%l sep=1.5cm,
			s sep=0mm,
			draw=none,
			if n children=0{tier=word}{}
		}
	}
}

\mode<presentation>
{
  \usetheme{Lecture}
  \setbeamercovered{invisible}
  \setbeamertemplate{items}[square]
}

%\usepackage{calc}
\usepackage{hyperref}
\usepackage{color}

% shortstack needed to highlight across \\ line break.
\newcommand\sshighlight[1]{%
	\highlight{\shortstack[l]{#1}}%
}

\newcommand{\backoneline}{\vspace{-\baselineskip}}

\begin{document}



{
\usebackgroundtemplate{\includegraphics[width=\paperwidth]{taxonomy_passerea}}
{	\tikzstyle{every picture}+=[remember picture,overlay]
\definecolor{orange5}{HTML}{F16913}
\begin{frame}[b, plain]

\begin{tikzpicture}


\draw[ultra thick, orange5] (0.9,7.3) rectangle (4.8,7.7);

\draw [<-, orange5, ultra thick] (4.9,7.5) -- (6.5, 7.5) node[minimum width=2cm, align=left, right] {Piciformes};

\draw[ultra thick, orange5] (0.9,8.7) rectangle (4.8,9.3);

\draw [<-, orange5, ultra thick] (4.9,9.0) -- (6.5, 9.0) node[right] {Passeriformes};

%\node[minimum width=2cm, align=left] at (8,5.25) {Passeriformes is divided
%into\\ Suboscine and Oscine clades.};

\end{tikzpicture}
	\tiny\hfill Jarvis et al. 2014. Science 346: 1320.
\end{frame}
}}


\begin{frame}{Piciformes: \highlight{Picidae} — woodpeckers.}

\backoneline

\begin{multicols}{2}
\includegraphics[width=0.95\linewidth]{taxonomy_picidae}

\columnbreak

\highlight{Zygodactyl feet:} Toes 1 and 4 reversed.

\medskip

Strong, chisel-like bill (most)

\medskip

Stiff, pointed rectrices.

\vspace{3\baselineskip}

Red-headed Woodpecker is only woodpecker with identical sexes.

\medskip

They are found in open areas with few trees.

\tinyfill \href{https://flickr.com/photos/usfwsmidwest/42163328934}{Red-headed Woodpecker: USFWS, Public Domain}
\end{multicols}


\end{frame}

\begin{frame}{Red-bellied Woodpeckers common in mature woods.}

\includegraphics[width=0.49\textwidth]{taxonomy_rbwo_female}\hfill
\includegraphics[width=0.49\textwidth]{taxonomy_rbwo_male}


\vfilll

\tiny \href{https://flickr.com/photos/188361162@N03/49879499767}{Female: Jean Weller, Public Domain} \hfill \href{https://flickr.com/photos/indianaivy/6957915661}{Male: Indiana Ivy Nature Photography, \ccby{2.0}}
\end{frame}



\begin{frame}{Downy and Hairy Woodpecker identified with practice.}

\vspace{-\baselineskip}

\begin{multicols}{2}
\reflectbox{\includegraphics[width=\linewidth]{taxonomy_dowo_hawo_feeder}}

\columnbreak

Downy Woodpecker is smaller, about the size of a House Sparrow.

\bigskip

Hairy Woodpecker is larger, about the size of an American Robin.

\end{multicols}



\tinyfill

\href{https://www.birdsandblooms.com/birding/bird-species/tell-difference-downy-hairy-woodpeckers/}{Female Hairy Woodpecker and female Downy Woodpecker: Marie Read, \textcopyright\,Birds and Blooms} 
\end{frame}

\begin{frame}

\includegraphics[width=\linewidth]{taxonomy_dowo_hawo_compare}

\vspace{-\baselineskip}

\begin{multicols}{2}

Downy Woodpecker

\medskip

Short bill about 1/3 of head length.

\smallskip

Small spots in outer tail feathers.

\smallskip

Red patch on males usually undivided.


\columnbreak

Hairy Woodpecker

\medskip

Longer bill about length of head.

\smallskip

No spots on outer tail feathers.

\smallskip

Red patch on males usually divided.

\end{multicols}


\vfilll

\tiny \href{https://macaulaylibrary.org/asset/47227441}{Male Downy Woodpecker: Evan Lipton, ML47227441} \hfill \href{https://macaulaylibrary.org/asset/25034271}{Male Hairy Woodpecker: Jean-Sébastien Guénette, ML25034271}

\end{frame}

\begin{frame}{Pileated Woodpecker is our largest woodpecker.}

\includegraphics[width=0.49\textwidth]{taxonomy_piwo_female}\hfill
\includegraphics[width=0.49\textwidth]{taxonomy_piwo_male}

\vfilll

\tiny \href{https://flickr.com/photos/79452129@N02/16950041431}{Female: Fyn Kynd, \ccby{2.0}} \hfill \href{https://flickr.com/photos/thebackroadphotogragher/26201612377}{Male: The Backroad Photographer, \ccby{2.0}}
\end{frame}

\begin{frame}{Northern Flicker often forages on the ground.}

\includegraphics[width=0.49\textwidth]{taxonomy_nofl_female}\hfill
\includegraphics[width=0.49\textwidth]{taxonomy_nofl_male}

\vfilll

\tiny \href{https://flickr.com/photos/95782365@N08/29975120108}{Female: Susan Young, Public Domain} \hfill \href{https://flickr.com/photos/shivashenoy/50804171637}{Male: Shiva Shenoy, \ccby{2.0}}
\end{frame}

\begin{frame}{Yellow-bellied Sapsucker is a winter visitor.}

\includegraphics[width=0.49\textwidth]{taxonomy_ybsa_male}\hfill
\includegraphics[width=0.49\textwidth]{taxonomy_ybsa_female}

\vfilll

\tiny \href{https://flickr.com/photos/16502322@N03/49792022926}{Male: fishhawk, \ccby{2.0}} \hfill \href{https://flickr.com/photos/135081788@N03/51861239439}{Female: Channel City Camera Club, \ccby{2.0}}
\end{frame}

{
\usebackgroundtemplate{\includegraphics[width=\paperwidth]{taxonomy_passerea}}
{	\tikzstyle{every picture}+=[remember picture,overlay]
\definecolor{orange5}{HTML}{F16913}
\begin{frame}[b, plain]

\begin{tikzpicture}


%\draw[ultra thick, orange5] (0.9,7.3) rectangle (4.8,7.7);

%\draw [<-, orange5, ultra thick] (4.9,7.5) -- (6.5, 7.5) node[minimum width=2cm, align=left, right] {Piciformes};

\draw[ultra thick, orange5] (0.9,8.7) rectangle (4.8,9.3);

\draw [<-, orange5, ultra thick] (4.9,9.0) -- (6.5, 9.0) node[right] {Passeriformes};

\node[minimum width=2cm, align=left] at (8.2,7.5) {From here forward,\\ 
all families belong to\\ Order Passeriformes.};

\end{tikzpicture}
	\tiny\hfill Jarvis et al. 2014. Science 346: 1320.
\end{frame}
}}


{
\usebackgroundtemplate{\includegraphics[width=\paperwidth]{taxonomy_sub_oscines}}
\begin{frame}[t]{Passeriformes has two suborders: Tyranni and Passeri.}

\begin{tikzpicture}[remember picture,overlay]

\draw [thick] (-0.15,-0.4) -- (4.03,-0.4) node[above, midway] {Tyranni};
\draw [thick] (4.1,-0.4) -- (12.3,-0.4) node[above, midway] {Passeri};

\end{tikzpicture}

%\includegraphics[width=\linewidth]{taxonomy_sub_oscines}
\tinyfill \href{https://www.bird-phylogeny.de/superorders/australaves/passeriformes/}{Avitaxonomicon}
\end{frame}
}

{
\usebackgroundtemplate{\includegraphics[width=\paperwidth]{taxonomy_sub_oscines}}
\begin{frame}[t]{Tyranni = \highlight{suboscines.} Passeri = \highlight{oscines.}}

\begin{tikzpicture}[remember picture,overlay]

\draw [thick] (-0.15,-0.4) -- (4.03,-0.4) node[above, midway] {\highlight{suboscines}};
\draw [thick] (4.1,-0.4) -- (12.3,-0.4) node[above, midway] {\highlight{oscines}};


\end{tikzpicture}
%\includegraphics[width=\linewidth]{taxonomy_sub_oscines}

\tinyfill \href{https://www.bird-phylogeny.de/superorders/australaves/passeriformes/}{Avitaxonomicon}

\end{frame}
}


{
\usebackgroundtemplate{\includegraphics[width=\paperwidth]{taxonomy_suboscines}}
\begin{frame}

\begin{multicols}{2}
\phantom{Nothing to see here.}

\columnbreak

Most suboscines are in South America.

\vspace{4\baselineskip}
The only suboscines in North America are \highlight{Tyrannidae} (Tyrant Flycatchers).

\begin{tikzpicture}[overlay,every picture]
\draw [ultra thick,<-,color=orange5] (-1.5,1.05) -- (-0.15,1.05);
\end{tikzpicture}

\end{multicols}
\end{frame}
}


\begin{frame}{Flycatchers forage from conspicuous perch to catch insects.}

\includegraphics[width=\linewidth]{taxonomy_eaph_eawp}

\vspace{-\baselineskip}

\begin{multicols}{2}

Eastern Wood Pewee

\medskip

Does not bob longer tail.

\smallskip

Pair of distinct wing bars.

\smallskip

Head only slightly darker than body.


\columnbreak

Eastern Phoebe

\medskip

\textit{Bobs shorter tail.}

\smallskip

Weak wingbars.

\smallskip

Dark head, buffy belly.

\end{multicols}


\vfilll

\tiny \href{https://macaulaylibrary.org/asset/97009621}{Eastern Wood Pewee: John Deitsch, ML97009621} \hfill \href{https://macaulaylibrary.org/asset/58963211}{Eastern Phoebe: Epi Shemming, ML58963211}

\end{frame}


\begin{frame}{Acadian Flycatcher is our summer woodland \textit{Empidonax.}}
\backoneline

\begin{multicols}{2}
\includegraphics[width=\linewidth]{taxonomy_acfl}

\columnbreak

\textsc{Caution:} \textit{Empidonax} flycatchers can be identified reliably \emph{only} by voice and habitat.

\bigskip

Acadian Flycatcher: tee-chup

\smallskip

Willow Flycatcher: fitz-bew

\smallskip

Least Flycatcher: che-bek

\smallskip

Yellow-bellied Flycatcher: che-bunk

\smallskip

Alder Flycatcher: free-beer!
\end{multicols}


\tinyfill \href{https://flickr.com/photos/wildreturn/27174213407}{Acadian Flycatcher: Andy Reago \& Chrissy McClarren, \ccby{2.0}}
\end{frame}


\begin{frame}{Great Crested Flycatcher is common in our woods.}
	\backoneline
	
	\begin{multicols}{2}
		\includegraphics[width=0.95\linewidth]{taxonomy_gcfl}
	
		\columnbreak
	
		Large flycatcher, about size of Northern Cardinal.
	
		\bigskip
	
		Watch for crested, dark head,
		gray upper breast,
		yellow lower breast and belly, and
		rufous tail.
	
	\end{multicols}
	
	\tinyfill \href{https://macaulaylibrary.org/asset/56379261}{Ryan Schain, ML56379261}
\end{frame}

\begin{frame}{Eastern Kingbird is uncommon in open fields.}
	\includegraphics[width=0.49\linewidth]{taxonomy_eaki_front} \hfill
	\includegraphics[width=0.49\linewidth]{taxonomy_eaki_back}
	
	\vfilll
	
	\tiny \href{https://flickr.com/photos/wildreturn/34996512566}{Andy Reago \& Chrissy McClarren, \ccby{2.0}} \hfill \href{https://flickr.com/photos/21531749@N06/34415322023}{Andrew Weitzel, \ccbysa{2.0}}
\end{frame}


\begin{frame}{Oscine birds have a complex “song control center” that enables song learning.}

\includegraphics[width=\linewidth]{anatomy_brain_suboscine_oscine}

\smallskip

Oscine birds are the so-called “song birds.” 

\smallskip

\textsc{Caution:} parrots and hummingbirds learn their songs.


\tinyfill Liu et al.~2013. Nature Comm.~4:2082. 10.1038/ncomms3082


\end{frame}

\begin{frame}{\highlight{Vireonidae} — vireos are small birds with short tails.}

\backoneline
	\begin{multicols}{2}
		
		\includegraphics[width=\linewidth]{taxonomy_revi}
		
		Red-eyed Vireo
		
		\medskip
		
		Forages in mid to upper canopy.
		
		\smallskip
		
		Watch for distinct eyebrow, olive-green back, no wingbars.
		
		\smallskip
	
		Sings incessantly all summer.
	
		\columnbreak
		
		\includegraphics[width=\linewidth]{taxonomy_wavi}
		
		Warbling Vireo
	
		\medskip
			
		Forages in large trees near water.
		
		\smallskip
		
		Watch for weak eyebrow,  olive-gray back, no wing bars.
		
		\smallskip
		
		\phantom{stuff}
		
	\end{multicols}
	
	\vfilll
	
	\tiny \href{https://flickr.com/photos/wildreturn/48905242882}{Red-eyed Vireo: Andy Reago \& Chrissy McClarren, \ccby{2.0}} \hfill  \href{https://flickr.com/photos/135081788@N03/51501843041}{Warbling Vireo: Channel City Camera Club, \ccby{2.0}}
		
\end{frame}

\begin{frame}{\highlight{Vireonidae} — vireos have hooked bills with notch near tip.}
	
	\backoneline
	\begin{multicols}{2}
		
		\includegraphics[width=\linewidth]{taxonomy_ytvi}
		
		Yellow-throated Vireo
		
		\medskip
		
		Forages in mid to upper canopy.
		
		\smallskip
		
		Watch for yellow throat, yellow spectacles, two wingbars.
				
		\columnbreak
		
		\includegraphics[width=\linewidth]{taxonomy_wevi}
		
		White-eyed Vireo
		
		\medskip
		
		Forages in thickets.
		
		\smallskip
		
		Watch for white throat, yellow spectacles, two wingbars.
		
	\end{multicols}
	
	\vfilll
	
	\tiny  \href{https://flickr.com/photos/mattyfioner/4996456327}{Yellow-throated Vireo: Matt Tillett, \ccby{2.0}} \hfill  \href{https://flickr.com/photos/waynenf/52315407133}{White-eyed Vireo: Wayne National Forest, Public Domain}
	
\end{frame}


\begin{frame}{\highlight{Corvidae} — crows and jays have stout, slightly decurved bills, rictal bristles.}
	
	\backoneline
	\begin{multicols}{2}
		
		\includegraphics[width=\linewidth]{taxonomy_amcr}
		
		American Crow
		
		\medskip
		
		Medium-sized all-black bird in variety of habitats.
		
		\smallskip
		
		Often seen foraging in fields.
		
		\smallskip
		
		Distinguish from Fish Crow by voice.
		
		\columnbreak
		
		\includegraphics[width=\linewidth]{taxonomy_blja}
		
		Blue Jay
		
		\medskip
		
		Robin-sized blue bird in variety of habitats.
		
		\smallskip
		
		Watch for high contrast head marking and wing and tail markings.
		
		\smallskip
		
%		Vocal. Tricksters.
		
	\end{multicols}
	
	\vfilll
	
	\tiny  \href{https://flickr.com/photos/btrentler/13551831983}{American Crow: Brandon Trentler, \ccby{2.0}} \hfill  \href{https://flickr.com/photos/indianaivy/2980015664}{Blue Jay: Indiana Ivy Nature Photography, \ccby{2.0}}
	
\end{frame}

\end{document}
