%!TEX TS-program = lualatex
%!TEX encoding = UTF-8 Unicode

\documentclass[t]{beamer}

%%%% HANDOUTS For online Uncomment the following four lines for handout
%\documentclass[t,handout]{beamer}  %Use this for handouts.
%\includeonlylecture{student}
%\usepackage{handoutWithNotes}
%\pgfpagesuselayout{3 on 1 with notes}[letterpaper,border shrink=5mm]


%%% Including only some slides for students.
%%% Uncomment the following line. For the slides,
%%% use the labels shown below the command.
%\includeonlylecture{student}

%% For students, use \lecture{student}{student}
%% For mine, use \lecture{instructor}{instructor}


%\usepackage{pgf,pgfpages}
%\pgfpagesuselayout{4 on 1}[letterpaper,border shrink=5mm]

% FONTS
\usepackage{fontspec}
\def\mainfont{Linux Biolinum O}
\setmainfont[Ligatures=TeX, Contextuals={NoAlternate}, BoldFont={* Bold}, ItalicFont={* Italic}, Numbers={Proportional}]{\mainfont}
\setmonofont[Scale=MatchLowercase]{Linux Libertine Mono O} 
\setsansfont[Scale=MatchLowercase]{Linux Biolinum O} 
\usepackage{microtype}

\usepackage{graphicx}
	\graphicspath{%
	{/Users/goby/Pictures/teach/466/lectures/}%
	{img/}}%
%	{/Users/goby/Pictures/teach/common/}} % set of paths to search for images

%\usepackage{amsmath,amssymb}

%\usepackage{units}

%\usepackage{booktabs}
\usepackage{multicol}
%	\setlength{\columnsep=1em}

\usepackage{array}
\newcolumntype{L}[1]{>{\raggedright\let\newline\\\arraybackslash\hspace{0pt}}p{#1}}
\newcolumntype{C}[1]{>{\centering\let\newline\\\arraybackslash\hspace{0pt}}p{#1}}
\newcolumntype{R}[1]{>{\raggedleft\let\newline\\\arraybackslash\hspace{0pt}}p{#1}}

%usepackage{tikz}
%	\tikzstyle{every picture}+=[remember picture,overlay]

\usepackage{forest}
\usetikzlibrary{trees}
\tikzstyle{block} = [rectangle, draw, fill=white, rounded corners, minimum size=2em]
\tikzstyle{branch} = [thick, draw]

%\usetikzlibrary{positioning, backgrounds}


\forestset{
	every leaf node/.style={
		if n children=0{#1}{}
	},
	every tree node/.style={
		if n children=0{}{#1}
	},
	mytree/.style={
		for tree={
			font=\footnotesize\selectfont,
			edge path={
				\noexpand\path [draw, thick, \forestoption{edge}] (!u.parent anchor) |- (.child anchor)\forestoption{edge label};
			},
			every tree node={draw=none,inner sep=0, outer sep=0, minimum size=0},
			%every leaf node/.style={align=left},
			grow'=0,
			parent anchor=east, 
			child anchor=west,
			anchor=west,
			l = 2cm,
			%l sep=1.5cm,
			s sep=0mm,
			draw=none,
			if n children=0{tier=word}{}
		}
	}
}

\mode<presentation>
{
  \usetheme{Lecture}
  \setbeamercovered{invisible}
  \setbeamertemplate{items}[square]
}

%\usepackage{calc}
\usepackage{hyperref}

% shortstack needed to highlight across \\ line break.
\newcommand\sshighlight[1]{%
	\highlight{\shortstack[l]{#1}}%
}

\newcommand{\backoneline}{\vspace{-\baselineskip}}

\begin{document}


\lecture{student}{student}


\begin{frame}{You must learn these major taxa of Neornithes.}


\begin{forest} mytree
[[, l sep=+1.7cm, edge label = {node [xshift=-0.8cm, text width = 1.5cm] {\footnotesize Subclass Neornithes}}
	[,name=neognathae, edge label = {node [text width=2cm, midway, xshift=1.1cm] {\footnotesize Infraclass Neognathae}}
		[, name=neoaves, edge label = {node [text width=2cm, midway, xshift=1.1cm] {\footnotesize Superorder Neoaves}}
			[,l-=1cm
				[\sshighlight{Clade\\ Passerea}, align=left]
				[Clade\\ Columbimorphae, align=left]
			]
			[Clade\\ Mirandornithes, align=left]
		]
		[, name=galloanseres, edge label = {node [text width=2cm, midway, xshift=1.1cm] {\footnotesize Superorder Galloanseres}}
			[Order\\ Galliformes, align=left]
			[Order\\ Anseriformes, align=left]
		]
	]
	[,name=paleognathae, edge label = {node [text width=1.75cm, midway, xshift=1cm] {\footnotesize Infraclass Paleognathae}}
		[Tinamous\\ and “Ratites”, align=left]
		%[Grade\\ “Ratites”, align=left]
	]
]]
\end{forest}

\end{frame}


{
\usebackgroundtemplate{\includegraphics[width=\paperwidth]{taxonomy_passerea}}
{	\tikzstyle{every picture}+=[remember picture,overlay]
\definecolor{orange5}{HTML}{F16913}
\begin{frame}[b, plain]

\begin{tikzpicture}

\draw [white, fill=white] (5,0.5) rectangle (12,1.1);

\draw[ultra thick, orange5] (0.8,3.05) rectangle (5.0,3.6) node (placeholder){};
\draw [->, orange5, ultra thick] (7,3.35) -- (5.1, 3.35);

\node[anchor=south west, minimum width=5.5cm, align=left, color=orange5] at (5.7,2.9) {Charadriiformes\\ Gruiformes};

%\node[anchor=south west, minimum width=5.5cm, align=right, color=orange5] at (5.2,2.8) {Gruiformes};


\end{tikzpicture}
	\tiny\hfill Jarvis et al. 2014. Science 346: 1320.
\end{frame}
}}





{
\usebackgroundtemplate{\includegraphics[width=\paperwidth]{taxonomy_charadriiformes}}
\begin{frame}[t]{\highlight{Charadriiformes} (shorebirds) has four families in Missouri.}

\tikzset{
    use page relative coordinates/.style={
        shift={(current page.south west)},
        x={(current page.south east)},
        y={(current page.north west)}
    },
}

\begin{tikzpicture}[remember picture, overlay, use page relative coordinates]

\node[anchor=south west, minimum width=5.5cm, align=right, color=white] at (0.0,0.44) {\tiny Black-necked Stilt by  \href{https://flickr.com/photos/95782365@N08/26203432462}{Susan Young, Flickr, Public Domain}};


\node[anchor=south west, minimum width=5.5cm, align=right, color=white] at (0.59,0.44) {\tiny Greater Yellowlegs by  \href{https://flickr.com/photos/joeweav/30133422748}{Joe Weaver, Flickr, \ccby{2.0}}};
\end{tikzpicture}

\vfilll

\tiny \textcolor{white}{American Golden-Plover by  \href{https://flickr.com/photos/usfws_alaska/51101401474}{Peter Pearsall/USFWS, Public Domain}\hfill Forster's Term by \href{https://flickr.com/photos/dfaulder/35770210905}{dfaulder, Flickr, \ccby{2.0}}}


\end{frame}
}


\begin{frame}{Charadriiformes: Charadriidae — Plovers}
\includegraphics[width=0.49\linewidth]{taxonomy_sepl}\hfill \includegraphics[width=0.49\linewidth]{taxonomy_killdeer}

Charadriidae are small shorebirds, with medium to long legs and short bills. The hallux absent or short and elevated. Toes usually semipalmate to palmate.

\vspace{\baselineskip}

Most plovers have 1–2 complete or incomplete breast bands\dots


\vfilll

\tiny Semipalmated Plover by  \href{https://flickr.com/photos/island_deborah/51161493609}{Deborah Freeman, \ccbysa{2.0}} \hfill Killdeer by \href{https://flickr.com/photos/beckymatsubara/46556538305}{Becky Matsubara, Flickr, \ccby{2.0}}
\end{frame}
%
\begin{frame}{Charadriiformes: Charadriidae — Plovers}
\includegraphics[width=0.49\linewidth]{taxonomy_agpl}\hfill \includegraphics[width=0.49\linewidth]{taxonomy_bbpl}

\dots but two species have solid black chins, necks, and breasts.

\vspace{\baselineskip}

They are sight feeders, standing still until they spot prey, then run to it. 


\vfilll

\tiny American Golden-Plover by  \href{https://flickr.com/photos/usfws_alaska/51101401474}{Peter Pearsall/USFWS, Public Domain} \hfill Black-bellied Plover by \href{https://flickr.com/photos/71119007@N03/17440547326}{Under the same moon, Flickr, \ccby{2.0}}
\end{frame}



\begin{frame}{Charadriiformes: Scolopacidae — Sandpipers}
\includegraphics[width=0.49\linewidth]{taxonomy_sosa} \hfill \includegraphics[width=0.49\linewidth]{taxonomy_grye}

Scolopacidae are small to medium-size shorbirds with long, slender bills that are straight to decurved. 

\vspace{\baselineskip}

They use the bills to probe down into the water or sediment. 


\vfilll

\tiny Solitary Sandpiper by \href{https://flickr.com/photos/95782365@N08/52243404588}{Susan Young, Flickr, Public Domain}\hfill Greater Yellowlegs by  \href{https://flickr.com/photos/joeweav/30133422748}{Joe Weaver, Flickr, \ccby{2.0}}

\end{frame}

%


\begin{frame}{Charadriiformes: Scolopacidae — Sandpipers}
\includegraphics[width=0.49\linewidth]{taxonomy_spsa} \hfill \includegraphics[width=0.49\linewidth]{taxonomy_pesa}

Hallux absent or short and elevated. Toes usually semipalmate to palmate.

\vspace{\baselineskip}

The Spotted Sandpiper (left) stops every few steps to bob its but up and down.

\vspace{\baselineskip}

“Peeps” are a group of very small sandpipers, like the Least Sandpiper, that can be tricky to identify.


\vfilll

\tiny Spotted Sandpiper by \href{https://flickr.com/photos/airboy123/26032281188}{Sunny, Flickr, \ccby{2.0}}\hfill Pectoral Sandpiper by  \href{https://flickr.com/photos/wildreturn/46104489551}{Andy Reago \& Chrissy McClarren, Flickr, \ccby{2.0}}

\end{frame}

{
\usebackgroundtemplate{\includegraphics[width=\paperwidth]{taxonomy_amwo}}
\begin{frame}[t]{American Woodcock is found in fields and woods.}

	\tinyfill \textcolor{white}{American Woodock by  \href{https://www.flickr.com/photos/79452129@N02/15728605426}{Fyn Kynd, Flickr, \ccby{2.0}}}
\end{frame}
}


\begin{frame}{Charadriiformes: Laridae — Gulls and Terns}

\includegraphics[width=0.49\linewidth]{taxonomy_rbgu}\hfill
\includegraphics[width=0.49\linewidth]{taxonomy_fote}

\begin{tabular}{L{5.7cm}L{6cm}}
\textbf{Gulls}	& \textbf{Terns} \tabularnewline
Bill slightly hooked & Bill acutely pointed \tabularnewline
Tail square to slightly rounded & Tail forked\tabularnewline
Upright posture & 	Squat, elongated posture\tabularnewline
\end{tabular}


\vfilll

\tiny Ring-billed Gull by \href{https://flickr.com/photos/marknenadov/33299083036}{Mark Nenadov, Flickr, \ccby{2.0}} \hfill
Forstner's Tern by \href{https://flickr.com/photos/dfaulder/35770210905}{dfaulder, Flickr, \ccby{2.0}}
4\end{frame}

\begin{frame}{Charadriiformes: Laridae — Gulls and Terns}

\includegraphics[width=0.49\linewidth]{taxonomy_rbgu_flying}\hfill
\includegraphics[width=0.49\linewidth]{taxonomy_cate}

Gulls are larger and less graceful than terns. Terns fly with their bills pointed toward water as they watch for prey. Tern wings are more pointed than gull wings.

\vfilll

\tiny Ring-billed Gull by \href{https://flickr.com/photos/pavdw/16805907032}{Paul VanDerWerf, Flickr, \ccby{2.0}} \hfill
Caspian Tern by \href{https://flickr.com/photos/seneynwr/15191931442}{Seney Natural History Association, Flickr, \ccbysa{2.0}}


\end{frame}



\end{document}
