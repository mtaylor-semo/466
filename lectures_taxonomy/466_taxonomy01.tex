%!TEX TS-program = lualatex
%!TEX encoding = UTF-8 Unicode

\documentclass[t]{beamer}

%%%% HANDOUTS For online Uncomment the following four lines for handout
%\documentclass[t,handout]{beamer}  %Use this for handouts.
%\includeonlylecture{student}
%\usepackage{handoutWithNotes}
%\pgfpagesuselayout{3 on 1 with notes}[letterpaper,border shrink=5mm]


%%% Including only some slides for students.
%%% Uncomment the following line. For the slides,
%%% use the labels shown below the command.
%\includeonlylecture{student}

%% For students, use \lecture{student}{student}
%% For mine, use \lecture{instructor}{instructor}


%\usepackage{pgf,pgfpages}
%\pgfpagesuselayout{4 on 1}[letterpaper,border shrink=5mm]

% FONTS
\usepackage{fontspec}
\def\mainfont{Linux Biolinum O}
\setmainfont[Ligatures=TeX, Contextuals={NoAlternate}, BoldFont={* Bold}, ItalicFont={* Italic}, Numbers={Proportional}]{\mainfont}
\setmonofont[Scale=MatchLowercase]{Linux Libertine Mono O} 
\setsansfont[Scale=MatchLowercase]{Linux Biolinum O} 
\usepackage{microtype}

\usepackage{graphicx}
	\graphicspath{%
	{/Users/goby/Pictures/teach/466/lectures/}%
	{img/}}%
%	{/Users/goby/Pictures/teach/common/}} % set of paths to search for images

%\usepackage{amsmath,amssymb}

%\usepackage{units}

%\usepackage{booktabs}
\usepackage{multicol}
%	\setlength{\columnsep=1em}

%\usepackage{textcomp}
%\usepackage{setspace}
\usepackage{tikz}
%	\tikzstyle{every picture}+=[remember picture,overlay]

\usepackage{forest}
\usetikzlibrary{trees}
\tikzstyle{block} = [rectangle, draw, fill=white, rounded corners, minimum size=2em]
\tikzstyle{branch} = [thick, draw]

%\usetikzlibrary{positioning, backgrounds}


\forestset{
	every leaf node/.style={
		if n children=0{#1}{}
	},
	every tree node/.style={
		if n children=0{}{#1}
	},
	mytree/.style={
		for tree={
			font=\footnotesize\selectfont,
			edge path={
				\noexpand\path [draw, thick, \forestoption{edge}] (!u.parent anchor) |- (.child anchor)\forestoption{edge label};
			},
			every tree node={draw=none,inner sep=0, outer sep=0, minimum size=0},
			%every leaf node/.style={align=left},
			grow'=0,
			parent anchor=east, 
			child anchor=west,
			anchor=west,
			l = 2cm,
			%l sep=1.5cm,
			s sep=0mm,
			draw=none,
			if n children=0{tier=word}{}
		}
	}
}

\mode<presentation>
{
  \usetheme{Lecture}
  \setbeamercovered{invisible}
  \setbeamertemplate{items}[square]
}

%\usepackage{calc}
\usepackage{hyperref}

% shortstack needed to highlight across \\ line break.
\newcommand\sshighlight[1]{%
	\highlight{\shortstack[l]{#1}}%
}



\newcommand{\backoneline}{\vspace{-\baselineskip}}

\begin{document}


\lecture{student}{student}
{
\begin{frame}[t,plain]{Our taxonomic goals for this course:}


\hangpara \highlight{Learn higher classification of birds to order and family level.} Classification will generally follow Sangster~et~al.~2022. Phylogenetic definitions for 25 higher-level clade names of birds. Avian Research 13: 100027.


\hangpara \highlight{Learn identifying characters of selected bird orders and families.} Family identification will focus on those with Missouri representation.

\end{frame}
}

{
\usebackgroundtemplate{\includegraphics[width=\paperwidth]{theropod_phylogeny}}
\begin{frame}[b,plain]{Recall: the Avialae clade is defined by powered flight, and includes \highlight{Neornithes,} the modern birds.}

\begin{tikzpicture}
	\node at (12.4,0.1) [left] {\tiny Zimmer 2011. \textit{The Tangled Bank}, Roberts and Co.};
	\node at (5.3,0.83)[left, fill=white] {\normalsize Avialae};
	\node at (2.55,4.25)[left]{\normalsize \textcolor{white}{Maniraptora}};
	\node at (12.01,0.8)[left, fill=white]{\footnotesize \highlight{Neornithes}};
	\node at (11.7,1.2)[left, fill=white]{\phantom{fr}};
%	\node at (12.4,7.6)[gray,left]{See also Fig. 2–10 of your text.};
\end{tikzpicture}
\end{frame}
}


\begin{frame}{We will use this phylogeny as our guide. Names to learn will be highlighted in orange.}

\backoneline

\begin{forest} mytree
[[, l sep=+1.7cm, edge label = {node [xshift=-0.8cm, text width = 1.5cm] {\footnotesize \highlight{Subclass Neornithes}}}
	[,name=neognathae, edge label = {node [text width=2cm, midway, xshift=1.1cm] {\footnotesize Infraclass Neognathae}}
		[, name=neoaves, edge label = {node [text width=2cm, midway, xshift=1.1cm] {\footnotesize Superorder Neoaves}}
			[,l-=1cm
				[Clade\\ Passerea, align=left]
				[Clade\\ Columbimorphae, align=left]
			]
			[Clade\\ Mirandornithes, align=left]
		]
		[, name=galloanseres, edge label = {node [text width=2cm, midway, xshift=1.1cm] {\footnotesize Superorder Galloanseres}}
			[Order\\ Galliformes, align=left]
			[Order\\ Anseriformes, align=left]
		]
	]
	[,name=paleognathae, edge label = {node [text width=1.75cm, midway, xshift=1cm] {\footnotesize Infraclass Paleognathae}}
		[Tinamous\\ and “Ratites”, align=left]
		%[Grade\\ “Ratites”, align=left]
	]
]]
\end{forest}

\bigskip

I will not ask you to draw the phylogeny from memory.

\end{frame}

\begin{frame}{\highlight{Palaeognathae} are basal. \highlight{Neognathae} are derived.}

\backoneline

\begin{forest} mytree
[[, l sep=+1.7cm, edge label = {node [xshift=-0.8cm, text width = 1.5cm] {\footnotesize Subclass Neornithes}}
	[,name=neognathae, edge label = {node [text width=2cm, midway, xshift=1.1cm] {\footnotesize \highlight{Infraclass Neognathae}}}
		[, name=neoaves, edge label = {node [text width=2cm, midway, xshift=1.1cm] {\footnotesize Superorder Neoaves}}
			[,l-=1cm
				[Clade\\ Passerea, align=left]
				[Clade\\ Columbimorphae, align=left]
			]
			[Clade\\ Mirandornithes, align=left]
		]
		[, name=galloanseres, edge label = {node [text width=2cm, midway, xshift=1.1cm] {\footnotesize Superorder Galloanseres}}
			[Order\\ Galliformes, align=left]
			[Order\\ Anseriformes, align=left]
		]
	]
	[,name=paleognathae, edge label = {node [text width=1.75cm, midway, xshift=1cm] {\footnotesize \highlight{Infraclass Paleognathae}}}
		[Tinamous\\ and “Ratites”, align=left]
		%[Grade\\ “Ratites”, align=left]
	]
]]
\end{forest}

\end{frame}


{
\usebackgroundtemplate{\includegraphics[width=\paperwidth]{skulls_palaeo_neognath}}
\begin{frame}[b,plain]{Palaeo- and neognaths have different palate structures.}
	\hfill\tiny Modified from Tyne and Berger 1976, \textit{Fundamentals of Ornithology}, Wiley \& Sons.
\end{frame}
}

{
\usebackgroundtemplate{\includegraphics[width=\paperwidth]{skulls_palaeo_neognath_small}}
\begin{frame}

\tinyfill  Modified from Tyne and Berger 1976, \textit{Fundamentals of Ornithology}, Wiley \& Sons.
\end{frame}
}

\begin{frame}{Tinamous and “ratites” have palaeognath palates.}

\vspace{0.5\baselineskip}

\begin{forest} mytree
[[, l sep=+1.7cm, edge label = {node [xshift=-0.8cm, text width = 1.5cm] {\footnotesize Subclass Neornithes}}
	[,name=neognathae, edge label = {node [text width=2cm, midway, xshift=1.1cm] {\footnotesize Infraclass Neognathae}}
		[, name=neoaves, edge label = {node [text width=2cm, midway, xshift=1.1cm] {\footnotesize Superorder Neoaves}}
			[,l-=1cm
				[Clade\\ Passerea, align=left]
				[Clade\\ Columbimorphae, align=left]
			]
			[Clade\\ Mirandornithes, align=left]
		]
		[, name=galloanseres, edge label = {node [text width=2cm, midway, xshift=1.1cm] {\footnotesize Superorder Galloanseres}}
			[Order\\ Galliformes, align=left]
			[Order\\ Anseriformes, align=left]
		]
	]
	[,name=paleognathae, edge label = {node [text width=1.75cm, midway, xshift=1cm] {\footnotesize Infraclass Paleognathae}}
		[\sshighlight{Tinamous\\ and “Ratites”}, align=left]
		%[Grade\\ “Ratites”, align=left]
	]
]]
\end{forest}

\bigskip

“Ratite” is Latin for “raft” in reference to the keel-less breastbone.

\end{frame}

{
\usebackgroundtemplate{\includegraphics[width=\paperwidth]{crested_tinamou}}
\begin{frame}[b,plain]{\textcolor{white}{Tinamous are capable of weak flight.}}
	\tiny\hfill\textcolor{white}{Crested Tinamou by Evanphoto, Wikimedia Commons.}
\end{frame}
}

{
\usebackgroundtemplate{\includegraphics[width=\paperwidth]{cassowary}}
\begin{frame}[b,plain]{\textcolor{white}{Cassowaries and other “ratites” cannot fly.}}
	\tiny\hfill\textcolor{white}{Double-wattled Cassowary by Brian Gratwicke, Flickr Creative Commons.}
\end{frame}
}

{
\usebackgroundtemplate{\includegraphics[width=\paperwidth]{ratite_phylogeny}}
\begin{frame}[b,plain]
	\tiny\hfill Mitchell et al. 2014, Science 344: 898.
\end{frame}
}

%\lecture{instructor}{instructor}
%
%\begin{frame}{You must learn these major taxa of Neornithes.}
%
%\begin{forest} mytree
%[[, l sep=+1.7cm, edge label = {node [xshift=-0.8cm, text width = 1.5cm] {\footnotesize Subclass Neornithes}}
%	[,name=neognathae, edge label = {node [text width=2cm, midway, xshift=1.1cm] {\footnotesize Infraclass Neognathae}}
%		[, name=neoaves, edge label = {node [text width=2cm, midway, xshift=1.1cm] {\footnotesize Superorder Neoaves}}
%			[,l-=1cm
%				[Clade\\ Passerea, align=left]
%				[Clade\\ Columbimorphae, align=left]
%			]
%			[Clade\\ Mirandornithes, align=left]
%		]
%		[, name=galloanseres, edge label = {node [text width=2cm, midway, xshift=1.1cm] {\footnotesize Superorder Galloanseres}}
%			[\sshighlight{Order\\ Galliformes}, align=left]
%			[\sshighlight{Order\\ Anseriformes}, align=left]
%		]
%	]
%	[,name=paleognathae, edge label = {node [text width=1.75cm, midway, xshift=1cm] {\footnotesize Infraclass Paleognathae}}
%		[Tinamous\\ and “Ratites”, align=left]
%		%[Grade\\ “Ratites”, align=left]
%	]
%]]
%\end{forest}
%
%\end{frame}



%\lecture{instructor}{instructor}
%
%\begin{frame}{You must learn these major taxa of Neornithes.}
%
%\begin{forest} mytree
%[[, l sep=+1.7cm, edge label = {node [xshift=-0.8cm, text width = 1.5cm] {\footnotesize Subclass Neornithes}}
%	[,name=neognathae, edge label = {node [text width=2cm, midway, xshift=1.1cm] {\footnotesize Infraclass Neognathae}}
%		[, name=neoaves, edge label = {node [text width=2cm, midway, xshift=1.1cm] {\footnotesize Superorder Neoaves}}
%			[,l-=1cm
%				[\sshighlight{Clade\\ Passerea}, align=left]
%				[\sshighlight{Clade\\ Columbimorphae}, align=left]
%			]
%			[\sshighlight{Clade\\ Mirandornithes}, align=left]
%		]
%		[, name=galloanseres, edge label = {node [text width=2cm, midway, xshift=1.1cm] {\footnotesize Superorder Galloanseres}}
%			[Order\\ Galliformes, align=left]
%			[Order\\ Anseriformes, align=left]
%		]
%	]
%	[,name=paleognathae, edge label = {node [text width=1.75cm, midway, xshift=1cm] {\footnotesize Infraclass Paleognathae}}
%		[Tinamous\\ and “Ratites”, align=left]
%		%[Grade\\ “Ratites”, align=left]
%	]
%]]
%\end{forest}
%
%\end{frame}
%


\begin{frame}{Superorder Galloanseres has two orders: \highlight{Anseriformes} and \highlight{Galliformes.}}

\begin{forest} mytree
[[, l sep=+1.7cm, edge label = {node [xshift=-0.8cm, text width = 1.5cm] {\footnotesize Subclass Neornithes}}
	[,name=neognathae, edge label = {node [text width=2cm, midway, xshift=1.1cm] {\footnotesize Infraclass Neognathae}}
		[, name=neoaves, edge label = {node [text width=2cm, midway, xshift=1.1cm] {\footnotesize Superorder Neoaves}}
			[,l-=1cm
				[Clade\\ Passerea, align=left]
				[Clade\\ Columbimorphae, align=left]
			]
			[Clade\\ Mirandornithes, align=left]
		]
		[, name=galloanseres, edge label = {node [text width=2cm, midway, xshift=1.1cm] {\footnotesize Superorder Galloanseres}}
			[\sshighlight{Order\\ Galliformes}, align=left]
			[\sshighlight{Order\\ Anseriformes}, align=left]
		]
	]
	[,name=paleognathae, edge label = {node [text width=1.75cm, midway, xshift=1cm] {\footnotesize Infraclass Paleognathae}}
		[Tinamous\\ and “Ratites”, align=left]
		%[Grade\\ “Ratites”, align=left]
	]
]]
\end{forest}

\end{frame}

%\begin{frame}{Superorder Galloanseres has two orders: \highlight{Anseriformes} and \highlight{Galliformes.}}
%
%
%\begin{forest} mytree
%[, l sep=+1.7cm, edge label = {node [xshift=-0.8cm, text width = 1.5cm] {\footnotesize Subclass Neornithes}}
%	[,name=neognathae, edge label = {node [text width=2cm, midway, xshift=1.1cm] {\footnotesize Infraclass Neognathae}}
%		[, name=neoaves, edge label = {node [text width=2cm, midway, xshift=1.1cm] {\footnotesize Superorder Neoaves}}
%			[,l-=1cm
%				[Clade\\ Passerea, align=left]
%				[Clade\\ Columbimorphae, align=left]
%			]
%			[Clade\\ Mirandornithes, align=left]
%		]
%		[, name=galloanseres, edge label = {node [text width=2cm, midway, xshift=1.1cm] {\footnotesize Superorder Galloanseres}}
%			[\sshighlight{Order\\ Galliformes}, align=left]
%			[\sshighlight{Order\\ Anseriformes}, align=left]
%		]
%	]
%]
%\end{forest}
%
%\end{frame}


{
\usebackgroundtemplate{\includegraphics[width=\paperwidth]{galloanseres_phylogeny}}
\begin{frame}[b,plain]
	\tiny Kriegs et al. 2007. BMC Evolutionary Biology 7: 190.
\end{frame}
}
%
%
%
%\begin{frame}[t]{Mirandornithes contains the flamingos and grebes.}
%\centering
%\includegraphics[height=0.82\textheight]{taxonomy_mirandornithes}
%
%\vfilll
%
%\tinyfill \textcopyright\,\href{https://www.deviantart.com/gredinia/art/Bird-cladistic-Mirandornithes-diversity-698780401}{Gredinia, Deviant Art}
%\end{frame}
%
%\begin{frame}[t]{Columbimorphae contains the doves.}
%\centering
%\includegraphics[height=0.82\textheight]{taxonomy_columbimorphae}
%
%\vfilll
%
%\tinyfill \textcopyright\,\href{https://www.deviantart.com/gredinia/art/Bird-cladistic-Columbimorphae-diversity-705097984}{Gredinia, Deviant Art}
%\end{frame}
%
%{
%\usebackgroundtemplate{\includegraphics[width=\paperwidth]{taxonomy_passerea}}
%\begin{frame}[b,plain]
%	\tiny\hfill Jarvis et al. 2014. Science 346: 1320.
%\end{frame}
%}


{
\usebackgroundtemplate{\includegraphics[width=\paperwidth]{common_goldeneye}}
\begin{frame}[b,plain]{\textcolor{white}{Nearly all Anseriformes belong to the Anatidae.}}
	\tiny\textcolor{white}{Common Goldeneye by  Bill Thompson/USFWS.}
\end{frame}
}


\begin{frame}{Simplified phylogeny of Missouri \highlight{Anatidae.}}
\begin{forest} mytree
[
	[, for tree={l=1cm}
		[Whistling-ducks]
		[
			[Stiff-tailed ducks]
			[
				[
					[Geese]
					[Swans]
				]
				[
					[Wood Duck]
					[
						[Mergansers and allies]
						[
							[Dabbling ducks]
							[Diving ducks]
						]
					]
				]
			]
		]
	]
]
\end{forest}

Bill: lamellate or serrate; spatulate, or somewhat laterally compressed to terete. \\
Feet: palmate; hallux is elevated, lobed in some groups.
\end{frame}


\begin{frame}[t]{Whistling-ducks are large ducks with an upright posture.}

\backoneline

\begin{multicols}{2}
%
\includegraphics[width=\linewidth]{taxonomy_whistling_duck_walking}
%
\columnbreak
%
\includegraphics[width=\linewidth]{taxonomy_whistling_duck_flying}
%
\end{multicols}

\backoneline


Black-bellied Whistling-duck: Bright orange bill; pinkish legs and feet dangle extend beyond the tail in flight. 

\vfilll

\tiny \textcopyright\,\href{https://macaulaylibrary.org/asset/35646961}{Shawn Billerman, Macaulay Library} \hfill \textcopyright\,\href{https://macaulaylibrary.org/asset/54166281}{Georges Duriaux, Macaulay Library}

\end{frame}
%

\begin{frame}{Stiff-tailed ducks often hold tail stiffly upright.}

\vspace{-\baselineskip}

\begin{multicols}{2}
\includegraphics[width=\linewidth]{taxonomy_ruddy_duck}

\columnbreak

Ruddy Duck: note bright blue bill and white mask of the male (upper).

\end{multicols}

\vfilll

\tiny \href{https://www.flickr.com/photos/49208525@N08/14409068487}{U\textsc{sfws}–Midwest Region, Public Domain} 
\end{frame}


\begin{frame}{Geese have somewhat compressed bills at the base, becoming spatulate near tip.}

%\backoneline

\includegraphics[width=\textwidth]{taxonomy_geese}

Canada Goose has black neck and head with white chinstrap.

Snow Goose has white or blue phases (color forms). Note black “grin” on bill.

Greater White-fronted Goose has white on face at base of bill.

\vfilll

\tiny Canada Goose \href{https://commons.wikimedia.org/wiki/File:Canada_goose_(Branta_canadensis),_Lake_Victoria,_Christchurch,_New_Zealand_04.jpg}{Michal Klajban, \ccbysa{4.0}} 
\hfill 
Snow Goose \href{https://macaulaylibrary.org/asset/47881261}{\textcopyright\,Charles Hundertmark, Macaulay Library}
\hfill
Greater White-fronted Goose \href{https://macaulaylibrary.org/asset/36909271}{\textcopyright\,Ryan Schain, Macaulay Library}


\end{frame}

\begin{frame}{Swans are very large, all white waterfowl.}

\vspace{-0.5\baselineskip}
\includegraphics[width=\textwidth]{taxonomy_trumpeter_swan}

\includegraphics[width=\textwidth]{taxonomy_swan_comparison}

\vfilll

\tiny Trumpeter Swan (top), \href{https://www.fws.gov/species/trumpeter-swan-cygnus-buccinator}{Trumpeter Swan, \textsc{usfws,} public domain} 
\hfill Comparison: \href{https://finwr.org/which-swan-is-which/}{Sue Barth, Celeste Morien}

\end{frame}

\begin{frame}{Wood Duck is a small duck found in wooded swamps or other water bodies with nearby trees.}
\includegraphics[width=\textwidth]{taxonomy_wood_duck}

Wood Ducks have clawed toes; they can climb and perch in trees.
\end{frame}

{
\usebackgroundtemplate{\includegraphics[width=\paperwidth]{taxonomy_mergansers}}
\begin{frame}



\vfilll

\tiny \hfill
Hooded Merganser \href{https://macaulaylibrary.org/asset/36549821}{\textcopyright\,Ryan Schain, Macaulay Library}\\ \hfill
Red-breasted Merganser \href{https://macaulaylibrary.org/asset/23444211}{\textcopyright\,Ian Davies, Macaulay Library} \\ \hfill
Common Merganser \href{https://macaulaylibrary.org/asset/51564761}{\textcopyright\,Alix d'Entremont, Macaulay Library}
\end{frame}
}


\begin{frame}[t]{Dabbling ducks dabble.}
\vspace{-0.5\baselineskip}

\centering
\includegraphics[height=0.82\textheight]{taxonomy_dabbling_ducks}


\vfilll
\tiny \href{https://www.youtube.com/watch?v=UaW66BVuZgM}{Link to video} \hfill \textcopyright\,Suan Young
\end{frame}


\begin{frame}{Some common Missouri dabbling ducks.}
\includegraphics[width=\textwidth]{taxonomy_dabblers}

Dabbling ducks have an \textit{unlobed} hallux. Legs are closer to mid-body.

\vfilll

\tinyfill All photos from Macaulay Library.
\end{frame}


\begin{frame}{Diving ducks dive.}
\includegraphics[width=\textwidth]{taxonomy_divers}



Diving ducks have a \textit{lobed} hallux. Legs are closer to tail.

\vfilll

\tiny \href{https://www.youtube.com/watch?v=ncNBfmVMCkI}{Link to video} \hfill All photos from Macaulay Library.
\end{frame}


{
\usebackgroundtemplate{\includegraphics[width=\paperwidth]{galloanseres_phylogeny}}
\begin{frame}[b,plain]
	\tiny Kriegs et al. 2007. BMC Evolutionary Biology 7: 190.
\end{frame}
}

%\lecture{student}{student}
%
%\begin{frame}{Superorder Galloanseres has two orders: \highlight{Anseriformes} and \highlight{Galliformes.}}
%
%
%\begin{forest} mytree
%[, l sep=+1.7cm, edge label = {node [xshift=-0.8cm, text width = 1.5cm] {\footnotesize Subclass Neornithes}}
%	[,name=neognathae, edge label = {node [text width=2cm, midway, xshift=1.1cm] {\footnotesize Infraclass Neognathae}}
%		[, name=neoaves, edge label = {node [text width=2cm, midway, xshift=1.1cm] {\footnotesize Superorder Neoaves}}
%			[,l-=1cm
%				[Clade\\ Passerea, align=left]
%				[Clade\\ Columbimorphae, align=left]
%			]
%			[Clade\\ Mirandornithes, align=left]
%		]
%		[, name=galloanseres, edge label = {node [text width=2cm, midway, xshift=1.1cm] {\footnotesize \highlight{Superorder Galloanseres}}}
%			[\sshighlight{Order\\ Galliformes}, align=left]
%			[\shortstack{Order\\ Anseriformes}, align=left]
%		]
%	]
%]
%\end{forest}
%
%\end{frame}

\begin{frame}{Galliformes has two families in Missouri: \highlight{Odontophoridae} and \highlight{Phasianidae.}}

\backoneline

\begin{multicols}{2}
Bills are stout, slightly decurved, and slightly hooked.

\vspace{\baselineskip}

Nostrils are slightly feathered or exposed.

\vspace{\baselineskip}

Wings are short and rounded.

\columnbreak


\includegraphics[width=\linewidth]{taxonomy_wild_turkey}

\end{multicols}

\vfilll

\tinyfill Wild Turkey, \href{https://macaulaylibrary.org/asset/536973431}{Arun Christopher Manoharan, Macaulay Library}
\end{frame}

{
\usebackgroundtemplate{\includegraphics[width=\paperwidth]{bobwhite_quail}}
\begin{frame}[t,plain]

%	\vspace{1em}
	\hangpara\Large\textcolor{white}{Odontophoridae contain the quail. \\ They are small birds with\\short, rounded tails.}
	
	\vskip0pt plus 1filll
	
	\tiny\textcolor{white}{Bobwhite Quail by Les Howard, Flickr Creative Commons.}
\end{frame}
}


{
\usebackgroundtemplate{\includegraphics[width=\paperwidth]{greater_prairie_chicken}}
\begin{frame}[b,plain]


	\tiny\textcolor{white}{Greater Prairie Chicken by Tony Ifland/USFWS.}
\end{frame}
}



\end{document}
